%#!platex sample.tex; dvipdfmx sample; open sample.pdf
\documentclass[a4j,12pt,twocolumn]{jsarticle}
\usepackage[margin=2cm]{geometry}
\pagestyle{empty}
\usepackage[dvipdfmx]{color}
\color[cmyk]{0,0,0,.5}
\makeatletter
  \def\@maketitle{%
    \newpage\null\vspace*{-5zw}
    \begin{center}%
      \let\footnote\thanks
      {\LARGE \@title \par}%
      \vskip 1.5em
      {\large
        \lineskip .5em
        \begin{tabular}[t]{c}%
          \@author
        \end{tabular}\par}%
      \vskip 1em
      {\large \@date}%
    \end{center}%
    \par\vskip 1.5em
    \ifvoid\@abstractbox\else\centerline{\box\@abstractbox}\vskip1.5em\fi
  }
\makeatother
\title{\LaTeXe 入門}% 題名
\author{A. U. Th\'or}%      著者
\date{\today}%            日付
\begin{document}%         本文
\onecolumn
\maketitle%               表紙
\tableofcontents%         目次
%
\section{節見出し}%       節見出し
節見出しは \verb|\section| コマンドを使います。
%
\subsection{小節見出し}%  小節見出し
小節見出しは \verb|\subsection| を使います。
%
\section{文章の記述}
この節では文章の記述について論じます。
%
\subsection{引用}
一文を引用する場合はカギ括弧を使います。一説によると
「カギ括弧は引用に使う」と言われている。
段落ごと引用するということは次のようになっている。
\begin{quote}
一つの段落の引用の場合は \verb|quote| 環境を使い、\emph{行頭を字下げしな
い}のが普通である。複数段落の引用の場合は \verb|quotation| 環境を使い、
行頭を字下げする。
\end{quote}
%
\subsection{箇条書き}
箇条書きには以下の三つが用意されている。
\begin{description}
 \item[記号付箇条書き] ラベルの先頭に記号がついた箇条書き。
 \item[番号付箇条書き] ラベルの先頭に番号がついた箇条書き。
 \item[説明付箇条書き] ラベルの先頭に説明がついた箇条書き。
\end{description}
%
\end{document}	
