\documentclass[twocolumn,papersize]{jsarticle}
\columnseprule 0.5pt% 段間の罫線
\usepackage{type1cm}
\usepackage{epic,eepic,amssymb,amsmath,graphicx,url}
\title{2段組での中間報告のサンプル}
\author{{\small システム情報科学部 情報アーキテクチャ学科}\\
m1201234 函館 花子 \\ 指導教員 未来 太郎}
\date{\today} % \today 命令は文書を作成した日付が代入される
% 本文開始
\begin{document}
\begin{abstract}% 概要
論文作成においては\LaTeX{}を使用するのが望ましいが,近年では事務処理用の
Wordがその代わりとなっているように見受けられる.今回は, はこだて未来大学
においてどの程度Wordや\LaTeX{}が浸透しているのかを2003年度の卒業研究から
提出される中間レポートを参考に統計を取ってみた.結果は予想通りWord人口が
圧倒的に多かった.また,この中間報告のサンプルの内容は出たら目であるので,
あくまで入力例として参考にしてもらいたい.
\end{abstract}
\maketitle% 表題

\section{目的}
当大学では卒業研究の中間報告として中間レポートを提出するようになってい
る.各自がどのようなアプリケーションを使っているのかを調査することが今
回の目的である.

\section{方法}
直接研究生にアンケートをとったわけではなく,ウェブページ上で2003年9月
10日までに提出されているレポートを調査対象とした.

\section{結果}
提出されているレポートを大まかに調査した結果が表~\ref{2bansenji}となる.
これは研究生がどのようなアプリケーションで中間レポートを作成したのかを
調べた結果である.どうしても判別できないものは\emph{その他}の項目に入れ
てある.レポートの最終形態ではなく,原稿を作成する段階で使ったアプリケ
ーションを示している.
\begin{table}[htbp]
 \begin{center}
  \caption{データの集計結果}\label{2bansenji}
  \begin{tabular}{lrr}
   \hline
   項目        & 人数 (人)& 割合 (\%)  \\
   \hline
   Word        &  75 & 45.2  \\
   \LaTeX{}    &  26 & 15.6  \\
   HTML        &  54 & 32.5  \\
   Illustrator &   4 &  2.4  \\
   OpenOffice  &   1 &  0.6  \\
   その他      &   6 &  3.0  \\\hline
   合計        & 166 &  100  \\\hline
  \end{tabular}
 \end{center}
\end{table}
これらの結果は二次的に入手した情報のため,データに若干の誤りがある.直
接アンケートをとって調べればもっと正確な情報が収集できるが,今回は簡易
的な形をとった.

\section{考察}
以上の結果から,現在HTMLで作成している人物はWordを使う事になるだろう.
結果があくまで中間報告である事を考えれば,Word人口がこれから増えること
は明白である.今度の働きかけ次第で当大学の\LaTeX{}人口を増加させること
も可能である.

この現象を天下り的にフーリエ変換で解析する.まず,フーリエ変換で関数
$f(x)$を定義する.この関数$f(x)$は変換のための区間を必要とするので,
区間を$[-L,L]$とする.すると以下の式が定義から導出される.
\begin{eqnarray*}
f(x)& = & \frac{a_0}{2} + \sum^{\infty}_{n=1} \left( a_n \cos 
          \frac{n\pi x}{L} + b_n \sin \frac{n\pi x}{L} \right) \\
a_n & = & \frac{1}{L} \int^{L}_{-L} f(u) \cos \frac{n\pi u}{L} du\\
b_n & = & \frac{1}{L} \int^{L}_{-L} f(u) \sin \frac{n\pi u}{L} du
\end{eqnarray*}
よって,次式~(\ref{eq:fourier1})が新たに得られる.
\begin{eqnarray}
f(x) & = & \frac{1}{2L} \int^{L}_{-L} f(u) du \nonumber\\
     & + & \sum^{\infty}_{n=1} \left[ \frac{1}{L} \int^{L}_{-L}
           f(u) \cos \frac{n\pi x}{L} du \cdot \cos \frac{n\pi x}{L}
           \right. \nonumber \\
     & + & \left. \frac{1}{L} \int^{L}_{-L} f(u) \sin 
           \frac{n\pi u }{L}du \cdot \sin \frac{n\pi x}{L} \right]
           \label{eq:fourier1}
\end{eqnarray}
式~(\ref{eq:fourier1})を\(L\rightarrow\infty\)にしたりしてフーリエ変
換は一般に式~(\ref{eq:fourier2})のように書き表すことができる.
\begin{equation}
F(\alpha )= \frac{1}{\sqrt{2\pi}} \int^{\infty}_{-\infty} 
            f(u) e^{-t\alpha u}du \label{eq:fourier2}
\end{equation}
式~(\ref{eq:fourier2})を使って今回の結果を解析することは,現段階では非
常に困難であると容易に考察できる.
\section{今後の展望}
今回得られた調査結果を下にGnuplotでデータをプロットする作業が続くもの
と思われる.また,グラフは主にGnuplotから挿入するのが望ましいとされる.
Gnuplotから挿入したグラフは図~\ref{fig:sample}となる.
\begin{figure}[htbp]
\begin{center}
%\input{abstgnu.tex} %  ファイルからの読み込み
\fbox{\rule{0pt}{3zw}\rule{3zw}{0pt}}
\caption{picture環境で描画した図形}\label{fig:sample}
\end{center}
\end{figure}
\nocite{*}
\begin{thebibliography}{10}%参考文献
\bibitem{latexcompanion}
   Michel Goossens, Frank Mittelbach, and Alexander Samarin.
   The \LaTeX コンパニオン.
   東京アスキー, 1998.
\bibitem{latexgraphics}
   Michel Goossens, Sebastian Rahtz, and Frank Mittelbach.
   \LaTeX グラフィックスコンパニオン.
   株式会社アスキー, 2000.
\bibitem{bibunsyo}
   奥村晴彦.
   [改訂第3版] {\LaTeXe} 美文書作成入門.
   技術評論社, 2004.
\bibitem{platex2e}
   乙部厳己, 江口庄英.
   {\em {p\LaTeXe} for Windows Another Manual Vol.1 Basic Kit 1999}.
   ソフトバンク, 1998.
\bibitem{linuxthesis}
   臼田昭司, 伊藤敏, 井上祥史.
   Linux論文作成術.
   オーム社, 1999.
\bibitem{metafont}
   Donald~E. Knuth.
   \textsf{METAFONT} ブック.
   アスキー, 1994.
\bibitem{jtexbook}
   Donald~E. Knuth.
   改訂新版{\TeX}ブック.
   アスキー出版局, 1992.
\end{thebibliography}
\end{document}