%#!platex -kanji=utf8 hb.tex
\chapter{記号}
% これはただのダミーテキスト.
% 文字コードを判定するための意味のない文字列.
% これくらい記述すれば大丈夫かな.
% 付録のくせに生意気な.
% 付録の分際で.

\begingroup

\makeatletter
\renewenvironment{table}[1][htbp]%
   {\vskip 0pt plus 1pt minus 1pt
    \def\@captype{table}\vbox\bgroup\small}%
   {\egroup\vskip 0pt plus 1pt minus 1pt}

% 文字 (Text) 記号
\newcommand*{\T}[1]{%
   \glossary{#1@\hspace*{-1.2ex}\texttt{\protect\BS\string#1}%
\hskip1em(\protect\mynameuse{#1})}%
   \csname#1\endcsname&\texttt{\BS\string#1}}
% 数学 (Math) 記号

\newcommand*{\m}[1]{$#1$&\texttt{\string#1}}

\newcommand*{\M}[1]{%
   \glossary{#1@\hspace*{-1.2ex}\texttt{\protect\BS#1}%
\hskip1em($\protect\csname #1\endcsname$)}%
   $\csname #1\endcsname$&\texttt{\BS#1}}
% 
\newcommand*{\B}[2]{%
   \glossary{#1@\hspace*{-1.2ex}\texttt{\protect\BS#1}%
      \hskip1em(\csname#1\endcsname{#2}\relax)}%
   \csname#1\endcsname{#2}&%
    \texttt{\protect\BS\string#1\string{#2\string}}}

\newcommand*{\BM}[1]{%
   \glossary{#1@\hspace*{-1.2ex}\texttt{\protect\BS#1}%
\hskip1em($\protect\expandafter\protect\big\protect\csname #1\endcsname$)}%
   \texttt{\BS#1}}

\makeatother

%#!platex -kanji=utf8 hb.tex
\section{文字記号}

\begin{table}[htbp]
 \centering
 \glossary{"#@\hspace*{-1.2ex}\verb+"\#+}%"
 \glossary{"_@\hspace*{-1.2ex}\verb+"\_+}%"
 \glossary{"$@\hspace*{-1.2ex}\verb+"\$+}%"
 \glossary{"%@\hspace*{-1.2ex}\verb+"\%+}%"
 \glossary{"{1"}@\hspace*{-1.2ex}\protect\bgroup\verb+"\"{+"}}%
 \glossary{"{2"}@\hspace*{-1.2ex}"{\verb+"\"}+\protect\egroup}%
 \glossary{"&@"\hspace*{-1.2ex}"\verb+"\"&+}%"
 \index{ナンバー}%
 \index{ドル}%
 \index{パーセント}%
 \index{アンパサンド}%
 \index{波括弧}%
 \index{チルダ}%
 \index{ハット}%
 \index{バックスラッシュ}%
 \index{縦棒}%
 \index{小なり}%
 \index{大なり}
 \caption{アスキー文字}\tablab{アスキー文字}
 \begin{tabular}{*{2}{l@{\thickspace}l}}
  \hline
  \# & \verb|\#| & \T{textasciitilde}\\
  \$ & \verb|\$| & \T{textasciicircum}\\
  \% & \verb|\%| & \T{textbackslash}\\
  \& & \verb|\&| & \T{textbar}\\
  \_ & \verb|\_| & \T{textless}\\
  \{ & \verb|\{| & \T{textgreater}\\
  \} & \verb|\}| \\  
  \hline
 \end{tabular} 
\end{table}

\LaTeX では10個の半角記号(\W{アスキー文字})は\W{特殊な文字}として
解釈されてしまうため,面倒でも\tabref{アスキー文字}のコマンドを
用いる必要があります.
\index{"#@\verb+#+}\glossary{"#@\verb+#+}%
\index{"$@\verb+$+}\glossary{"$@\verb+$+}%
\index{"&@\verb+&+}\glossary{"&@\verb+&+}%
\index{"\@\verb+\+}\glossary{"\@\verb+\+}%
\index{"^@\verb+^+}\glossary{"^@\verb+^+}%
\index{"_@\verb+_+}\glossary{"_@\verb+_+}%
\index{"{1"}@\protect\bgroup\verb+"{+"}}%
\glossary{"{1"}@\protect\bgroup\verb+"{+"}}%
\index{"{2"}@"{\verb+"}+\protect\egroup}%
\glossary{"{2"}@"{\verb+"}+\protect\egroup}%
\index{"~@\verb+~+}%
\glossary{"~@\verb+~+}%
\index{%@\verb+%+}%"
\glossary{%@\verb+%+}%"

%\begin{quote}
%\verb|# $ % & _ { } ~ ^ \ |
%\end{quote}

さらに3個の記号は出力が違う文字記号になります.%"
\index{"|@\texttt{\symbol{'174}}}%"
\index{"<@\verb+<+}%"
\index{">@\verb+>+}%"
%\begin{quote}
\str| \str< \str> はそれぞれ --- !` ?`
%\end{quote}
として表示されてしまいます.


\begin{table}[htbp] \centering
\index{見える空白}%
\index{ダガー}\index{短剣符}%
\index{ダブルダガー}\index{二重短剣符}%
\index{セクション}\index{章標}%
\index{パラグラフ}\index{段標}%
\index{たんらくきこう@段落記号\hskip1em(\P)}\indindz{記号}{段落}%
\index{せつきこう@節記号\hskip1em(\S)}\indindz{記号}{節}%
\glossary{"!@\verb+"!`+\hskip1em("!`)}%"
\glossary{?@\verb+?`+\hskip1em(?`)}%
\index{シャープS}%
\index{スラッシュ付きO}%
\index{伏字L}%
\index{合字!OEの\zdash}%
\index{合字!AEの\zdash}%
\index{ポンド}%
\index{リングA}%
\indindz{記号}{特殊な}%
\zindind{点}{のないi}\zindind{点}{のないj}%
\indindz{記号}{特殊な}%
\caption{特殊な文字記号}\tablab{app:tokusyukigou}%
\begin{tabular}{*{5}{l@{\thickspace}l}}
 \hline
\T{aa} & \T{AA} & \T{ae} & \T{AE} &\T{oe}\\
\T{OE} & \T{l} & \T{L} & \T{o} & \T{O}\\
\T{i}  & \T{j} & \T{ss} & \T{SS} & \T{S}\\
\T{P}  & \T{dag} & \T{ddag} & \T{pounds} & !`&\verb|!`|  \\
 {?`}&\verb|?`| & \\ 
\end{tabular}\\
\begin{tabular}{*{2}{l@{\thickspace}l}}
% \hline
\T{textvisiblespace} & \T{copyright}\\
\T{textregistered} & \T{texttrademark}\\
 \hline
\end{tabular}
\end{table}

\begin{table}
 \begin{center}
  \caption{その他の文字記号}
  \begin{tabular}{*{2}{l@{\thickspace}l}}
   \hline
   \T{textbraceleft}  &
   \T{textbraceright}\\
   \T{textunderscore}&
   \T{textendash}\\
   \T{textemdash}&
   \T{textellipsis}\\
   \T{textquoteleft}&
   \T{textquoteright}\\
   \T{textquotedblleft}&
   \T{textquotedblright}\\
   \T{textquestiondown}&
   \T{textexclamdown}\\
   \hline
  \end{tabular}   
 \end{center}
\end{table}

\tabref{T1エンコーディングで使用できる文字記号}の記号は\Y{fontenc}
パッケージを\qu{\Option{T1}}というオプション付きで読み込むと出力できます.
\begin{usage}
\usepackage[T1]{fontenc} 
\end{usage}
このとき,\sty{pxfonts}や\sty{txfonts},\Y{lmodern},\Y{type1ec}パッ
ケージを読み込むとアウトラインフォントがPDFに埋め込まれるようになります.

\begin{table}  \centering
 \caption{\option{T1}エンコーディングで使用できる文字記号}
 \tablab{T1エンコーディングで使用できる文字記号}
\begin{tabular}{*{2}{l@{\thickspace}l}}
 \hline
 \T{DJ} &\T{guillemotleft}\\
 \T{ng} &\T{guillemotright}\\
 \T{NG} &\T{guilsinglleft}\\
 \T{th} &\T{guilsinglright}\\
 \T{TH} &\T{quotedblbase}\\
 \T{dh} &\T{quotesinglbase}\\
 \T{DH} &\T{textquotedbl}\\
 \T{dj} \\
 \hline
\end{tabular}
\end{table}

\begin{table}
 \caption{ダイアクリティカルマーク(アクセント)}%
 \tablab{ダイアクリティカルマーク}%
 \begin{center}
  \index{ダイアクリティカルマーク}%
  \indindz{記号}{アクセント}\index{アクセント記号}%
  \indindz{アクセント}{文中の}%
  \makeatletter
  \glossary{""@\hspace*{-1.2ex}\verb+\""+\hskip1em(\""u)}%"
  \glossary{"~@\hspace*{-1.2ex}\verb+\~+\hskip1em(\~n)}%"
  \glossary{"^@\hspace*{-1.2ex}\verb+\^+\hskip1em(\^o)}%"
 \newcommand*\A[4][]{%
   \glossary{#2@\hspace*{-1.2ex}\texttt{\protect\BS#2}%
      \hskip1em(\csname#2\endcsname{#3}\relax)}%
  \index{#1}%
   #1&\texttt{\BS\string #2}&\csname#2\endcsname{#3}&%
   \texttt{\BS\string#2\@charlb#3\@charrb}&{#4}%
  }%
 \newcommand*\NA[4][]{%
  \index{#1}%
   #1&\texttt{\BS\string #2}&\csname#2\endcsname{#3}&%
   \texttt{\BS\string#2\@charlb#3\@charrb}&{#4}%
  }%
 \makeatother
  \index{acute}\index{breve}\index{circumflex}\index{cedilla}%
  \index{hungarumlaut}\index{double acute}\index{grave}\index{caron}%
  \index{macron}\index{dot accent}\index{ring}\index{tie}%
  \index{tilde}\index{umlaut}\index{subscript dot}\index{under dot}%
  \index{underscore}%
  \index{ogonek}%
 \begin{tabular}{lllll}
  \toprule
  名称 & 命令 & 出力例 & 入力例 & 別称 \\
  \midrule
  \A[アキュート]         {'}{a} {\W{揚音符}}\\%
  \A[ブレーヴェ]         {u}{u} {\W{短音府}}\\% 
  \NA[サーカムフレックス]{^}{a} {\W{抑揚音符}}\\% これは noindex
  \A[セディーユ]         {c}{C} {\W{鈎形符}}\\ %
  \A[ダブルアキュート]   {H}{o} {}\\% %長短音符?
  \A[グレイヴ]           {`}{a} {\W{抑音符}}\\% 
  \A[ハーチェク]         {v}{a} {\W{キャロン}}\\% 
  \A[マクロン]           {=}{e} {\W{長音符}}\\% 
  \A[ドット]             {.}{a} {}\\% 
  \A[リング]             {r}{o} {}\\% 
  \A[タイ]               {t}{oo}{}\\% 
  \maketildeletter
  \NA[チルダ]            {~}{o} {\W{波音符}}\\%  これは noindex
  \maketildeother
  \NA[ウムラウト]        {"}{a} {\W{分音符}}\\%" これは noindex
  \A[下付きドット]       {d}{t} {}\\% 
  \A[下線]               {b}{z} {}\\% 
  \midrule
  \A[点なしj]            {j}{} {}\\% 
  \A[点なしi]            {i}{} {}\\% 
  \midrule
  \A[オゴネク*]          {k}{c} {}\\ % T1 のみ
  \bottomrule
 \end{tabular}
 \end{center}
\end{table}
%
\tabref{ダイアクリティカルマーク}のオゴネクはT1エンコーディングで出力可
能です. \tabref{T1エンコーディングで使用できる文字記号}の説明を参照して
ください.
\begin{inout}
J\"org mu\ss\ ein Gel\"ande f\"ur seine Fabrik erwerben.
\end{inout}

\subsection{\sty{pifont}}
\begin{table}[htbp]
\begin{small}
 \makeatletter
 \newcommand*\fntsymb[1]{{\usefont{U}{pzd}{m}{n}\char#1}}
 \newcommand*\fnttbl[1]{%
   \fntsymb{'#10} & \fntsymb{'#11} &  \fntsymb{'#12} & \fntsymb{'#13} &%
   \fntsymb{'#14} & \fntsymb{'#15} &  \fntsymb{'#16} & \fntsymb{'#17} &%
 }
 \makeatother
 \begin{center}
  \caption{\sty{pifont} (\W{ZapDingbats}) 中の記号一覧}\tablab{ZapDingbats}
  \begin{tabular}{c|*8c|c}
   \textit{x}
   & \textit{'0} & \textit{'1} & \textit{'2} & \textit{'3}
   & \textit{'4} & \textit{'5} & \textit{'6} & \textit{'7}
   &  \\ \hline
   \textit{'04x} & \fnttbl{04} \texttt{"2x} \\ \cline{1-9}
   \textit{'05x} & \fnttbl{05} \texttt{"2y} \\ \hline
   \textit{'06x} & \fnttbl{06} \texttt{"3x} \\ \cline{1-9}
   \textit{'07x} & \fnttbl{07} \texttt{"3y} \\ \hline
   \textit{'10x} & \fnttbl{10} \texttt{"4x} \\ \cline{1-9}
   \textit{'11x} & \fnttbl{11} \texttt{"4y} \\ \hline
   \textit{'12x} & \fnttbl{12} \texttt{"5x} \\ \cline{1-9}
   \textit{'13x} & \fnttbl{13} \texttt{"5y} \\ \hline
   \textit{'14x} & \fnttbl{14} \texttt{"6x} \\ \cline{1-9}
   \textit{'15x} & \fnttbl{15} \texttt{"6y} \\ \hline
   \textit{'16x} & \fnttbl{16} \texttt{"7x} \\ \cline{1-9}
   \textit{'17x} & \fnttbl{17} \texttt{"7y}\\
   \hline
   \multicolumn{10}{c}{}\\
   \hline
   \textit{'24x} & \fnttbl{24} \texttt{"Ax} \\ \cline{1-9}
   \textit{'25x} & \fnttbl{25} \texttt{"Ay}\\\hline
   \textit{'26x} & \fnttbl{26} \texttt{"Bx} \\ \cline{1-9}
   \textit{'27x} & \fnttbl{27} \texttt{"By}\\\hline
   \textit{'30x} & \fnttbl{30} \texttt{"Cx} \\ \cline{1-9}
   \textit{'31x} & \fnttbl{31} \texttt{"Cy}\\\hline
   \textit{'32x} & \fnttbl{32} \texttt{"Dx} \\ \cline{1-9}
   \textit{'33x} & \fnttbl{33} \texttt{"Dy}\\\hline
   \textit{'34x} & \fnttbl{34} \texttt{"Ex} \\ \cline{1-9}
   \textit{'35x} & \fnttbl{35} \texttt{"Ey}\\\hline
   \textit{'36x} & \fnttbl{36} \texttt{"Fx} \\ \cline{1-9}
   \textit{'37x} & \fnttbl{37} \texttt{"Fy}\\\hline
   & \texttt{"8} & \texttt{"9} & \texttt{"A} & \texttt{"B}
   & \texttt{"C} & \texttt{"D} & \texttt{"E} & \texttt{"F}
   & \texttt{y}
  \end{tabular}
 \end{center}
\end{small}
\end{table}
\Y{pifont}パッケージに含まれる記号を使うには次のようにします.
\begin{usage}
 \usepackage{pifont}
 \ding{$\<文字コード>$}
\end{usage}
\val{文字コード}は10進数,8進数,16進数で指定可能です.
\tabref{ZapDingbats}には左側に8進数,右側に16進数の数値を示してあります.
飛行機の記号を出力するために,
10進数では40,8進数では\textit{'050},16進数では \verb|"28| となっていますので,%"
次のようにします.
\begin{inout}
\ding{40} $=$ \ding{'050} $=$ \ding{"28}%"
\end{inout}

\begin{usage}
\dingfill{$\<文字コード>$}% 記号で1行の残りの部分を埋める
\dingline{$\<文字コード>$}% 記号で1行全部を埋める
\end{usage}

\begin{inout}
\dingfill{'044} ※ここから切り取ってください.
\dingline{'134} 
\end{inout}

\begin{usage}
\begin{dinglist}{$\<項目>$}% itemize と似た機能です
 \item $\<項目>$
\end{dinglist}

\begin{dingautolist}{$\<項目>$}% enumerate と似た機能です
 \item $\<項目>$
\end{dingautolist}
\end{usage}


\subsection{\Y{textcomp}}
\begin{usage}
\usepackage[T1]{fontenc}
\usepackage{textcomp}
\usepackage{mathcomp}% 数式中で使う時 
% \textleaf であれば \tcleaf のように短い名前(\tc)で参照
\end{usage}

%\tablab{app:textcomp}
\begingroup%{table}[htbp]\index{著作権記号}
%\newcommand*\TC[1]{\mynameuse{#1}&\C{#1}}%
\newcommand*\TCS[2]{\T{#1}&\T{#2}\\}
\begin{longtable}{*{2}{l@{\thickspace}l}}
  \caption{\textsf{textcomp}で使える記号}\\
 \hline
 \endfirsthead
 \hline
 \multicolumn{4}{c}{前ページからの続きです}\\
 \endhead
 \multicolumn{4}{c}{次ページへ続きます}\\
 \hline
 \endfoot
 \hline
 \endlastfoot
  \TCS{textquotestraightbase}{textquotestraightdblbase}
  \TCS{texttwelveudash}{textthreequartersemdash}
  \TCS{textleftarrow}{textrightarrow}
  \TCS{textblank}{textdollar}
  \TCS{textquotesingle}{textasteriskcentered}
  \TCS{textdblhyphen}{textfractionsolidus}
  \TCS{textzerooldstyle}{textoneoldstyle}
  \TCS{texttwooldstyle}{textthreeoldstyle}
  \TCS{textfouroldstyle}{textfiveoldstyle}
  \TCS{textsixoldstyle}{textsevenoldstyle}
  \TCS{texteightoldstyle}{textnineoldstyle}
  \TCS{textlangle}{textminus}
  \TCS{textrangle}{textmho}
  \TCS{textbigcircle}{textohm}
  \TCS{textlbrackdbl}{textrbrackdbl}
  \TCS{textuparrow}{textdownarrow}
  \TCS{textasciigrave}{textborn}
  \TCS{textdivorced}{textdied}
  \TCS{textleaf}{textmarried}
  \TCS{textmusicalnote}{texttildelow}
  \TCS{textdblhyphenchar}{textasciibreve}
  \TCS{textasciicaron}{textgravedbl}
  \TCS{textacutedbl}{textdagger}
  \TCS{textdaggerdbl}{textbardbl}
  \TCS{textperthousand}{textbullet}
  \TCS{textcelsius}{textdollaroldstyle}
  \TCS{textcentoldstyle}{textflorin}
  \TCS{textcolonmonetary}{textwon}
  \TCS{textnaira}{textguarani}
  \TCS{textpeso}{textlira}
  \TCS{textrecipe}{textinterrobang}
  \TCS{textinterrobangdown}{textdong}
  \TCS{texttrademark}{textpertenthousand}
  \TCS{textpilcrow}{textbaht}
  \TCS{textnumero}{textdiscount}
  \TCS{textestimated}{textopenbullet}
  \TCS{textservicemark}{textlquill}
  \TCS{textrquill}{textcent}
  \TCS{textsterling}{textcurrency}
  \TCS{textyen}{textbrokenbar}
  \TCS{textsection}{textasciidieresis}
  \TCS{textcopyright}{textordfeminine}
  \TCS{textcopyleft}{textlnot}
  \TCS{textcircledP}{textregistered}
  \TCS{textasciimacron}{textdegree}
  \TCS{textpm}{texttwosuperior}
  \TCS{textthreesuperior}{textasciiacute}
  \TCS{textmu}{textparagraph}
  \TCS{textperiodcentered}{textreferencemark}
  \TCS{textonesuperior}{textordmasculine}
  \TCS{textsurd}{textonequarter}
  \TCS{textonehalf}{textthreequarters}
  \TCS{texteuro}{texttimes}
  \T{textdiv}
\end{longtable}
\endgroup



%・windingbats

%\section{ほげ}

%    文字記号
\section{数学記号}

\begin{table}[htbp]
\begin{center}
\caption{ギリシャ小文字}\tablab{app:greek:lower}
\begin{tabular}{*{4}{c@{\thickspace\thinspace}l}}
 \hline
\M{alpha}   & \M{eta}    & \M{nu}    & \M{tau}     \\
\M{beta}    & \M{theta}  & \M{xi}    & \M{upsilon} \\
\M{gamma}   & \M{iota}   & $o$&o     & \M{phi}     \\
\M{delta}   & \M{kappa}  & \M{pi}    & \M{chi}     \\
\M{epsilon} & \M{lambda} & \M{rho}   & \M{psi}     \\
\M{zeta}    & \M{mu}     & \M{sigma} & \M{omega}   \\
  \hline
\end{tabular}
\end{center}
\end{table}

\begin{table}[htbp]
\begin{center}
 \makeatletter
 \newcommand*\LG[1]{\C{mathrm}\texttt{\@charlb#1\@charrb}}%
 \newcommand*\LGS[1]{$\mathrm{#1}$&\LG{#1}}%
 \makeatother
\caption{ギリシャ大文字}\tablab{app:greek:upper}
\begin{tabular}{*{4}{c@{\thickspace}l}}
 \hline
 \LGS{A}   & \LGS{H}    & \LGS{N}   & \LGS{T}\\
 \LGS{B}   & \M{Theta}  & \M{Xi}    & \M{Upsilon}\\
 \M{Gamma} & \LGS{I}    & \LGS{O}   & \M{Phi}\\
 \M{Delta} & \LGS{K}    & \M{Pi}    & \LGS{X}\\
 \LGS{E}   & \M{Lambda} & \LGS{P}   & \M{Psi}\\
 \LGS{Z}   & \LGS{M}    & \M{Sigma} & \M{Omega}\\
  \hline
\end{tabular}
\end{center}
\end{table}

%\begin{inout}
%\begin{eqnarray*}
%\cos^2\theta + \sin^2\theta & \neq & \cos^2x + \sin^2x 
%\end{eqnarray*}
%\end{inout}
%

%\begin{inout}
%\begin{eqnarray*}
%     A      & \neq & \mathrm{A}\\
%F(x)+C      & \neq & F(x)+ \mathrm{C}\\
%\mathit{foo}& \neq & \mathrm{foo}
%\end{eqnarray*}
%\end{inout}

\begin{table}[htbp]
 \begin{center}\zindind{ギリシャ文字}{の変体小文字}%
\caption{ギリシャ小文字の変体文字}\tablab{app:greek:lower:hen}
 \begin{tabular}{*{3}{c@{\thickspace\thinspace}l}}
  \hline
 \M{varepsilon} & \M{vartheta} & \M{varpi} \\
 \M{varrho}     & \M{varsigma} & \M{varphi}\\
  \hline
 \end{tabular}
 \end{center}
\end{table}


\begin{table}[htbp]
\begin{center}
\caption{大型演算子}\indindz{演算子}{大型}\index{大型演算子}%
\begin{tabular}{*{4}{c@{\thickspace\thinspace}l}}
 \hline
\M{sum}    & \M{oint}     & \M{bigvee}   & \M{bigoplus}  \\
\M{prod}   & \M{bigcup}   & \M{bigwedge} & \M{bigotimes} \\
\M{coprod} & \M{bigcap}   &    &          & \M{bigodot}  \\
\M{int}    & \M{bigsqcup} &    &          & \M{biguplus} \\
  \hline
\end{tabular}
\end{center}
\end{table}


\begin{table}[htbp]
\begin{center}
\caption{括弧の大きさを指定する例}
\tablab{app:ookiikakko}
\index{"/@"\verb+"/+}%"
\begin{tabular}{*{5}{c@{\thickspace}l}}
 \hline
\m{/}      & \m{(}       & \m{)}       & \m{|}       &
  $\|$      & \verb+\|+\\
\m{\big/}  & \m{\bigl(}  & \m{\bigr)}  & \m{\bigm|}  &
  $\bigm\|$ & \verb+\bigm\|+\\[4pt]
\m{\Big/}  & \m{\Bigl(}  & \m{\Bigr)}  & \m{\Bigm|}  &
  $\Bigm\|$ & \verb+\Bigm\|+\\[5pt]
\m{\bigg/} & \m{\biggl(} & \m{\biggr)} & \m{\biggm|} &
  $\biggm\|$&  \verb+\biggm\|+\\[6pt]
\m{\Bigg/} & \m{\Biggl(} & \m{\Biggr)} & \m{\Biggm|} &
  $\Biggm\|$&  \verb+\Biggm\|+\\[7pt]
 \hline
\end{tabular}
\end{center}
\end{table}

\begin{table}[htbp]
\begin{center}
\caption{主な区切り記号}\tablab{app:brace1}
\glossary{"{1"}@\hspace*{-1.2ex}\protect\bgroup\verb+"\"{+"}}%
\glossary{"{2"}@\hspace*{-1.2ex}"{\verb+"\"}+\protect\egroup}%
\glossary{"|@"\hspace*{-1.2ex}"\verb+"\"|+}%}
\index{"(@\verb+(+}%"
\index{")@\verb+)+}%
\index{"[@\verb+[+}%
\index{"]@\verb+]+}%
\index{"|@\texttt{\symbol{'174}}}%""
\begin{tabular}{*{4}{c@{\thickspace}l}}
 \hline
$($ &\verb+(+ & \M{rfloor}   & \M{updownarrow}& \M{lbrace}\\
$)$ &\verb+)+ & \M{lfloor}   & \M{Uparrow}&     \M{rceil}\\
$[$ &\verb+[+ & \M{arrowvert}& \M{Downarrow}&   \M{lceil}\\
$]$ &\verb+]+ & \M{Arrowvert}& \M{Updownarrow}& 
 $\big\lmoustache$&\BM{lmoustache}~${}^*$\\[2pt]
$\{$&\verb+\{+& \M{Vert}&      \M{backslash}&   
 $\big\rmoustache$&\BM{rmoustache}~${}^*$\\[2pt]
$\}$&\verb+\}+& \M{vert}&      \M{rangle}&      
 $\big\lgroup$&\BM{lgroup}~${}^*$\\[2pt]
$|$ &\verb+|+ & \M{uparrow}&   \M{langle}&      
 $\big\rgroup$&\BM{rgroup}~${}^*$\\[2pt]
$\|$&\verb+\|+& \M{downarrow}& \M{rbrace}&      
 $\big\bracevert$&\BM{bracevert}~${}^*$\\
  \hline
\end{tabular}
\\ {\small${}^{*}$\ 大型の区切り記号です.}
\end{center}
\end{table}
%}}}"

%\begin{inout}
% $<x, y> \neq \langle x, y\rangle$
%\end{inout}

%\begin{inout}
%\begin{displaymath}
%\left( \frac{1}{1+\frac{1}{1+x}} \right) 
%\end{displaymath}
%\end{inout}

%\begin{inout}
%\[ \left\lmoustache \left\{ 
%  \left(\frac{1}{x}+1\right)
%  +\left(\frac{1}{x^2}+2\right) 
%\right\} \right\rmoustache \]
%\end{inout}

%\begin{inout}
%\begin{displaymath}
% \left\uparrow \int f(x)dx 
%   \right\downarrow + \left\lgroup 
%     \int g(x)dx \right\rgroup
%\end{displaymath}
%\end{inout}


\begin{table}[htbp]
\begin{center}%
 \indindz{アクセント}{数式中の}%
 \zindind{ベクトル}{記号}%
 \makeatletter
 \newcommand*{\WA}[2]{%
 \glossary{#1@\hspace*{-1.2ex}\texttt{\protect\BS#1}%
 \hskip1em($\csname#1\endcsname{#2}$)}%
 $\csname#1\endcsname{#2}$ & %
 \texttt{\BS#1\@charlb\string#2\@charrb}}%
 \makeatother
\caption{小さいアクセント}\tablab{app:smallac}
\begin{tabular}{*{4}{c@{\thickspace\thinspace}l}}
 \hline
\WA{hat}{a}  & \WA{check}{a}& \WA{breve}{a}&\WA{acute}{a}\\
\WA{grave}{a}& \WA{tilde}{a}& \WA{bar}{a}  &\WA{dot}{a}  \\
\WA{ddot}{a} & \WA{vec}{a}  &       &      &         &   \\
  \hline
\end{tabular}
\end{center}
\end{table}

%\begin{inout}
%\( \vec{a}+\vec{b}\neq \vec{a+b} 
%   \neq \overrightarrow{a+b} \)
%\end{inout}



\begin{table}[htbp]
\begin{center}
\caption{大きいアクセント}\tablab{app:bigac}
\begin{tabular}{*{2}{c@{\thickspace\thinspace}l}}
 \hline
$\overline{m+M}$      &\C{overline}      & 
  $\overbrace{m+M}$& \C{overbrace}  \rule{0pt}{1.5em}\\
$\underline{m+M}$     &\C{underline}     &
  $\underbrace{m+M}$&  \C{underbrace} \rule{0pt}{1.5em}\\
$\overleftarrow{m+M}$ &\C{overleftarrow} & 
  $\widehat{m+M}$& \C{widehat} \rule{0pt}{1.5em}\\
$\overrightarrow{m+M}$&\C{overrightarrow}& 
  $\widetilde{m+M}$& \C{widetilde}  \rule{0pt}{1.5em}\\
  \hline
\end{tabular}
\end{center}
\end{table}

%\begin{inout}
%\begin{displaymath}
% \overbrace{a+b+c+d+e+f+g}^{h+i+j+k}+
% \underbrace{l+m+n}_{o+p+q}
%\end{displaymath} 
%\end{inout}


\begin{table}[htbp]
\begin{center}
 \caption{主な数学関数}\tablab{app:suugakukannsuu}
 \begin{tabular}{*{4}{c@{\thickspace\thinspace}l}}
  \hline
  \M{arccos} & \M{csc} & \M{ker}    & \M{min}  \\
  \M{arcsin} & \M{deg} & \M{lg}     & \M{Pr}   \\
  \M{arctan} & \M{det} & \M{liminf} & \M{sec}  \\
  \M{arg}    & \M{dim} & \M{limsup} & \M{sin}  \\
  \M{cos}    & \M{exp} & \M{lim}    & \M{sinh} \\
  \M{cosh}   & \M{gcd} & \M{ln}     & \M{sup}  \\
  \M{cot}    & \M{hom} & \M{log}    & \M{tan}  \\
  \M{coth}   & \M{inf} & \M{max}    & \M{tanh} \\
  \hline
 \end{tabular}
\end{center}
\end{table}

%\begin{inout}
%\[cos^2x+sin^2x \neq \cos^2x+\sin^2x\]
%\end{inout}

\begin{table}[htbp]
\begin{center}
\caption{関係子}\tablab{app:kannkeisi}
\begin{tabular}{*{4}{c@{\thickspace\thinspace}l}}
 \hline
\M{le}         & \M{in}        & \M{sqsupseteq} & \M{neq}    \\
\M{prec}       & \M{notin}     & \M{dashv}      & \M{doteq}  \\
\M{preceq}     & \M{ge}        & \M{ni}         & \M{propto} \\
\M{ll}         & \M{succ}      & \M{equiv}      & \M{models} \\
\M{subset}     & \M{succeq}    & \M{sim}        & \M{perp}   \\
\M{subseteq}   & \M{gg}        & \M{simeq}      & \M{mid}    \\
\M{sqsubseteq} & \M{supset}    & \M{asymp}      & \M{cong}   \\
\M{vdash}      & \M{supseteq}  & \M{approx}     & \M{bowtie} \\
\M{smile}      & \M{frown}     & \M{parallel}   &     &      \\
  \hline
\end{tabular}
\\{\footnotesize これらのコマンドの前に \C{not}を付ければ
その関係子の否定になります}
\end{center}
\end{table}



%\begin{inout}
%\( \sum^n_{i=0} a_i \neq a_o+
% {\displaystyle\sum^{n-1}_{i=1}a_i}\)
%\end{inout}

\begin{table}[htbp]
\begin{center}\indindz{演算子}{2項}%
\caption{2項演算子}\tablab{app:ennzannsi}
\begin{tabular}{*{4}{c@{\thickspace\thinspace}l}}
 \hline
\M{pm}     & \M{cdot}  & \M{setminus}        & \M{ominus} \\
\M{mp}     & \M{cap}   & \M{wr}              & \M{otimes} \\
\M{times}  & \M{cup}   & \M{diamond}         & \M{oslash} \\
\M{div}    & \M{uplus} & \M{bigtriangleup}   & \M{odot}   \\
\M{ast}    & \M{sqcap} & \M{bigtriangledown} & \M{bigcirc}\\
\M{star}   & \M{sqcup} & \M{triangleleft}    & \M{dagger} \\
\M{circ}   & \M{vee}   & \M{triangleright}   & \M{ddagger}\\
\M{bullet} & \M{wedge} & \M{oplus}           & \M{amalg}  \\
 \hline
\end{tabular}
\end{center}
\end{table}

%\begin{inout}
%\begin{displaymath}
%  (p\to r)\vee  (q\to s)
%\end{displaymath}
%\end{inout}

\begin{table}[htbp]
 \begin{center}
  \caption{点}\tablab{tenn}
  \index{点}%
  \index{3点リーダ}%
  \index{3点リーダ!中点\zdash}%
  \index{3点リーダ!下付き\zdash}%
  \index{...@\ldots(下付3点リーダ)}%
  \index{...@$\cdots$(中点3点リーダ)}%
  \begin{tabular}{*{5}{c@{\thickspace\thinspace}l}}
   \hline
   \M{dots}  & 
   \M{ldots} &  \M{cdots} & \M{vdots} &  \M{ddots}\\
   \hline
  \end{tabular}
 \end{center}
\end{table}

\begin{inout}
\begin{eqnarray*}
 (a_0+a_1+\cdots+a_n) &\neq& \{a_0,a_1,\ldots,a_n\}\\
 \{f_n\} &=& f_1, f_2, \dots, f_n
\end{eqnarray*}
\end{inout}
\cmd{ldots} や \cmd{cdots} 以外に \C{dots} という命令
もあります.これは前後の数式要素に応じて自動的に \cmd{ldots} と \cmd{cdots} を
切り替える命令です.
%\begin{inout}
%\( \{f_n\} = f_1, f_2, \dots, f_n \)
%\end{inout} 
%しばしば適切に選定されない場合がありますので,その場合は手動で
%対処します.

\begin{table}[htbp]
\begin{center}
 \caption{矢印}\index{矢印}
 \begin{tabular}{*{3}{c@{\thickspace\thinspace}l}}
  \hline
 \M{leftarrow}       & \M{rightarrow}      \\
 \M{uparrow}         & \M{downarrow}        \\
 \M{Leftarrow}       & \M{Rightarrow}      \\
%  \hline
 \M{Uparrow}         & \M{Downarrow}        \\
 \M{updownarrow}     &  \M{Updownarrow}       \\
%  \hline
 \M{mapsto}          & \M{longmapsto}       \\
%  \hline
 \M{hookleftarrow}   & \M{hookrightarrow}   \\
 \M{leftrightarrow}  & \M{Leftrightarrow}     \\
 \M{longleftarrow}   & \M{longrightarrow}   \\
 \M{Longleftarrow}   & \M{Longrightarrow}   \\
 \M{Longleftrightarrow}\\
%  \hline
 \M{leftharpoonup}   & \M{rightharpoonup}   \\
 \M{leftharpoondown} & \M{rightharpoondown} \\
 \M{rightleftharpoons} \\
%  \hline
 \M{nearrow}         & \M{nwarrow}           \\
 \M{searrow}         & \M{swarrow}           \\
  \hline
 \end{tabular}
\end{center}
\end{table}


\begin{table}[htbp]
\begin{center}
\caption{特殊な数学記号}
\begin{tabular}{*{4}{c@{\thickspace\thinspace}l}}
 \hline
 \M{aleph} & \M{partial}  & \M{bot}       & \M{natural}     \\
 \M{hbar}  & \M{infty}    & \M{angle}     & \M{sharp}       \\
 \M{imath} & \M{prime}    & \M{triangle}  & \M{clubsuit}    \\
 \M{jmath} & \M{emptyset} & \M{forall}    & \M{diamondsuit} \\
 \M{ell}   & \M{nabla}    & \M{exists}    & \M{heartsuit}   \\
 \M{wp}    & \M{surd}     & \M{neg}       & \M{spadesuit}   \\
 \M{Re}    & $|$&\verb+|+& \M{backslash} &     &           \\
 \M{Im}    & \M{top}      & \M{flat}      &     &           \\
  \hline
\end{tabular}
\end{center}
\end{table}

%\begin{inout}
%\[ \forall{x}\forall{y}( 
%     P(x,y)\vee(f(x)\wedge g(x))) \] 
%\end{inout}
%
%\begin{inout}
%\( e^{j\theta}=\Re{\{e^{j\theta}\}}
%   +\Im{\{e^{j\theta}\}}
%   =\cos\theta+j\sin\theta\)
%\end{inout}

%\subsection{\Y{latexsym}}

%{\LaTeXe}からはこぼれた記号類を出力するためには,
%\Person{Frank}{Mittelbach}が作成した\Y{latexsym}
%を読み込むと良いでしょう.すでに\Y{amssymb}か
%\Y{amsfonts}を読み込んでいるならば,そちらに定
%義されているので\sty{latexsym}を読み込む必要はありません.

%\begin{table}[htbp]
% \begin{center}
%\caption{\Y{latexsym}パッケージに含まれる数学記号}
%  \begin{tabular}{*{4}{c@{\thickspace\thinspace}l}}
%   \hline
%   \M{mho}     & \M{Join}     & \M{Box}      & \M{Diamond} \\
%   \M{leadsto} & \M{sqsubset} & \M{sqsupset} & \M{lhd} \\
%   \M{unlhd}   & \M{rhd}      & \M{unrhd}    &    & \\
%   \hline
%  \end{tabular}
% \end{center}
%\end{table}



%    LaTeX \& latexsym 数学記号
\subsection{\sty{amsmath}}

\begin{table}[htbp]
 \begin{center}
  \zindind{ギリシャ文字}{の変体大文字}%
  \caption{\sty{amsmath}で追加されたギリシャ大文字の変体文字}\tablab{ams:upper:hen}
  \begin{tabular}{*{4}{c@{\thickspace\thinspace}l}}
   \hline
  \M{varGamma} & \M{varLambda} & \M{varSigma} & \M{varPsi}\\
  \M{varDelta} & \M{varXi} & \M{varUpsilon} & \M{varOmega}\\ 
  \M{varTheta} & \M{varPi} & \M{varPhi} & \\
   \hline
  \end{tabular}
 \end{center}
\end{table}

\begin{table}[htbp]
\begin{center}
 \caption{\sty{amsmath}で追加された数学関数}\tablab{ams:mathfunc}
 \begin{tabular}{*{3}{c@{\thickspace\thinspace}l}}
     \hline
\M{injlim}    & \M{projlim}   & \M{varliminf} \\[1ex]
\M{varlimsup} & \M{varinjlim} & \M{varprojlim}\\
     \hline
 \end{tabular}
\end{center}
\end{table}

\begin{table}[htbp]
\begin{center}
 \caption{\sty{amsmath}で追加された積分記号}
 \begin{tabular}{*{3}{c@{\thickspace\thinspace}l}}
     \hline
\M{oint} & \M{iint} & \M{iiint} \\[1ex]
\M{iiiint} & \M{idotsint} & \\
   \hline
 \end{tabular}
\end{center}
\end{table}


\tabref{ams:accents}で追加されたアクセントにおいて,
\cmd{dddot} と \cmd{ddddot}以外は基本的に二重のアクセントを
出力するために使われます.

\CIS{Hat,Acute,Bar,Dot,Check,Grave,Vec,Ddot,Breve,Tilde}%
\begin{table}[htbp]
\begin{center}
\newcommand*{\AW}[2]{%
  \glossary{#1@\hspace*{-1.2ex}\texttt{\protect\bs#1}%
    \hskip1em($\protect\csname#1\endcsname{#2}$)}%
  $\protect\csname#1\endcsname{#2}$ & %
  \texttt{\bs#1\lb\string#2\rb}}%
 \newcommand*\BW[1]{$#1{#1{A}}$ & \texttt{\string#1\lb\string#1\lb A\rb\rb}}
 \caption{\Y{amsmath}で追加されたアクセント記号}\tablab{ams:accents}
 \begin{tabular}{*{3}{c@{\thickspace\thinspace}l}}
     \hline
  \AW{dddot}{a}  & \BW{\Dot}   \\
  \AW{ddddot}{a} & \BW{\Ddot}  \\
  \BW{\Acute}    & \BW{\Grave} \\
  \BW{\Bar}      & \BW{\Hat}   \\
  \BW{\Breve}    & \BW{\Tilde} \\
  \BW{\Check}    & \BW{\Vec}   \\
   \hline
 \end{tabular}
\end{center}
\end{table}

\begin{table}[htbp]
\begin{center}
 \caption{\Y{amsmath}で追加された空白命令}\tablab{ams:aki}
\let \DW = \demowidth
 \begin{tabular}{llll}
   \hline
\C{thinspace} & \DW{1.66702truept} & \C{negthinspace} & \DW{-1.66702truept} \\
\C{medspace}  & \DW{2.22198truept} & \C{negmedspace} & \DW{-2.22198truept} \\
\C{thickspace} & \DW{2.77695truept} & \C{negthickspace} & \DW{-2.77695truept} \\
   \hline
 \end{tabular}
\end{center}
\end{table}



\begin{table}[htbp]
 \centering
 \caption{\Y{amsmath}で追加された大きいアクセント}\tablab{ams:big:accent}
 \begin{tabular}{*{2}{c@{\thickspace\thinspace}l}}
   \hline
  $\overrightarrow{a+b}$ & \C{overrightarrow}  & 
  $\underleftarrow{a+b}$ & \C{underleftarrow} \\
  $\underleftrightarrow{a+b}$ & \C{underleftrightarrow} & 
	  $\underrightarrow{a+b}$ & \C{underrightarrow} \\
   \hline
 \end{tabular}
\end{table}

\vskip.5\cvs plus 2pt minus 3pt% amsmath
% これはただのダミーテキスト.
% 文字コードを判定するための意味のない文字列.
% これくらい記述すれば大丈夫かな.
% Emacs のくせに生意気な.
% Emacs の分際で自動判別とか.
% Mac OS X のテキストエディッタの文字コード自動判別はうまくいかないぞ.

\subsection{\sty{amsxtra}で追加されたアクセント記号}
\begin{table}[htbp]
%\begin{center}
 \centering
 \caption{\Y{amsxtra}で追加された添字アクセント記号}%\tablab{ams:accents}
\newcommand*\SPC[1]{$A\csname#1\endcsname$ & \C{#1}}
{\small 上付き添字としてのアクセントですから`\verb|A\sphat|'のように使い
ます.}\\
 \begin{tabular}{*6l}
  \hline
 \SPC{sphat} & \SPC{spcheck} & \SPC{sptilde} \\
 \SPC{spdot} & \SPC{spddot} & \SPC{spdddot} \\
 \SPC{spbreve} \\
  \hline
 \end{tabular}
%\end{center}
\end{table}
% amsxtra
\subsection{\sty{amssymb} で拡張された記号}
%
\begin{table}[htbp]
 \centering
%\begin{center}
 \caption{\sty{amssymb}のギリシャ文字とヘブライ文字}
 \begin{tabular}{*{4}{c@{\thickspace\thinspace}l}}
  \hline
 \M{digamma}      & \M{beth}    & \M{gimel} \\
 \M{varkappa}     & \M{daleth}  & \\
    \hline
 \end{tabular}
%\end{center}
\end{table}
%
\begin{table}[htbp]
%\begin{center}
  \centering
 \caption{\sty{amssymb}の2項演算子}
 \begin{tabular}{*{4}{c@{\thickspace\thinspace}l}}
  \hline
 \M{dotplus}       & \M{boxtimes}       & \M{curlywedge}\\
 \M{smallsetminus} & \M{boxdot}         & \M{curlyvee}\\
 \M{Cap}           & \M{boxplus}        & \M{circleddash}\\
 \M{Cup}           & \M{divideontimes}  & \M{circledast}\\
 \M{barwedge}      & \M{ltimes}         & \M{circledcirc}\\
 \M{veebar}        & \M{rtimes}         & \M{centerdot}\\
 \M{doublebarwedge}& \M{leftthreetimes} & \M{intercal}\\
 \M{boxminus}      & \M{rightthreetimes}&      &      \\
  \hline
 \end{tabular}
%\end{center}
\end{table}
%
\begin{table}[htbp]
 \centering
\caption{\sty{amssymb} の区切り記号}
\begin{tabular}{*{4}{c@{\thickspace\thinspace}l}}
 \hline
\M{ulcorner} & \M{urcorner} & \M{llcorner} & \M{lrcorner} \\
 \hline 
\end{tabular}
\end{table}
%
\begin{table}[htbp]
  \centering
 \caption{\sty{amssymb} の矢印記号}\index{矢印}
 \begin{tabular}{*{2}{c@{\thickspace\thinspace}l}}
  \hline
  \M{Lsh}              & \M{Rsh} \\
 \M{circlearrowleft}   &\M{circlearrowright}\\
 \M{curvearrowleft}    &\M{curvearrowright}\\
 \M{dashleftarrow}     &\M{dashrightarrow}   \\
 \M{leftarrowtail}    & \M{rightarrowtail}\\
 \M{leftleftarrows}   &  \M{rightrightarrows} \\
 \M{leftrightarrows}  & \M{rightleftarrows}  \\
 \M{leftrightharpoons}& \M{rightleftharpoons}\\
 \M{looparrowleft}    & \M{looparrowright}\\
 \M{leftrightsquigarrow} & \M{rightsquigarrow}\\
  \M{Lleftarrow}  & \M{Rrightarrow}\\  
 \M{twoheadleftarrow} & \M{twoheadrightarrow}\\
 \M{upharpoonleft}    & \M{upharpoonright}\\
 \M{downdownarrows}    &\M{upuparrows}       \\
 \M{downharpoonleft}   &\M{downharpoonright}\\
   \M{multimap}         & \\
  \hline
 \end{tabular}
\end{table}
%
\begin{table}[htbp]
  \centering
 \caption{\sty{amssymb} の否定矢印記号}
 \begin{tabular}{*{3}{c@{\thickspace\thinspace}l}}
   \hline
 \M{nleftarrow}  & \M{nleftrightarrow} & \M{nLeftarrow}\\
 \M{nrightarrow} & \M{nLeftrightarrow} &     &         \\
  \hline
 \end{tabular}
\end{table}
%
\begin{longtable}{*{2}{c@{\thickspace\thinspace}l}}
  \caption{\sty{amssymb} の2項関係子}\\
  \hline
  \endfirsthead
  \hline
  \multicolumn{4}{c}{前ページからの続きです}\\
  \endhead
 \multicolumn{4}{c}{次ページへ続きます}\\
  \hline
  \endfoot
  \hline
  \endlastfoot
 %
  \M{because} & \M{therefore}\\
  \M{between}\\
  \M{preccurlyeq} & \M{succcurlyeq} \\
  \M{trianglelefteq} & \M{trianglerighteq}\\
  \M{triangleq} \\
  \M{vartriangleleft} & \M{vartriangleright}\\
  \M{lll} & \M{ggg}\\
  \M{blacktriangleleft} & \M{blacktriangleright}\\
  \M{bumpeq} & \M{Bumpeq}\\
  \M{Vdash} & \M{Vvdash}\\
  \M{vDash}\\
  \M{varpropto} & \M{pitchfork}\\
  \M{sqsubset} & \M{sqsupset}\\
  \M{subseteqq}& \M{supseteqq}\\
  \M{Subset}   & \M{Supset}\\
  \M{thicksim} & \M{thickapprox}\\
  \M{lessapprox} &\M{gtrapprox} \\
  \M{precapprox} & \M{succapprox}\\
  \M{approxeq}\\
  \M{backsim} & \M{backsimeq}\\
  \M{lesssim} & \M{gtrsim}\\
  \M{precsim} & \M{succsim}\\
  \M{circeq}    & \M{eqcirc}\\
  \M{curlyeqprec} & \M{curlyeqsucc}\\
  \M{eqslantless} & \M{eqslantgtr} \\
  \M{fallingdotseq} & \M{risingdotseq}\\
   \M{lessdot} &\M{gtrdot}\\
  \M{leqq} & \M{geqq}\\
  \M{lesseqgtr} & \M{gtreqless}\\
  \M{lesseqqgtr} & \M{gtreqqless}\\
  \M{lessgtr} & \M{gtrless}\\
  \M{leqslant} & \M{geqslant}\\
  \M{shortparallel} & \M{shortmid}\\
  \M{smallfrown} & \M{smallsmile}\\
  \M{backepsilon} & \M{doteqdot}\\
\end{longtable}
%
\begin{table}[htbp]
 \centering
 \caption{\sty{amssymb} の否定2項関係子}
 \begin{tabular}{*{3}{c@{\thickspace\thinspace}l}}
  \hline
 \M{nless}&     \M{ntriangleleft}&\M{nsucceq}\\
 \M{nleq}&\M{ntrianglelefteq}    &\M{succnsim}\\
 \M{nleqslant}&     \M{nsubseteq}&\M{succnapprox}\\
 \M{nleqq}&         \M{subsetneq}&\M{ncong}\\
 \M{lneq}&       \M{varsubsetneq}&\M{nshortparallel}\\
 \M{lneqq}&        \M{subsetneqq}&\M{nparallel}\\
 \M{lvertneqq}&\M{varsubsetneqq} &\M{nvDash}\\
 \M{lnsim}&\M{ngtr}              &\M{nVDash}\\
 \M{lnapprox}&\M{ngeq}           &\M{ntriangleright}\\
 \M{nprec}&\M{ngeqslant}         &\M{ntrianglerighteq}\\
 \M{npreceq}& \M{ngeqq}          &\M{nsupseteq}\\
 \M{precnsim}&\M{gneq}           &\M{nsupseteqq}\\
 \M{precnapprox}&\M{gneqq}       &\M{supsetneq}\\
 \M{nsim}&\M{gvertneqq}          &\M{varsupsetneq}\\
 \M{nshortmid}& \M{gnsim}        &\M{supsetneqq}\\
 \M{nmid}&\M{gnapprox}           &\M{varsupsetneqq}\\
 \M{nvdash}&\M{nsucc}            & & \\
 \M{nvDash}&\M{nsucceq}          & & \\
    \hline
 \end{tabular}
\end{table}
%
\begin{table}[htbp]
 \centering
 \caption{その他の\sty{amssymb}記号\tablab{app:ams-misc}}
 \begin{tabular}{*{3}{c@{\thickspace}l}}
  \hline
\M{angle}   & \M{vartriangle}       & \M{lozenge}\\
\M{Bbbk}    & \M{triangledown}      & \M{blacklozenge}\\
\M{eth}     & \M{blacktriangle}     & \M{bigstar}\\
\M{Finv}    & \M{blacktriangledown} & \M{diagup}\\
\M{Game}    & \M{sphericalangle}    & \M{diagdown}\\
\M{hbar}    & \M{measuredangle}     & \M{circledS}\\
\M{hslash}  & \M{square}            & \M{backprime}\\
\M{mho}     & \M{blacksquare}       & \M{varnothing} \\
\M{nexists} & \M{complement}\\
  \hline
 \end{tabular}
\end{table}
%
\begin{table}[htbp]
 \centering
  \caption{その他の\sty{amssymb}文字記号\tablab{app:ams-text-symb}}
  \begin{tabular}{*{4}{c@{\thickspace\thinspace}l}}
   \hline
    \T{checkmark} & \T{circledR} & \T{maltese} & \T{yen} \\
   \hline
  \end{tabular}
\end{table}













% amssymb
\subsection{\Y{txfonts}/\Y{pxfonts}での拡張}
\newcommand*\torpxfonts{\sty{txfonts}/\sty{pxfonts}\xspace}
\begin{usage}
\usepackage{amsmath,amssymb}% 先に読み込みます
\usepackage[T1]{fontenc}
\usepackage{pxfonts}
\end{usage}

\begin{table}[htbp]
\centering
 \caption{\torpxfonts で拡張された2項演算子}
 \tablab{app:txfonts:BinOpe}
 \begin{tabular}{*{3}{c@{\thickspace\thinspace}l}}
   \hline
 \M{medcirc}       &  \M{nplus}     & \M{sqcapplus}\\
 \M{medbullet}     &  \M{boxast}    & \M{rhd}\\
 \M{invamp}        &  \M{boxbslash} & \M{lhd}\\
 \M{circledwedge}  &  \M{boxbar}    & \M{unrhd}\\
 \M{circledvee}    &  \M{boxslash}  & \M{unlhd}\\
 \M{circledbar}    &  \M{Wr}        &  \\
 \M{circledbslash} &  \M{sqcupplus} &  \\
   \hline
 \end{tabular}
\end{table}
%
\begin{table}[htbp]
 \centering
  \indindz{記号}{数学}%
 \caption{\torpxfonts で拡張された数学記号}
 \tablab{app:txfonts:OrdSym}
 \begin{tabular}{*{3}{c@{\thickspace\thinspace}l}}
   \hline
 \M{alphaup}   &\M{nuup}      &\M{omegaup}\\
 \M{betaup}    &\M{xiup}      &\M{Diamond}\\
 \M{gammaup}   &\M{piup}      &\M{Diamonddot}\\
 \M{deltaup}   &\M{varpiup}   &\M{Diamondblack}\\
 \M{epsilonup} &\M{rhoup}     &\M{lambdaslash}\\
 \M{varepsilonup}&\M{varrhoup}&\M{lambdabar}\\
 \M{zetaup}    &\M{sigmaup}   &\M{varclubsuit}\\
 \M{etaup}     &\M{varsigmaup}&\M{vardiamondsuit}\\
 \M{thetaup}   &\M{tauup}     &\M{varheartsuit}\\
 \M{varthetaup}&\M{upsilonup} &\M{varspadesuit}\\
 \M{iotaup}    &\M{phiup}     &\M{Top}\\
 \M{kappaup}   &\M{varphiup}  &\M{Bot}\\
 \M{lambdaup}  &\M{chiup}     &\\
 \M{muup}      &\M{psiup}     &\\
   \hline  
 \end{tabular}
\end{table}
%
\begin{table}[htbp]
 \centering
 \caption{\torpxfonts で拡張された大型演算子}
 \tablab{app:txfonts:LargeOpe}
  \begin{tabularx}{\linewidth}{l@{\hskip1em}X@{\thinspace}l@{\hskip1em}X}
   \hline
  \M{bignplus} & \M{varprod}\\
  \M{bigsqcupplus} & \M{bigsqcapplus}\\
  \M{bigsqcup} & \M{bigsqcap}\\
  \M{iint} & \M{iiint}\\[2pt]
  \M{iiiint} & \M{idotsint}\\[2pt]
  \M{oiint} & \M{oiiint}\\[2pt]
  \M{ointclockwise} & \M{ointctrclockwise}\\[3pt]
  \M{oiintclockwise} & \M{oiintctrclockwise}\\[10pt]
  \M{oiiintclockwise} & \M{oiiintctrclockwise}\\[10pt]
  \M{varointclockwise}   & \M{varointctrclockwise}\\[5pt]
  \M{varoiintclockwise}  & \M{varoiintctrclockwise}\\[9pt]
  \M{varoiiintclockwise} & \M{varoiiintctrclockwise}\\[10pt]
  \M{fint}  & \M{sqint}\\[8pt]
  \M{sqiintop} & \M{sqiiintop}\\[10pt]
     \hline
 \end{tabularx}
\end{table}
%
\begin{longtable}{*{2}{c@{\thickspace\thinspace}l}}
\caption{\torpxfonts で拡張された2項関係子}\\%
%\tablab{app:txfonts:Bin:Rel}}
  \hline
  \endfirsthead
  \hline
  \multicolumn{4}{c}{前ページからの続きです}\\
  \endhead
  \multicolumn{4}{c}{次ページへ続きます}\\
  \hline
  \endfoot
  \hline
  \endlastfoot
 %
 \M{mappedfrom} & \M{longmappedfrom}\\
 \M{Mappedfrom} & \M{Mapsto}\\
 \M{Longmappedfrom} &\M{Longmapsto}\\
 \M{mmappedfrom} & \M{mmapsto}\\
 \M{longmmappedfrom} & \M{longmmapsto}\\
 \M{Mmappedfrom} & \M{Mmapsto}\\
 \M{Longmmappedfrom} & \M{Longmmapsto}\\
 \M{varparallel}&\M{varparallelinv}\\
 \M{nvarparallel}&\M{nvarparallelinv}\\
 \M{colonapprox}&\M{colonsim}\\
 \M{Colonapprox}&\M{Colonsim}\\
 \M{doteq}&\\
 \M{multimapinv}& \M{multimapboth}\\
 \M{multimapdot}&\M{multimapdotinv}\\
 \M{multimapdotbothA} & \M{multimapdotbothB}\\
 \M{multimapdotboth}&\M{VDash}\\
 \M{VvDash}&\M{cong}\\
 \M{preceqq}&\M{succeqq}\\
 \M{nprecsim} & \M{nsuccsim}\\
 \M{nlesssim} & \M{ngtrsim}\\
 \M{nlessapprox} & \M{ngtrapprox}\\
 \M{npreccurlyeq} & \M{nsucccurlyeq}\\
 %
 \M{ngtrless} & \M{nlessgtr}\\
 \M{nbumpeq} & \M{nBumpeq}\\
 \M{nbacksim} & \M{nbacksimeq}\\
 \M{ne} & \M{nasymp}\\
 \M{nequiv} & \M{nsim}\\
 \M{napprox} & \\
 \M{nsubset} & \M{nsupset}\\
 \M{nll} & \M{ngg}\\
 %
 \M{nthickapprox} & \M{napproxeq}\\
 \M{nprecapprox} & \M{nsuccapprox}\\
 \M{npreceqq} & \M{nsucceqq}\\
 \M{nsimeq} \\
 \M{notin} & \M{notni} \\
 \M{nSubset} & \M{nSupset}\\
 \M{nsqsubseteq} & \M{nsqsupseteq}\\
 \M{coloneqq} & \M{eqqcolon}\\
 \M{coloneq} & \M{eqcolon}\\
 \M{Coloneqq} & \M{Eqqcolon}\\
 \M{Coloneq} & \M{Eqcolon}\\
 \M{strictif} & \M{strictfi}\\
 \M{strictiff} & \\
 \M{circledless} & \M{circledgtr}\\
 \M{lJoin} &\M{rJoin} \\
 %
 \M{Join} & \M{openJoin}\\
 \M{lrtimes} & \M{opentimes}\\
 \M{nsqsubset} & \M{nsqsupset}\\
 \M{dashleftarrow} & \M{dashrightarrow}\\
 \M{dashleftrightarrow} & \M{leftsquigarrow}\\
 \M{ntwoheadleftarrow} & \M{ntwoheadrightarrow}\\
 \M{Nearrow} & \M{Searrow}\\
 \M{Nwarrow} & \M{Swarrow}\\
 \M{Perp} \\
 \M{leadstoext} & \M{leadsto}\\
 %
 \M{boxleft} & \M{boxright}\\
 \M{boxdotleft} & \M{boxdotright}\\
 \M{Diamondleft} & \M{Diamondright}\\
 \M{Diamonddotleft} & \M{Diamonddotright}\\
 \M{boxLeft} & \M{boxRight}\\
 \M{boxdotLeft} & \M{boxdotRight}\\
 \M{DiamondLeft} & \M{DiamondRight}\\
 \M{DiamonddotLeft} & \M{DiamonddotRight}\\
 \M{circleleft} & \M{circleright}\\
 \M{circleddotleft} & \M{circleddotright}\\
 %
 \M{multimapbothvert} & \M{multimapdotbothvert}\\
 \M{multimapdotbothAvert} & \M{multimapdotbothBvert}\\
\end{longtable}
%
\begin{table}[htbp]
 \centering
  \indindz{記号}{区切り}%
 \caption{\torpxfonts で拡張された区切り記号}
 \tablab{app:txfonts:delimi}
 \begin{tabular}{*{4}{c@{\thickspace\thinspace}l}}
   \hline
 \M{llbracket} & \M{rrbracket} & \M{lbag} & \M{rbag}\\
     \hline
 \end{tabular}
\end{table}
%
%    TXFonts/PXFonts
%#!platex -kanji=utf8 hb.tex
%\chapter{記号}

\section{OTFパッケージ}
\begin{usage}
\usepackage[$\<オプション>$]{otf} 
\end{usage}
\Y{OTF}パッケージは\LaTeX で\W{Open Typeフォント}を扱うためのマクロパッ
ケージです.\W{udvips},\W{dvipdfmx},\W{Mxdvi},\W{xdvi}の\W{デバイスド
ライバ} (\W{dviware}) に対応しています.\W{dviout}は制限付きで対応してい
ます.単に\W{ユニコード}文字を使うだけであれば\Y{UTF}パッケージが使えま
す.
%ただし,\sty{UTF}パッケージは開発が終了したものです.
%\sty{OTF}パッケージのオプションとしては以下が用意されています.
\begin{description}
 \item[noreplace] 
	    クラスファイルで元々定義されている\W{TFM}を用います.
	    何も指定しなければTFMが置き換えられます.
 \item[bold]      
	    ゴシック体を太字として割り当てます.
 \item[expert]    
	    組方向に応じた専用仮名を使います.
	    仮名が縦組専用,または横組専用のものに切り替わり,ルビ用の仮
	    名を使えるようになります.
	    \C{rubyfamily} コマンドで使用できます.
 \item[deluxe]    多ウェイト化.
	    明朝体,ゴシック体を2ウェイト化します。(該当フォントが存在
	    する場合)丸ゴシック体も使えるようになります.
 \item[multi]     
	    \W{繁体字}、\W{簡体字}、\W{ハングル}を使えるようにします.
 \item[nomacros]  
	    \Y{ajmacros}を読み込まないようにします.
\end{description}

%\begin{itemize}
% \item \pTeX は\W{JIS~X~0208}までの\W{文字集合}しか扱えません.
% \item OTFは\W{JIS~X~0213}に完全対応しています.
% \item Unicode~3.0の\W{字体}をUTF16コードで指定可能です.
% \item 日本語に限りAdobe-Japan~1-5の\W{CID番号}を指定して出力可能です.
% \item \sty{UTF}パッケージの拡張バージョンであり,\W{Open Typeフォント}
%       が扱えるようになって名称が変更されました.
% \item \Y{OTF}と表記されていますが,実際には \verb|\usepackage{otf}| と
%       原稿に書きます.
% \item \W{CID番号}はAdobe社の開発者向けサイトで公開されている技術資料か
%       ら知る事が出来ます\footnote{\webAdobeJapanList}.
%\end{itemize}

\begin{usage}
\UTF{$\<4桁の16進数>$} % UTF16コードで指定
\CID{$\<10進数>$}% CID 番号で指定
\end{usage}
\W{CID番号}はAdobe社の開発者向けサイトで公開されている技術資料か
ら知る事が出来ます\footnote{\webAdobeJapanList}.

%\begin{itemize}
% \item \C{UTF}\val{4桁の16進数}
% \item \C{CID}\val{10進数}
%\end{itemize}

%\CID{16315}
%\CID{16321}
%\CID{16319}

%\makeatletter
%\@tempcnta=\z@
%\@whilenum\@tempcnta<20317\do{%
%  \CID{\the\@tempcnta}
%  \advance\@tempcnta\@ne
%}
%\makeatother

%\CID{2234}\CID{2233}
%\CID{2244}


\begin{inout}
\UTF{2318}+\keytop{Tab}キーでアプリケーションを選択してから,
\UTF{23ce}キーを押して下さい.
\end{inout}


%HOGE パッケージオプション

\subsection{囲みつき文字 (\protect\ajKuroKaku{5}, \protect\ajMaru{99}) を出力する}

\begin{longtable}{*4l}
  % 最初のヘッダ
  \caption{\sty{OTF}の囲みつき文字}\tablab{OTFの囲みつき文字}\\
  \toprule 
  命令 & 最小値 & 最大値 & 出力例\\
  \midrule
  \endfirsthead
  % 続きのヘッダ
  \toprule 
  命令 & 最小値 & 最大値 & 出力例\\
  \midrule 
  \endhead
  % 続きのフッタ
  \multicolumn{4}{c}{次ページに続きます}\\
  \bottomrule  
  \endfoot
  % 最後のフッタ
  \bottomrule  
  \endlastfoot

 \otfKakomi{ajMaru}{0}{100}
 \otfKakomi[*]{ajMaru}{0}{100}
 \otfKakomi{ajKuroMaru}{0}{100}
 \otfKakomi[*]{ajKuroMaru}{0}{100}
 \otfKakomi{ajKaku}{0}{100}
 \otfKakomi[*]{ajKaku}{0}{100}
 \otfKakomi{ajKuroKaku}{0}{100}
 \otfKakomi[*]{ajKuroKaku}{0}{100}
 \otfKakomi{ajMaruKaku}{0}{100}
 \otfKakomi[*]{ajMaruKaku}{0}{100}
 \otfKakomi{ajKuroMaruKaku}{0}{100}
 \otfKakomi[*]{ajKuroMaruKaku}{0}{100}
 \otfKakomi{ajKakko}{0}{100}
 \otfKakomi[*]{ajKakko}{0}{100}
 \otfKakomi{ajRoman}{1}{15}
 \otfKakomi[*]{ajRoman}{1}{15}
 \otfKakomi{ajroman}{1}{15}
 \otfKakomi{ajPeriod}{1}{9}% AJ1-6?
 \otfKakomi{ajKakkoYobi}{1}{9}
 \otfKakomi{ajKakkoroman}{1}{15}
 \otfKakomi{ajKakkoRoman}{1}{15}
 \otfKakomi{ajKakkoalph}{1}{26}
 \otfKakomi{ajKakkoAlph}{1}{26}
 \otfKakomi{ajKakkoHira}{1}{48}
 \otfKakomi{ajKakkoKata}{1}{48}
 \otfKakomi{ajKakkoKansuji}{1}{20}
 \otfKakomi{ajMaruKansuji}{1}{10}
 \otfKakomi{ajMarualph}{1}{26}
 \otfKakomi{ajMaruAlph}{1}{26}
 \otfKakomi{ajMaruHira}{1}{48}
 \otfKakomi{ajMaruKata}{1}{48}
 \otfKakomi{ajMaruYobi}{1}{7}
 \otfKakomi{ajKuroMarualph}{1}{26}
 \otfKakomi{ajKuroMaruAlph}{1}{26}
 \otfKakomi{ajKuroMaruHira}{1}{48}
 \otfKakomi{ajKuroMaruKata}{1}{48}
 \otfKakomi{ajKuroMaruYobi}{1}{7}
 \otfKakomi{ajKakualph}{1}{26}
 \otfKakomi{ajKakuAlph}{1}{26}
 \otfKakomi{ajKakuHira}{1}{26}
 \otfKakomi{ajKakuKata}{1}{48}
 \otfKakomi{ajKakuYobi}{1}{7}
 \otfKakomi{ajKuroKakualph}{1}{26}
 \otfKakomi{ajKuroKakuAlph}{1}{26}
 \otfKakomi{ajKuroKakuHira}{1}{48}
 \otfKakomi{ajKuroKakuKata}{1}{48}
 \otfKakomi{ajKuroKakuYobi}{1}{7}
 \otfKakomi{ajMaruKakualph}{1}{26}
 \otfKakomi{ajMaruKakuAlph}{1}{26}
 \otfKakomi{ajMaruKakuHira}{1}{48}
 \otfKakomi{ajMaruKakuKata}{1}{48}
 \otfKakomi{ajMaruKakuYobi}{1}{7}
 \otfKakomi{ajKuroMaruKakualph}{1}{26}
 \otfKakomi{ajKuroMaruKakuAlph}{1}{26}
 \otfKakomi{ajKuroMaruKakuHira}{1}{48}
 \otfKakomi{ajKuroMaruKakuKata}{1}{48}
 \otfKakomi{ajKuroMaruKakuYobi}{1}{7}
 \otfKakomi{ajNijuMaru}{1}{10}
  \otfKakomi{ajRecycle}{0}{11}
\end{longtable}

\begin{inout}
\usepackage{otf}
リチウムイオン電池には識別マーク\ajRecycle{0}が表示されていま
すので,使用済み電池はお近くのリサイクル協力店にお持ちください.
\end{inout}

\subsection{合字 (\protect\ajLig{株式会社}, \protect\ajLig{アパート}) を出力する}
\begin{usage}
\ajLig{$\<引数>$}
\end{usage}

\begin{small}
\begin{longtable}{*6l}
  % 最初のヘッダ
  \caption{\sty{OTF}の合字}\tablab{OTFの合字}\\
  \toprule 
  引数 & 横組 & 縦組  & 引数 & 横組 & 縦組\\
  \midrule
  \endfirsthead
  % 続きのヘッダ
  \toprule 
  引数 & 横組 & 縦組  & 引数 & 横組 & 縦組\\
  \midrule 
  \endhead
  % 続きのフッタ
  \multicolumn{6}{c}{次ページに続きます}\\
  \bottomrule  
  \endfoot
  % 最後のフッタ
  \bottomrule  
  \endlastfoot

 \otfLig{明治} \otfLig{大正} \otfLig{昭和} 
 \otfLig{ミリ} \otfLig{キロ} \otfLig{センチ} \otfLig{センチ*} \otfLig{メートル}
 \otfLig{グラム} \otfLig{グラム*} \otfLig{トン} \otfLig{アール}
 \otfLig{アール*} \otfLig{ヘクタール} \otfLig{リットル} \otfLig{ワット}
 \otfLig{ワット*} \otfLig{カロリー} \otfLig{ドル} \otfLig{セント}
 \otfLig{セント*} \otfLig{パーセント} \otfLig{ミリバール} \otfLig{ページ}
 \otfLig{ページ*} \otfLig{キロメートル} \otfLig{キログラム} \otfLig{アパート}
 \otfLig{ビル} \otfLig{マンション} \otfLig{株式会社} \otfLig{有限会社}
 \otfLig{財団法人} \otfLig{平成} \otfLig{フィート} \otfLig{インチ}
 \otfLig{インチ*} \otfLig{ヤード} \otfLig{ヤード*} \otfLig{ヘルツ}
 \otfLig{ヘルツ*} \otfLig{ホーン} \otfLig{ホーン*} \otfLig{コーポ}
 \otfLig{コーポ*} \otfLig{ハイツ} \otfLig{ハイツ*} \otfLig{さじ}
 \otfLig{アト} \otfLig{アルファ} \otfLig{アンペア} \otfLig{イニング}
 \otfLig{ウォン} \otfLig{ウルシ} \otfLig{エーカー} \otfLig{エクサ}
 \otfLig{エスクード} \otfLig{オーム} \otfLig{オングストローム} \otfLig{オンス}
 \otfLig{オントロ} \otfLig{カイリ} \otfLig{カップ} \otfLig{カラット}
 \otfLig{ガロン} \otfLig{ガンマ} \otfLig{ギガ} \otfLig{ギニー}
 \otfLig{キュリー} \otfLig{ギルダー} \otfLig{キロリットル} \otfLig{キロワット}
 \otfLig{グスーム} \otfLig{グラムトン} \otfLig{クルサード} \otfLig{クルゼイロ}
 \otfLig{クローネ} \otfLig{ケース} \otfLig{コルナ} \otfLig{サイクル}
 \otfLig{サンチーム} \otfLig{シリング} \otfLig{ダース} \otfLig{デカ}
 \otfLig{デシ} \otfLig{テラ} \otfLig{ドラクマ} \otfLig{ナノ}
 \otfLig{ノット} \otfLig{バーツ} \otfLig{バーレル} \otfLig{パスカル}
 \otfLig{バレル} \otfLig{ピアストル} \otfLig{ピクル} \otfLig{ピコ}
 \otfLig{ファラッド} \otfLig{ファラド} \otfLig{フェムト} \otfLig{ブッシェル}
 \otfLig{フラン} \otfLig{ベータ} \otfLig{ヘクト} \otfLig{ヘクトパスカル}
 \otfLig{ペセタ} \otfLig{ペソ} \otfLig{ペタ} \otfLig{ペニヒ}
 \otfLig{ペンス} \otfLig{ポイント} \otfLig{ホール} \otfLig{ボルト}
 \otfLig{ホン} \otfLig{ポンド} \otfLig{マイクロ} \otfLig{マイル}
 \otfLig{マッハ} \otfLig{マルク} \otfLig{ミクロン} \otfLig{メガ}
 \otfLig{メガトン} \otfLig{ヤール} \otfLig{ユアン} \otfLig{ユーロ} \otfLig{ラド} \otfLig{リラ} 
 \otfLig{ルーブル} \otfLig{ルクス} \otfLig{ルピア} \otfLig{ルピー} \otfLig{レム} \otfLig{レントゲン} 
 \otfLig{医療法人} \otfLig{学校法人} \otfLig{共同組合} \otfLig{協同組合} \otfLig{合資会社}
 \otfLig{合名会社} \otfLig{社団法人} \otfLig{宗教法人} \otfLig{郵便番号}\\
 \end{longtable}
\end{small}

\begin{table}[htbp]
 \centering
  \caption{\sty{OTF}のくの字などの記号}\tablab{OTFのくの字などの記号類}
 \setcounter{otfligline}{0}
 \setcounter{otfligrow}{2}
 \begin{tabular}{*4l}
  \hline
  引数&合字&引数&合字\\%&引数&合字\\%&引数&合字\\
  \hline
  \otfAjt{Kunoji}
  \otfAjt{KunojiwithBou}
  \otfAjt{DKunoji}
  \otfAjt{DKunojiwithBou}
  \otfAjt{Ninoji}
  \otfAjt{varNinoji}
  \otfAjt{Yusuriten}
  \\
  \hline
 \end{tabular}
\end{table}

\begin{inout}
\usepackage{otf}
宛名書きにおいて「株式会社」を \ajLig{(株)} と表記したり,
(\ajLig{株式会社}) と表記するのは先方に対して失礼になる.
\end{inout}

 \setcounter{otfligline}{0}
 \setcounter{otfligrow}{3}
\begin{longtable}{*6l}
  \caption{\sty{OTF}の仮名文字の合字}\tablab{OTFの仮名文字の合字}\\
  \hline
  引数 & 合字 & 引数 & 合字 & 引数 & 合字\\
  \hline
  \otfLigS{!!} \otfLigS{??} \otfLigS{!?*} 
  \otfLigS{!?} \otfLigS{!*} \otfLigS{?!} \otfLigS{!!*} \\
  \hline\setcounter{otfligline}{0}%
  \otfLigS{う゛} \otfLigS{ワ゛} \otfLigS{ヰ゛} \otfLigS{ヱ゛} \otfLigS{ヲ゛}
  \otfLigS{か゜} \otfLigS{き゜} \otfLigS{く゜} \otfLigS{け゜} \otfLigS{こ゜}
  \otfLigS{カ゜} \otfLigS{キ゜} \otfLigS{ク゜} \otfLigS{ケ゜} \otfLigS{コ゜}
  \otfLigS{セ゜} \otfLigS{ツ゜} \otfLigS{ト゜}
  \hline\setcounter{otfligline}{0}%
  %
  \otfLigS{小か} \otfLigS{小け} \otfLigS{小こ} \otfLigS{小コ}
  \otfLigS{小ク} \otfLigS{小シ} \otfLigS{小ス} \otfLigS{小ト}
  \otfLigS{小ヌ} \otfLigS{小ハ} \otfLigS{小ヒ} \otfLigS{小フ}
  \otfLigS{小ヘ} \otfLigS{小ホ} \otfLigS{小プ} \otfLigS{小ム}
  \otfLigS{小ラ} \otfLigS{小リ} \otfLigS{小ル} \otfLigS{小レ} 
  \otfLigS{小ロ}
  %
  \hline
\end{longtable}

\begin{table}[htbp]
\centering
 \setcounter{otfligline}{0}
 \setcounter{otfligrow}{4}
 \caption{\sty{OTF}の丸文字・括弧文字の合字}\tablab{OTFの丸文字・括弧文字の合字}
 \begin{tabular}{*8l}
  \hline
  引数&合字&引数&合字&引数&合字&引数&合字\\
  \hline
  \otfLigS{○上} \otfLigS{○中} \otfLigS{○下} \otfLigS{○左} 
  \otfLigS{○右} \otfLigS{○〒} \otfLigS{○夜} \otfLigS{○企} 
  \otfLigS{○医} \otfLigS{○協} \otfLigS{○名} \otfLigS{○宗}
  \otfLigS{○労} \otfLigS{○学} \otfLigS{○有} \otfLigS{○株}
  \otfLigS{○社} \otfLigS{○監} \otfLigS{○資} \otfLigS{○財} 
  \otfLigS{○印} \otfLigS{○秘} \otfLigS{○大} \otfLigS{○小}
  \otfLigS{○優} \otfLigS{○控} \otfLigS{○調} \otfLigS{○注}
  \otfLigS{○副} \otfLigS{○減} \otfLigS{○標} \otfLigS{○欠} 
  \otfLigS{○基} \otfLigS{○禁} \otfLigS{○項} \otfLigS{○休} 
  \otfLigS{○女} \otfLigS{○男} \otfLigS{○正} \otfLigS{○写} 
  \otfLigS{○祝} \otfLigS{○出} \otfLigS{○適} \otfLigS{○特} 
  \otfLigS{○済} \otfLigS{○増} \otfLigS{○問} \otfLigS{○答} 
  \otfLigS{○例} \otfLigS{○電}
  \\\setcounter{otfligline}{0}%
  \otfLigS{(株)} \otfLigS{(有)} \otfLigS{(代)} \otfLigS{(至)} 
  \otfLigS{(企)} \otfLigS{(協)} \otfLigS{(名)} \otfLigS{(労)} 
  \otfLigS{(社)} \otfLigS{(監)} \otfLigS{(自)} \otfLigS{(資)} 
  \otfLigS{(財)} \otfLigS{(特)} \otfLigS{(学)} \otfLigS{(祭)} 
  \otfLigS{(呼)} \otfLigS{(祝)} \otfLigS{(休)} \otfLigS{(営)} 
  \otfLigS{(合)} \otfLigS{(注)} \otfLigS{(問)} \otfLigS{(答)} 
  \otfLigS{(例)}
  \\\setcounter{otfligline}{0}%
   \otfLigS{●問} \otfLigS{●答}  \otfLigS{●例} \otfLigS{□印}
   \otfLigS{□問} \otfLigS{□答}  \otfLigS{□例} \otfLigS{□負}
   \otfLigS{■問} \otfLigS{■答}  \otfLigS{■例} \otfLigS{□勝} 
   \otfLigS{◇問} \otfLigS{◇答}  \otfLigS{◇例} 
  \\\setcounter{otfligline}{0}%
   \otfLigS{◆問} \otfLigS{◆答}  \otfLigS{◆例}
  \\
  \hline  
 \end{tabular}
\end{table}

\begin{table}[htbp]
\centering
  \caption{\sty{OTF}の単位の合字}\tablab{OTFの単位の合字}
 \setcounter{otfligline}{0}
 \setcounter{otfligrow}{4}
 \begin{tabular}{*8l}
  \hline
  引数&合字&引数&合字&引数&合字&引数&合字\\
  \hline
  \otfLigS{mm} \otfLigS{cm} \otfLigS{km} \otfLigS{mg}
  \otfLigS{kg} \otfLigS{cc} \otfLigS{m2} \otfLigS{No.}
  \otfLigS{K.K.} \otfLigS{TEL} \otfLigS{cm2} \otfLigS{km2}
  \otfLigS{cm3} \otfLigS{m3} \otfLigS{dl} \otfLigS{l}
  \otfLigS{kl} \otfLigS{ms} \otfLigS{micros} \otfLigS{ns}
  \otfLigS{ps} \otfLigS{KB} \otfLigS{MB} \otfLigS{GB}
  \otfLigS{HP} \otfLigS{Hz} \otfLigS{mb} \otfLigS{ml}
  \otfLigS{KK.} \otfLigS{Tel} \otfLigS{in} \otfLigS{mm2}
  \otfLigS{mm3} \otfLigS{km3} \otfLigS{sec} \otfLigS{min}
  \otfLigS{cal} \otfLigS{kcal} \otfLigS{dB} \otfLigS{m}
  \otfLigS{g} \otfLigS{F} \otfLigS{TB} \otfLigS{FAX}
  \otfLigS{ohm} \otfLigS{AM} \otfLigS{KK} \otfLigS{No}
  \otfLigS{PH} \otfLigS{PM} \otfLigS{PR} \otfLigS{tel}
  \otfLigS{tm} \otfLigS{VS} \otfLigS{a/c} \otfLigS{a.m.}
  \otfLigS{c/c} \otfLigS{c.c.} \otfLigS{c/o} \otfLigS{dl*}
  \otfLigS{hPa} \otfLigS{kl*} \otfLigS{l*} \otfLigS{microg}
  \otfLigS{microm} \otfLigS{ml*} \otfLigS{m/m} \otfLigS{n/m}
  \otfLigS{pH} \otfLigS{p.m.} \otfLigS{mho} %\otfLigS{}
  \\
  \hline
 \end{tabular}
\end{table}

\subsection{濁音・拗音・丸付き/括弧付き文字を簡単に出力する}
\CI{゛,゜,!,○,(}%
\begin{usage}
\゛$\<濁音化したい文字>$    % $\text{例:\゛う}$
\゜$\<半濁音化したい文字>$  % $\text{例:\゜か}$
\!$\<拗音化したい文字>$    % $\text{例:\!ヌ}$
\○$\<丸付きにする文字>$    % $\text{例:\○秘}$
\($\<括弧付きにする文字>$)% $\text{例:\(労)}$
\end{usage}

\begin{usage}
\○$\<引数>$ \●$\<引数>$ \□$\<引数>$ 
\■$\<引数>$ \◇$\<引数>$ \◆$\<引数>$ 
\end{usage}
あ〜ん,ア〜ン,日〜休,半角のa〜z,半角のA〜Zでも合字が出せます.
上の表と同じ出力が得られますが,違う入力方法になります.

%OK: \ajLig{ohm*} \ajLig{mho*} 
%
%OK: \ajLig{euro} \ajLig{euro*} 

%\ajLig{JIS} \ajLig{JAS} 

%NG: \ajLig{No*}

%NG: \ajLig{ppb}  \ajLig{ppm}  \ajLig{'S}  \ajLig{H2} 
% \ajLig{O2}  \ajLig{Ox}  \ajLig{Nx}  \ajLig{Q2} 
% \ajLig{Jr.}  \ajLig{Dr.}  \ajLig{ガル}  \ajLig{グレイ} 
% \ajLig{クローナ}  \ajLig{シーベルト}  \ajLig{シュケル}  \ajLig{ジュール} 
% \ajLig{デシベル}  \ajLig{ドット}  \ajLig{バイト}  \ajLig{ビット} 
% \ajLig{ベクトル}  \ajLig{ボー}  \ajLig{ランド}  \ajLig{リンギット} 
% \ajLig{より} \ajLig{升} \ajLig{コト}  \ajLig{年}  

%NG:  \ajLig{□:A} \ajLig{:B} \ajLig{:C} \ajLig{:D} 
% \ajLig{:E} \ajLig{:F} \ajLig{:終} \ajLig{:CL} 
% \ajLig{:KCL} \ajLig{:BEL} \ajLig{:AS} \ajLig{:AM} 

%\def\@ajmojifam{□}
%\@ajligaturedef{:A}{:B}{:C}{:D}{:E}{:F}{:終}\@nil%AJ1-6
%\@ajligaturedef{:CL}{:KCL}{:BEL}{:AS}{:AM}{:段}{:ゴ}{:ミ}\@nil%AJ1-6

%NG: \ajLig{:段} \ajLig{:ゴ} \ajLig{:ミ} \ajLig{News} \ajLig{PV} 
% \ajLig{MV} \ajLig{SS} \ajLig{S1} \ajLig{S2} \ajLig{S3} 
% \ajLig{HV} 

%OK: \ajPICT{Club} \ajPICT{Heart} \ajPICT{Spade} \ajPICT{Diamond}
% \ajPICT{Club*} \ajPICT{Heart*} \ajPICT{Spade*} \ajPICT{Diamond*}

%OK: \ajPICT{〒} \ajPICT{晴} \ajPICT{曇} \ajPICT{雨}
% \ajPICT{雪} \ajPICT{→} \ajPICT{←} \ajPICT{↑}
% \ajPICT{↓} \ajPICT{野球} \ajPICT{湯} \ajPICT{花}
% \ajPICT{花*} 

%?: \ajPICT{電話} 

%NG: \ajPICT{サッカー} 
%
%?: \ajLig{ほか} \ajLig{▽▽} \ajLig{▽〒}
% \ajLig{△!} \ajLig{■◇} %\ajLig{} \ajLig{}
% \ajLig{} \ajLig{} \ajLig{} \ajLig{}

%\△#1
%\▽#1
% \※花 

%\def\@ajmojifam{◇}
%\@ajligaturedef {News}天再新映声前後終立交{ほか}劇司解株気二多文手{PV}{MV}双{SS}{S1}{S2}{S3}デ{HV}\@nil%AJ1-6

%hoge: \ajHishi{1}
%\ajHishi{2}
%\ajHishi{3}
%\ajHishi{4}

%hoge: %\ajFrac{1}{4}
% \ajFrac{1}{2} \ajFrac{3}{4} \ajFrac{1}{8} \ajFrac{3}{8} 
% \ajFrac{5}{8} \ajFrac{7}{8} \ajFrac{1}{3} \ajFrac{2}{3}
% \ajFrac{}{} \ajFrac{}{} \ajFrac{}{} \ajFrac{}{}
%	\or\ifcase#1\or9826\fi 1
% \ajFrac{1}{1}
% \ajFrac{1}{2} %\ajFrac{2}{2}
% \ajFrac{1}{3} \ajFrac{2}{3} %\ajFrac{3}{3}
% \ajFrac{1}{4} \ajFrac{2}{4} \ajFrac{3}{4} %\ajFrac{4}{4}
% \ajFrac{1}{5} \ajFrac{2}{5} \ajFrac{3}{5} \ajFrac{4}{5} %\ajFrac{5}{5}
% \ajFrac16 \ajFrac26 \ajFrac36 \ajFrac46 \ajFrac56 %\ajFrac66 
% \ajFrac17 \ajFrac27 \ajFrac37 \ajFrac47 \ajFrac57 \ajFrac67 
% \ajFrac18 \ajFrac28 \ajFrac38 \ajFrac48 \ajFrac58 \ajFrac68 \ajFrac78
% \ajFrac19 \ajFrac29 \ajFrac39 \ajFrac49 \ajFrac59 
% \ajFrac69 \ajFrac79 \ajFrac89 
% \ajFrac1{10} \ajFrac2{10} \ajFrac3{10} \ajFrac4{10} \ajFrac5{10} 
% \ajFrac6{10} \ajFrac7{10} \ajFrac8{10} \ajFrac9{10}
% \ajFrac1{11} \ajFrac2{11} \ajFrac3{11} \ajFrac4{11} \ajFrac5{11}
% \ajFrac6{11} \ajFrac7{11} \ajFrac8{11} \ajFrac9{11} \ajFrac{10}{11}
% \ajFrac1{12} \ajFrac2{12} \ajFrac3{12} \ajFrac4{12} \ajFrac5{12} 
% \ajFrac6{12} \ajFrac7{12} \ajFrac8{12} \ajFrac9{12} \ajFrac{10}{12} 
% \ajFrac{11}{12}

%hoge: \ajFrac{1}{100} 

%hoge: \ajArrow{RIGHT*} \ajArrow {LEFT*} \ajArrow{UP*} \ajArrow{DOWN*}

%hoge: 
%\ajArrow{DOWN} \ajArrow{UP} \ajArrow{LEFT} \ajArrow{RIGHT}

%hoge: \ajArrow{RightHand} \ajArrow{LeftHand} \ajArrow{UpHand} 
%\ajArrow{DownHand}

%hoge: 
%\ajArrow{Left/Right} \ajArrow{Right/Left} \ajArrow{Up/Down} 
%\ajArrow{Down/Up}

%hoge: 
%\ajArrow{LeftScissors} \ajArrow{RightScissors} \ajArrow{UpScissors}
%\ajArrow{DownScissors}

%hoge: 
%\ajArrow{LeftTriangle} \ajArrow{RightTriangle} 
%\ajArrow{LeftTriangle*} \ajArrow{RightTriangle*} 
%\ajArrow{LeftAngle} \ajArrow{RightAngle}

%\ajArrow{Left} \ajArrow{Right} \ajArrow{Up} \ajArrow{Down} 
%\ajArrow{LeftDouble}
%\ajArrow{RightDouble}

%\ajArrow{LeftRight*} 
%\ajArrow{LeftRightDouble}
%%\ajArrow{RightDown} \ajArrow{LeftDown}
%%\ajArrow{LeftUp} \ajArrow{RightUp} 
%\ajArrow{Right/Left*}
%\ajArrow{Left/Right*} \ajArrow{Right/Left+} \ajArrow{Down/Up+}
%\ajArrow{Left+} \ajArrow{Right+} \ajArrow{Up+} \ajArrow{Down+}
%\ajArrow{LeftRight+} \ajArrow{UpDown+}

%hoge:  \ajArrow{UpAngle}
%\ajArrow{DownAngle} \ajArrow{LeftAngle*} \ajArrow{RightAngle*} 
%\ajArrow{UpAngle*} \ajArrow{DownAngle*}
%
%\ajArrow{RightUp*}
%\ajArrow{RightDown*}
%
%良く使われる異字体: 
%\ajMayuHama, \ajTatsuSaki, \ajHashigoTaka, \ajTsuchiYoshi

\begin{table}[htbp]
 \centering
 \setcounter{otfligline}{0}
 \setcounter{otfligrow}{2}
  \caption{\sty{OTF}のその他の記号類}\tablab{OTFのその他の記号類}
 \begin{tabular}{*4l}
  \hline
  引数&合字&引数&合字\\%&引数&合字\\%&引数&合字\\
  \hline
 \otfAj{SenteMark}
 \otfAj{GoteMark}
 \otfAj{Club}
 \otfAj{Heart}
 \otfAj{Spade}
 \otfAj{Diamond}
 \otfAj{varClub}
 \otfAj{varHeart}
 \otfAj{varSpade}
 \otfAj{varDiamond}
 \otfAj{Phone}
 \otfAj{Postal}
 \otfAj{varPostal}
 \otfAj{Sun}
 \otfAj{Cloud}
 \otfAj{Umbrella}
 \otfAj{Snowman}
 \otfAj{JIS}
 \otfAj{JAS}
 \otfAj{Ball}
 \otfAj{HotSpring}
 \otfAj{WhiteSesame}
 \otfAj{BlackSesame}
 \otfAj{WhiteFlorette}
 \otfAj{BlackFlorette}
 \otfAj{RightBArrow}
 \otfAj{LeftBArrow}
 \otfAj{UpBArrow}
 \otfAj{DownBArrow}
 \otfAj{RightHand}
 \otfAj{LeftHand}
 \otfAj{UpHand}
 \otfAj{DownHand}
 \otfAj{RightScissors}
 \otfAj{LeftScissors}
 \otfAj{UpScissors}
 \otfAj{DownScissors}
 \otfAj{RightWArrow}
 \otfAj{LeftWArrow}
 \otfAj{UpWArrow}
 \otfAj{DownWArrow}
 \otfAj{RightDownArrow}
 \otfAj{LeftDownArrow}
 \otfAj{LeftUpArrow}
 \otfAj{RightUpArrow}
  \\ \setcounter{otfligline}{0}
 \otfAj{Masu}
 \otfAj{Yori}
 \otfAj{Koto}
 \otfAj{Uta}
 \otfAj{CommandKey}
 \otfAj{ReturnKey}
 \otfAj{Checkmark}
 \otfAj{VisibleSpace}
 \hline
 \end{tabular}
\end{table}


%          OTF
\endgroup
