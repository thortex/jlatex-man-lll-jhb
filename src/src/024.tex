\subsection{クラスファイルで提供されているスタイルを切り替える}
\begin{usage}
 \pagestyle{$\<スタイル>$}% 該当ページ以降を変更する
 \thispagestyle{$\<スタイル>$}% 該当ページのみを変更する
\end{usage}

\begin{description}
 \item[empty]  
 \item[plain] 
 \item[headings] 
 \item[myheadings] 
\end{description}

%\subsection{ヘッダやフッタの設定その1}\indindz{番号}{ページ}%
ヘッダやフッタなどに出力されるページ番号などを\zindind{ページ}{の番号}
変更したいときがあると思います.
\begin{usage}
\pagestyle{$\<表示形式>$}
\end{usage}

\C{pagestyle}命令を記述したページから指定した\Z{表示形式}に変更されま
す.指定できる形式は\tabref{pagestyle}の通りです.


\begin{table}[htbp]
\begin{center}
\caption{ヘッダやフッタの指定}\tablab{pagestyle}
\begin{tabular}{ll}
\TR
\Th{命令}        & \Th{内容} \\
\MR
\str{empty}      & ページ番号を表示しない                \\
\str{plain}      & フッタ中央部に表示する              \\
\str{headings}   & ヘッダにページ番号と章・節名を表示する\\
\str{myheadings} & ユーザ定義の表示形式にする          \\
\BR
\end{tabular}
\end{center}
\end{table}

\qu{\str{myheadings}}では二つの命令によってヘッダの出力を指定します.

\begin{usage}
\markright{$\<ヘッダ>$}
\markboth{$\<偶数ヘッダ>$}{$\<奇数ヘッダ>$}
\end{usage}


片面印刷のときに \cmd{markright}を使います.
両面印刷には \C{markboth}を使います.
2007年度版の「○○△△大学」の論文集を
作成しているのであれば,例えば次のようにします.

\begin{intext}
\pagestyle{myheadings}
\markboth{○○△△大学論文集}{2007年度版}
\end{intext}


任意の1ページだけのヘッダ・フッタは,
次のようにする事でそのページだけ変えられます.

\begin{usage}
\thispagestyle{$\<表示形式>$}
\end{usage}


ページ番号の表示形式を変更するには \C{pagenumbering}命令を使います.

\begin{usage}
\pagenumbering{$\<表示形式>$}
\end{usage}

その場所から指定した形式で1ページ目からカウントしてページ番号を表示しま
す.指定できる表示形式は\tabref{pagenumber}の通りです.

\begin{table}[htbp]
\begin{center}
\caption{ページ番号の種類の指定}\tablab{pagenumber}
\begin{tabular}{lll}
\TR
\Th{形式}          & \Th{内容}          & \Th{出力例}         \\
\MR
\verb+arabic+ & \Z{アラビア数字}       & 1, 2, 3,\,\ldots \\
\verb+roan+   & 小文字の\Z{ローマ数字}   & i, ii, iii,\,\ldots\\
\verb+Roman+  & 大文字のローマ数字   & I, II, III,\,\ldots\\
\verb+alph+   & アルファベット小文字& a, b, c,\,\ldots,\, z\\
\verb+Alph+   & アルファベット大文字& A, B, C,\,\ldots,\, Z\\
\BR
\end{tabular}
\end{center}
\end{table}
ここで注意する事はアルファベットにした場合は,
最大26ページまでしかカウントできないという事
です.27以上になった場合の対策は別にする事に
なります.

%もっと詳細なヘッダ・フッタ定義するときは自分で
%\cmd{ps@なんとか}というコマンドを定義してこれを
%\begin{intext}
%pagestyle{なんとか} 
%\end{intext}
%のように使います.本書で使われているスタイルは
%以下の通りです.\hito{奥村}{晴彦}の\cls{jsclasses}
%を少し変更しただけです.
%\begin{intext}
%\newcommand{\ps@myhead}{%
%  \let\@oddfoot\@empty
%  \let\@evenfoot\@empty
%  \def\@evenhead{%
%    \if@mparswitch \hss \fi
%    \underline{\hbox to \fullwidth{\autoxspacing
%        \textbf{\thepage}\hskip3zw\leftmark\hfill\@title}}%
%    \if@mparswitch\else \hss \fi}%
%  \def\@oddhead{\underline{\hbox to \fullwidth{\autoxspacing
%        \@title\hfill{\if@twoside\rightmark\else\leftmark\fi}%
%	\hskip3zw\textbf{\thepage}}}\hss}%
%  \let\@mkboth\markboth
%  \def\chaptermark##1{\markboth{%
%    \ifnum \c@secnumdepth >\m@ne
%      \if@mainmatter
%        \@chapapp\thechapter\@chappos\hskip1zw
%      \fi
%    \fi
%    ##1}{}}%
%  \def\sectionmark##1{\markright{%
%    \ifnum \c@secnumdepth >\z@ \thesection \hskip1zw\fi
%    ##1}}}%
%\end{intext}
%これを別ファイル\fl{hoge.sty}などにまとめて\cmd{usepackage}
%するか,\cmd{makeatletter}と\cmd{makeatother}で囲むかで
%実際の出力を確認してみてください.
