%#!platex -kanji=utf8 hb.tex
\section{文字記号}

\begin{table}[htbp]
 \centering
 \glossary{"#@\hspace*{-1.2ex}\verb+"\#+}%"
 \glossary{"_@\hspace*{-1.2ex}\verb+"\_+}%"
 \glossary{"$@\hspace*{-1.2ex}\verb+"\$+}%"
 \glossary{"%@\hspace*{-1.2ex}\verb+"\%+}%"
 \glossary{"{1"}@\hspace*{-1.2ex}\protect\bgroup\verb+"\"{+"}}%
 \glossary{"{2"}@\hspace*{-1.2ex}"{\verb+"\"}+\protect\egroup}%
 \glossary{"&@"\hspace*{-1.2ex}"\verb+"\"&+}%"
 \index{ナンバー}%
 \index{ドル}%
 \index{パーセント}%
 \index{アンパサンド}%
 \index{波括弧}%
 \index{チルダ}%
 \index{ハット}%
 \index{バックスラッシュ}%
 \index{縦棒}%
 \index{小なり}%
 \index{大なり}
 \caption{アスキー文字}\tablab{アスキー文字}
 \begin{tabular}{*{2}{l@{\thickspace}l}}
  \hline
  \# & \verb|\#| & \T{textasciitilde}\\
  \$ & \verb|\$| & \T{textasciicircum}\\
  \% & \verb|\%| & \T{textbackslash}\\
  \& & \verb|\&| & \T{textbar}\\
  \_ & \verb|\_| & \T{textless}\\
  \{ & \verb|\{| & \T{textgreater}\\
  \} & \verb|\}| \\  
  \hline
 \end{tabular} 
\end{table}

\LaTeX では10個の半角記号(\W{アスキー文字})は\W{特殊な文字}として
解釈されてしまうため,面倒でも\tabref{アスキー文字}のコマンドを
用いる必要があります.
\index{"#@\verb+#+}\glossary{"#@\verb+#+}%
\index{"$@\verb+$+}\glossary{"$@\verb+$+}%
\index{"&@\verb+&+}\glossary{"&@\verb+&+}%
\index{"\@\verb+\+}\glossary{"\@\verb+\+}%
\index{"^@\verb+^+}\glossary{"^@\verb+^+}%
\index{"_@\verb+_+}\glossary{"_@\verb+_+}%
\index{"{1"}@\protect\bgroup\verb+"{+"}}%
\glossary{"{1"}@\protect\bgroup\verb+"{+"}}%
\index{"{2"}@"{\verb+"}+\protect\egroup}%
\glossary{"{2"}@"{\verb+"}+\protect\egroup}%
\index{"~@\verb+~+}%
\glossary{"~@\verb+~+}%
\index{%@\verb+%+}%"
\glossary{%@\verb+%+}%"

%\begin{quote}
%\verb|# $ % & _ { } ~ ^ \ |
%\end{quote}

さらに3個の記号は出力が違う文字記号になります.%"
\index{"|@\texttt{\symbol{'174}}}%"
\index{"<@\verb+<+}%"
\index{">@\verb+>+}%"
%\begin{quote}
\str| \str< \str> はそれぞれ --- !` ?`
%\end{quote}
として表示されてしまいます.


\begin{table}[htbp] \centering
\index{見える空白}%
\index{ダガー}\index{短剣符}%
\index{ダブルダガー}\index{二重短剣符}%
\index{セクション}\index{章標}%
\index{パラグラフ}\index{段標}%
\index{たんらくきこう@段落記号\hskip1em(\P)}\indindz{記号}{段落}%
\index{せつきこう@節記号\hskip1em(\S)}\indindz{記号}{節}%
\glossary{"!@\verb+"!`+\hskip1em("!`)}%"
\glossary{?@\verb+?`+\hskip1em(?`)}%
\index{シャープS}%
\index{スラッシュ付きO}%
\index{伏字L}%
\index{合字!OEの\zdash}%
\index{合字!AEの\zdash}%
\index{ポンド}%
\index{リングA}%
\indindz{記号}{特殊な}%
\zindind{点}{のないi}\zindind{点}{のないj}%
\indindz{記号}{特殊な}%
\caption{特殊な文字記号}\tablab{app:tokusyukigou}%
\begin{tabular}{*{5}{l@{\thickspace}l}}
 \hline
\T{aa} & \T{AA} & \T{ae} & \T{AE} &\T{oe}\\
\T{OE} & \T{l} & \T{L} & \T{o} & \T{O}\\
\T{i}  & \T{j} & \T{ss} & \T{SS} & \T{S}\\
\T{P}  & \T{dag} & \T{ddag} & \T{pounds} & !`&\verb|!`|  \\
 {?`}&\verb|?`| & \\ 
\end{tabular}\\
\begin{tabular}{*{2}{l@{\thickspace}l}}
% \hline
\T{textvisiblespace} & \T{copyright}\\
\T{textregistered} & \T{texttrademark}\\
 \hline
\end{tabular}
\end{table}

\begin{table}
 \begin{center}
  \caption{その他の文字記号}
  \begin{tabular}{*{2}{l@{\thickspace}l}}
   \hline
   \T{textbraceleft}  &
   \T{textbraceright}\\
   \T{textunderscore}&
   \T{textendash}\\
   \T{textemdash}&
   \T{textellipsis}\\
   \T{textquoteleft}&
   \T{textquoteright}\\
   \T{textquotedblleft}&
   \T{textquotedblright}\\
   \T{textquestiondown}&
   \T{textexclamdown}\\
   \hline
  \end{tabular}   
 \end{center}
\end{table}

\tabref{T1エンコーディングで使用できる文字記号}の記号は\Y{fontenc}
パッケージを\qu{\Option{T1}}というオプション付きで読み込むと出力できます.
\begin{usage}
\usepackage[T1]{fontenc} 
\end{usage}
このとき,\sty{pxfonts}や\sty{txfonts},\Y{lmodern},\Y{type1ec}パッ
ケージを読み込むとアウトラインフォントがPDFに埋め込まれるようになります.

\begin{table}  \centering
 \caption{\option{T1}エンコーディングで使用できる文字記号}
 \tablab{T1エンコーディングで使用できる文字記号}
\begin{tabular}{*{2}{l@{\thickspace}l}}
 \hline
 \T{DJ} &\T{guillemotleft}\\
 \T{ng} &\T{guillemotright}\\
 \T{NG} &\T{guilsinglleft}\\
 \T{th} &\T{guilsinglright}\\
 \T{TH} &\T{quotedblbase}\\
 \T{dh} &\T{quotesinglbase}\\
 \T{DH} &\T{textquotedbl}\\
 \T{dj} \\
 \hline
\end{tabular}
\end{table}

\begin{table}
 \caption{ダイアクリティカルマーク(アクセント)}%
 \tablab{ダイアクリティカルマーク}%
 \begin{center}
  \index{ダイアクリティカルマーク}%
  \indindz{記号}{アクセント}\index{アクセント記号}%
  \indindz{アクセント}{文中の}%
  \makeatletter
  \glossary{""@\hspace*{-1.2ex}\verb+\""+\hskip1em(\""u)}%"
  \glossary{"~@\hspace*{-1.2ex}\verb+\~+\hskip1em(\~n)}%"
  \glossary{"^@\hspace*{-1.2ex}\verb+\^+\hskip1em(\^o)}%"
 \newcommand*\A[4][]{%
   \glossary{#2@\hspace*{-1.2ex}\texttt{\protect\BS#2}%
      \hskip1em(\csname#2\endcsname{#3}\relax)}%
  \index{#1}%
   #1&\texttt{\BS\string #2}&\csname#2\endcsname{#3}&%
   \texttt{\BS\string#2\@charlb#3\@charrb}&{#4}%
  }%
 \newcommand*\NA[4][]{%
  \index{#1}%
   #1&\texttt{\BS\string #2}&\csname#2\endcsname{#3}&%
   \texttt{\BS\string#2\@charlb#3\@charrb}&{#4}%
  }%
 \makeatother
  \index{acute}\index{breve}\index{circumflex}\index{cedilla}%
  \index{hungarumlaut}\index{double acute}\index{grave}\index{caron}%
  \index{macron}\index{dot accent}\index{ring}\index{tie}%
  \index{tilde}\index{umlaut}\index{subscript dot}\index{under dot}%
  \index{underscore}%
  \index{ogonek}%
 \begin{tabular}{lllll}
  \toprule
  名称 & 命令 & 出力例 & 入力例 & 別称 \\
  \midrule
  \A[アキュート]         {'}{a} {\W{揚音符}}\\%
  \A[ブレーヴェ]         {u}{u} {\W{短音府}}\\% 
  \NA[サーカムフレックス]{^}{a} {\W{抑揚音符}}\\% これは noindex
  \A[セディーユ]         {c}{C} {\W{鈎形符}}\\ %
  \A[ダブルアキュート]   {H}{o} {}\\% %長短音符?
  \A[グレイヴ]           {`}{a} {\W{抑音符}}\\% 
  \A[ハーチェク]         {v}{a} {\W{キャロン}}\\% 
  \A[マクロン]           {=}{e} {\W{長音符}}\\% 
  \A[ドット]             {.}{a} {}\\% 
  \A[リング]             {r}{o} {}\\% 
  \A[タイ]               {t}{oo}{}\\% 
  \maketildeletter
  \NA[チルダ]            {~}{o} {\W{波音符}}\\%  これは noindex
  \maketildeother
  \NA[ウムラウト]        {"}{a} {\W{分音符}}\\%" これは noindex
  \A[下付きドット]       {d}{t} {}\\% 
  \A[下線]               {b}{z} {}\\% 
  \midrule
  \A[点なしj]            {j}{} {}\\% 
  \A[点なしi]            {i}{} {}\\% 
  \midrule
  \A[オゴネク*]          {k}{c} {}\\ % T1 のみ
  \bottomrule
 \end{tabular}
 \end{center}
\end{table}
%
\tabref{ダイアクリティカルマーク}のオゴネクはT1エンコーディングで出力可
能です. \tabref{T1エンコーディングで使用できる文字記号}の説明を参照して
ください.
\begin{inout}
J\"org mu\ss\ ein Gel\"ande f\"ur seine Fabrik erwerben.
\end{inout}

\subsection{\sty{pifont}}
\begin{table}[htbp]
\begin{small}
 \makeatletter
 \newcommand*\fntsymb[1]{{\usefont{U}{pzd}{m}{n}\char#1}}
 \newcommand*\fnttbl[1]{%
   \fntsymb{'#10} & \fntsymb{'#11} &  \fntsymb{'#12} & \fntsymb{'#13} &%
   \fntsymb{'#14} & \fntsymb{'#15} &  \fntsymb{'#16} & \fntsymb{'#17} &%
 }
 \makeatother
 \begin{center}
  \caption{\sty{pifont} (\W{ZapDingbats}) 中の記号一覧}\tablab{ZapDingbats}
  \begin{tabular}{c|*8c|c}
   \textit{x}
   & \textit{'0} & \textit{'1} & \textit{'2} & \textit{'3}
   & \textit{'4} & \textit{'5} & \textit{'6} & \textit{'7}
   &  \\ \hline
   \textit{'04x} & \fnttbl{04} \texttt{"2x} \\ \cline{1-9}
   \textit{'05x} & \fnttbl{05} \texttt{"2y} \\ \hline
   \textit{'06x} & \fnttbl{06} \texttt{"3x} \\ \cline{1-9}
   \textit{'07x} & \fnttbl{07} \texttt{"3y} \\ \hline
   \textit{'10x} & \fnttbl{10} \texttt{"4x} \\ \cline{1-9}
   \textit{'11x} & \fnttbl{11} \texttt{"4y} \\ \hline
   \textit{'12x} & \fnttbl{12} \texttt{"5x} \\ \cline{1-9}
   \textit{'13x} & \fnttbl{13} \texttt{"5y} \\ \hline
   \textit{'14x} & \fnttbl{14} \texttt{"6x} \\ \cline{1-9}
   \textit{'15x} & \fnttbl{15} \texttt{"6y} \\ \hline
   \textit{'16x} & \fnttbl{16} \texttt{"7x} \\ \cline{1-9}
   \textit{'17x} & \fnttbl{17} \texttt{"7y}\\
   \hline
   \multicolumn{10}{c}{}\\
   \hline
   \textit{'24x} & \fnttbl{24} \texttt{"Ax} \\ \cline{1-9}
   \textit{'25x} & \fnttbl{25} \texttt{"Ay}\\\hline
   \textit{'26x} & \fnttbl{26} \texttt{"Bx} \\ \cline{1-9}
   \textit{'27x} & \fnttbl{27} \texttt{"By}\\\hline
   \textit{'30x} & \fnttbl{30} \texttt{"Cx} \\ \cline{1-9}
   \textit{'31x} & \fnttbl{31} \texttt{"Cy}\\\hline
   \textit{'32x} & \fnttbl{32} \texttt{"Dx} \\ \cline{1-9}
   \textit{'33x} & \fnttbl{33} \texttt{"Dy}\\\hline
   \textit{'34x} & \fnttbl{34} \texttt{"Ex} \\ \cline{1-9}
   \textit{'35x} & \fnttbl{35} \texttt{"Ey}\\\hline
   \textit{'36x} & \fnttbl{36} \texttt{"Fx} \\ \cline{1-9}
   \textit{'37x} & \fnttbl{37} \texttt{"Fy}\\\hline
   & \texttt{"8} & \texttt{"9} & \texttt{"A} & \texttt{"B}
   & \texttt{"C} & \texttt{"D} & \texttt{"E} & \texttt{"F}
   & \texttt{y}
  \end{tabular}
 \end{center}
\end{small}
\end{table}
\Y{pifont}パッケージに含まれる記号を使うには次のようにします.
\begin{usage}
 \usepackage{pifont}
 \ding{$\<文字コード>$}
\end{usage}
\val{文字コード}は10進数,8進数,16進数で指定可能です.
\tabref{ZapDingbats}には左側に8進数,右側に16進数の数値を示してあります.
飛行機の記号を出力するために,
10進数では40,8進数では\textit{'050},16進数では \verb|"28| となっていますので,%"
次のようにします.
\begin{inout}
\ding{40} $=$ \ding{'050} $=$ \ding{"28}%"
\end{inout}

\begin{usage}
\dingfill{$\<文字コード>$}% 記号で1行の残りの部分を埋める
\dingline{$\<文字コード>$}% 記号で1行全部を埋める
\end{usage}

\begin{inout}
\dingfill{'044} ※ここから切り取ってください.
\dingline{'134} 
\end{inout}

\begin{usage}
\begin{dinglist}{$\<項目>$}% itemize と似た機能です
 \item $\<項目>$
\end{dinglist}

\begin{dingautolist}{$\<項目>$}% enumerate と似た機能です
 \item $\<項目>$
\end{dingautolist}
\end{usage}


\subsection{\Y{textcomp}}
\begin{usage}
\usepackage[T1]{fontenc}
\usepackage{textcomp}
\usepackage{mathcomp}% 数式中で使う時 
% \textleaf であれば \tcleaf のように短い名前(\tc)で参照
\end{usage}

%\tablab{app:textcomp}
\begingroup%{table}[htbp]\index{著作権記号}
%\newcommand*\TC[1]{\mynameuse{#1}&\C{#1}}%
\newcommand*\TCS[2]{\T{#1}&\T{#2}\\}
\begin{longtable}{*{2}{l@{\thickspace}l}}
  \caption{\textsf{textcomp}で使える記号}\\
 \hline
 \endfirsthead
 \hline
 \multicolumn{4}{c}{前ページからの続きです}\\
 \endhead
 \multicolumn{4}{c}{次ページへ続きます}\\
 \hline
 \endfoot
 \hline
 \endlastfoot
  \TCS{textquotestraightbase}{textquotestraightdblbase}
  \TCS{texttwelveudash}{textthreequartersemdash}
  \TCS{textleftarrow}{textrightarrow}
  \TCS{textblank}{textdollar}
  \TCS{textquotesingle}{textasteriskcentered}
  \TCS{textdblhyphen}{textfractionsolidus}
  \TCS{textzerooldstyle}{textoneoldstyle}
  \TCS{texttwooldstyle}{textthreeoldstyle}
  \TCS{textfouroldstyle}{textfiveoldstyle}
  \TCS{textsixoldstyle}{textsevenoldstyle}
  \TCS{texteightoldstyle}{textnineoldstyle}
  \TCS{textlangle}{textminus}
  \TCS{textrangle}{textmho}
  \TCS{textbigcircle}{textohm}
  \TCS{textlbrackdbl}{textrbrackdbl}
  \TCS{textuparrow}{textdownarrow}
  \TCS{textasciigrave}{textborn}
  \TCS{textdivorced}{textdied}
  \TCS{textleaf}{textmarried}
  \TCS{textmusicalnote}{texttildelow}
  \TCS{textdblhyphenchar}{textasciibreve}
  \TCS{textasciicaron}{textgravedbl}
  \TCS{textacutedbl}{textdagger}
  \TCS{textdaggerdbl}{textbardbl}
  \TCS{textperthousand}{textbullet}
  \TCS{textcelsius}{textdollaroldstyle}
  \TCS{textcentoldstyle}{textflorin}
  \TCS{textcolonmonetary}{textwon}
  \TCS{textnaira}{textguarani}
  \TCS{textpeso}{textlira}
  \TCS{textrecipe}{textinterrobang}
  \TCS{textinterrobangdown}{textdong}
  \TCS{texttrademark}{textpertenthousand}
  \TCS{textpilcrow}{textbaht}
  \TCS{textnumero}{textdiscount}
  \TCS{textestimated}{textopenbullet}
  \TCS{textservicemark}{textlquill}
  \TCS{textrquill}{textcent}
  \TCS{textsterling}{textcurrency}
  \TCS{textyen}{textbrokenbar}
  \TCS{textsection}{textasciidieresis}
  \TCS{textcopyright}{textordfeminine}
  \TCS{textcopyleft}{textlnot}
  \TCS{textcircledP}{textregistered}
  \TCS{textasciimacron}{textdegree}
  \TCS{textpm}{texttwosuperior}
  \TCS{textthreesuperior}{textasciiacute}
  \TCS{textmu}{textparagraph}
  \TCS{textperiodcentered}{textreferencemark}
  \TCS{textonesuperior}{textordmasculine}
  \TCS{textsurd}{textonequarter}
  \TCS{textonehalf}{textthreequarters}
  \TCS{texteuro}{texttimes}
  \T{textdiv}
\end{longtable}
\endgroup



%・windingbats

%\section{ほげ}

