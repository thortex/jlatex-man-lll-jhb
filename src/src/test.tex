%#!platex -kanji=utf8 hb.tex
\chapter{chap序論}
ほげほげ
\section{secほげほげ}
ほげほえ

\subsection{subsec○○したい}

%\usage{documentclass}\opt{オプション}\arg{ファイル名}\opt{日付}

\cmdind{DeclareRobustCommand}

\begin{usage}
% ほげほげこれはコメント
\usepackage{$\<パッケージ名>$}
% ほげほげこれはコメント
\begin{document}
 \ruby{$\<漢字>$}{$\<フリガナ>$}
 これは文字列
\end{document}
\end{usage}


\subsection{文章を書きたい}

%\usage*{document}{文章}

\C{@hoge}
\C{@}

\begin{inout}
 入ryク
jss;adfklj
\LaTeX, \XeTeX, 
\end{inout}
s;jかsjdf;lさjkl;dsじゃkl;

\begin{inonly}
 ほげhごえほほぐぇおぐぇお
kasdj;ksadjfa;ldskjfl
jsafl;
\end{inonly}
%kfj;さldjfkらsdjfkじゃkdf;
\begin{outonly}
 入ryク
jss;adfklj
\LaTeX, \XeTeX,  
\end{outonly}

djskljさfjdさ;fksdjfs;lfk
\begin{inonly}
ほげほげほほddsあsfsf
inonly 
\end{inonly}
ほげほげ,これは罫線引いてよね

\begin{outonly*}
 入ryク
jss;adfklj
\LaTeX, \XeTeX,  
\end{outonly*}


ほぐぇあj;sdkfjs;f
\begin{intext}
  入ryク
 jss;adfklj
 \LaTeX, \XeTeX, 
\end{intext}
klあsjdkじゃds;lkふぁjkf