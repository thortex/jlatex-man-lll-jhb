\section{数学記号}

\begin{table}[htbp]
\begin{center}
\caption{ギリシャ小文字}\tablab{app:greek:lower}
\begin{tabular}{*{4}{c@{\thickspace\thinspace}l}}
 \hline
\M{alpha}   & \M{eta}    & \M{nu}    & \M{tau}     \\
\M{beta}    & \M{theta}  & \M{xi}    & \M{upsilon} \\
\M{gamma}   & \M{iota}   & $o$&o     & \M{phi}     \\
\M{delta}   & \M{kappa}  & \M{pi}    & \M{chi}     \\
\M{epsilon} & \M{lambda} & \M{rho}   & \M{psi}     \\
\M{zeta}    & \M{mu}     & \M{sigma} & \M{omega}   \\
  \hline
\end{tabular}
\end{center}
\end{table}

\begin{table}[htbp]
\begin{center}
 \makeatletter
 \newcommand*\LG[1]{\C{mathrm}\texttt{\@charlb#1\@charrb}}%
 \newcommand*\LGS[1]{$\mathrm{#1}$&\LG{#1}}%
 \makeatother
\caption{ギリシャ大文字}\tablab{app:greek:upper}
\begin{tabular}{*{4}{c@{\thickspace}l}}
 \hline
 \LGS{A}   & \LGS{H}    & \LGS{N}   & \LGS{T}\\
 \LGS{B}   & \M{Theta}  & \M{Xi}    & \M{Upsilon}\\
 \M{Gamma} & \LGS{I}    & \LGS{O}   & \M{Phi}\\
 \M{Delta} & \LGS{K}    & \M{Pi}    & \LGS{X}\\
 \LGS{E}   & \M{Lambda} & \LGS{P}   & \M{Psi}\\
 \LGS{Z}   & \LGS{M}    & \M{Sigma} & \M{Omega}\\
  \hline
\end{tabular}
\end{center}
\end{table}

%\begin{inout}
%\begin{eqnarray*}
%\cos^2\theta + \sin^2\theta & \neq & \cos^2x + \sin^2x 
%\end{eqnarray*}
%\end{inout}
%

%\begin{inout}
%\begin{eqnarray*}
%     A      & \neq & \mathrm{A}\\
%F(x)+C      & \neq & F(x)+ \mathrm{C}\\
%\mathit{foo}& \neq & \mathrm{foo}
%\end{eqnarray*}
%\end{inout}

\begin{table}[htbp]
 \begin{center}\zindind{ギリシャ文字}{の変体小文字}%
\caption{ギリシャ小文字の変体文字}\tablab{app:greek:lower:hen}
 \begin{tabular}{*{3}{c@{\thickspace\thinspace}l}}
  \hline
 \M{varepsilon} & \M{vartheta} & \M{varpi} \\
 \M{varrho}     & \M{varsigma} & \M{varphi}\\
  \hline
 \end{tabular}
 \end{center}
\end{table}


\begin{table}[htbp]
\begin{center}
\caption{大型演算子}\indindz{演算子}{大型}\index{大型演算子}%
\begin{tabular}{*{4}{c@{\thickspace\thinspace}l}}
 \hline
\M{sum}    & \M{oint}     & \M{bigvee}   & \M{bigoplus}  \\
\M{prod}   & \M{bigcup}   & \M{bigwedge} & \M{bigotimes} \\
\M{coprod} & \M{bigcap}   &    &          & \M{bigodot}  \\
\M{int}    & \M{bigsqcup} &    &          & \M{biguplus} \\
  \hline
\end{tabular}
\end{center}
\end{table}


\begin{table}[htbp]
\begin{center}
\caption{括弧の大きさを指定する例}
\tablab{app:ookiikakko}
\index{"/@"\verb+"/+}%"
\begin{tabular}{*{5}{c@{\thickspace}l}}
 \hline
\m{/}      & \m{(}       & \m{)}       & \m{|}       &
  $\|$      & \verb+\|+\\
\m{\big/}  & \m{\bigl(}  & \m{\bigr)}  & \m{\bigm|}  &
  $\bigm\|$ & \verb+\bigm\|+\\[4pt]
\m{\Big/}  & \m{\Bigl(}  & \m{\Bigr)}  & \m{\Bigm|}  &
  $\Bigm\|$ & \verb+\Bigm\|+\\[5pt]
\m{\bigg/} & \m{\biggl(} & \m{\biggr)} & \m{\biggm|} &
  $\biggm\|$&  \verb+\biggm\|+\\[6pt]
\m{\Bigg/} & \m{\Biggl(} & \m{\Biggr)} & \m{\Biggm|} &
  $\Biggm\|$&  \verb+\Biggm\|+\\[7pt]
 \hline
\end{tabular}
\end{center}
\end{table}

\begin{table}[htbp]
\begin{center}
\caption{主な区切り記号}\tablab{app:brace1}
\glossary{"{1"}@\hspace*{-1.2ex}\protect\bgroup\verb+"\"{+"}}%
\glossary{"{2"}@\hspace*{-1.2ex}"{\verb+"\"}+\protect\egroup}%
\glossary{"|@"\hspace*{-1.2ex}"\verb+"\"|+}%}
\index{"(@\verb+(+}%"
\index{")@\verb+)+}%
\index{"[@\verb+[+}%
\index{"]@\verb+]+}%
\index{"|@\texttt{\symbol{'174}}}%""
\begin{tabular}{*{4}{c@{\thickspace}l}}
 \hline
$($ &\verb+(+ & \M{rfloor}   & \M{updownarrow}& \M{lbrace}\\
$)$ &\verb+)+ & \M{lfloor}   & \M{Uparrow}&     \M{rceil}\\
$[$ &\verb+[+ & \M{arrowvert}& \M{Downarrow}&   \M{lceil}\\
$]$ &\verb+]+ & \M{Arrowvert}& \M{Updownarrow}& 
 $\big\lmoustache$&\BM{lmoustache}~${}^*$\\[2pt]
$\{$&\verb+\{+& \M{Vert}&      \M{backslash}&   
 $\big\rmoustache$&\BM{rmoustache}~${}^*$\\[2pt]
$\}$&\verb+\}+& \M{vert}&      \M{rangle}&      
 $\big\lgroup$&\BM{lgroup}~${}^*$\\[2pt]
$|$ &\verb+|+ & \M{uparrow}&   \M{langle}&      
 $\big\rgroup$&\BM{rgroup}~${}^*$\\[2pt]
$\|$&\verb+\|+& \M{downarrow}& \M{rbrace}&      
 $\big\bracevert$&\BM{bracevert}~${}^*$\\
  \hline
\end{tabular}
\\ {\small${}^{*}$\ 大型の区切り記号です.}
\end{center}
\end{table}
%}}}"

%\begin{inout}
% $<x, y> \neq \langle x, y\rangle$
%\end{inout}

%\begin{inout}
%\begin{displaymath}
%\left( \frac{1}{1+\frac{1}{1+x}} \right) 
%\end{displaymath}
%\end{inout}

%\begin{inout}
%\[ \left\lmoustache \left\{ 
%  \left(\frac{1}{x}+1\right)
%  +\left(\frac{1}{x^2}+2\right) 
%\right\} \right\rmoustache \]
%\end{inout}

%\begin{inout}
%\begin{displaymath}
% \left\uparrow \int f(x)dx 
%   \right\downarrow + \left\lgroup 
%     \int g(x)dx \right\rgroup
%\end{displaymath}
%\end{inout}


\begin{table}[htbp]
\begin{center}%
 \indindz{アクセント}{数式中の}%
 \zindind{ベクトル}{記号}%
 \makeatletter
 \newcommand*{\WA}[2]{%
 \glossary{#1@\hspace*{-1.2ex}\texttt{\protect\BS#1}%
 \hskip1em($\csname#1\endcsname{#2}$)}%
 $\csname#1\endcsname{#2}$ & %
 \texttt{\BS#1\@charlb\string#2\@charrb}}%
 \makeatother
\caption{小さいアクセント}\tablab{app:smallac}
\begin{tabular}{*{4}{c@{\thickspace\thinspace}l}}
 \hline
\WA{hat}{a}  & \WA{check}{a}& \WA{breve}{a}&\WA{acute}{a}\\
\WA{grave}{a}& \WA{tilde}{a}& \WA{bar}{a}  &\WA{dot}{a}  \\
\WA{ddot}{a} & \WA{vec}{a}  &       &      &         &   \\
  \hline
\end{tabular}
\end{center}
\end{table}

%\begin{inout}
%\( \vec{a}+\vec{b}\neq \vec{a+b} 
%   \neq \overrightarrow{a+b} \)
%\end{inout}



\begin{table}[htbp]
\begin{center}
\caption{大きいアクセント}\tablab{app:bigac}
\begin{tabular}{*{2}{c@{\thickspace\thinspace}l}}
 \hline
$\overline{m+M}$      &\C{overline}      & 
  $\overbrace{m+M}$& \C{overbrace}  \rule{0pt}{1.5em}\\
$\underline{m+M}$     &\C{underline}     &
  $\underbrace{m+M}$&  \C{underbrace} \rule{0pt}{1.5em}\\
$\overleftarrow{m+M}$ &\C{overleftarrow} & 
  $\widehat{m+M}$& \C{widehat} \rule{0pt}{1.5em}\\
$\overrightarrow{m+M}$&\C{overrightarrow}& 
  $\widetilde{m+M}$& \C{widetilde}  \rule{0pt}{1.5em}\\
  \hline
\end{tabular}
\end{center}
\end{table}

%\begin{inout}
%\begin{displaymath}
% \overbrace{a+b+c+d+e+f+g}^{h+i+j+k}+
% \underbrace{l+m+n}_{o+p+q}
%\end{displaymath} 
%\end{inout}


\begin{table}[htbp]
\begin{center}
 \caption{主な数学関数}\tablab{app:suugakukannsuu}
 \begin{tabular}{*{4}{c@{\thickspace\thinspace}l}}
  \hline
  \M{arccos} & \M{csc} & \M{ker}    & \M{min}  \\
  \M{arcsin} & \M{deg} & \M{lg}     & \M{Pr}   \\
  \M{arctan} & \M{det} & \M{liminf} & \M{sec}  \\
  \M{arg}    & \M{dim} & \M{limsup} & \M{sin}  \\
  \M{cos}    & \M{exp} & \M{lim}    & \M{sinh} \\
  \M{cosh}   & \M{gcd} & \M{ln}     & \M{sup}  \\
  \M{cot}    & \M{hom} & \M{log}    & \M{tan}  \\
  \M{coth}   & \M{inf} & \M{max}    & \M{tanh} \\
  \hline
 \end{tabular}
\end{center}
\end{table}

%\begin{inout}
%\[cos^2x+sin^2x \neq \cos^2x+\sin^2x\]
%\end{inout}

\begin{table}[htbp]
\begin{center}
\caption{関係子}\tablab{app:kannkeisi}
\begin{tabular}{*{4}{c@{\thickspace\thinspace}l}}
 \hline
\M{le}         & \M{in}        & \M{sqsupseteq} & \M{neq}    \\
\M{prec}       & \M{notin}     & \M{dashv}      & \M{doteq}  \\
\M{preceq}     & \M{ge}        & \M{ni}         & \M{propto} \\
\M{ll}         & \M{succ}      & \M{equiv}      & \M{models} \\
\M{subset}     & \M{succeq}    & \M{sim}        & \M{perp}   \\
\M{subseteq}   & \M{gg}        & \M{simeq}      & \M{mid}    \\
\M{sqsubseteq} & \M{supset}    & \M{asymp}      & \M{cong}   \\
\M{vdash}      & \M{supseteq}  & \M{approx}     & \M{bowtie} \\
\M{smile}      & \M{frown}     & \M{parallel}   &     &      \\
  \hline
\end{tabular}
\\{\footnotesize これらのコマンドの前に \C{not}を付ければ
その関係子の否定になります}
\end{center}
\end{table}



%\begin{inout}
%\( \sum^n_{i=0} a_i \neq a_o+
% {\displaystyle\sum^{n-1}_{i=1}a_i}\)
%\end{inout}

\begin{table}[htbp]
\begin{center}\indindz{演算子}{2項}%
\caption{2項演算子}\tablab{app:ennzannsi}
\begin{tabular}{*{4}{c@{\thickspace\thinspace}l}}
 \hline
\M{pm}     & \M{cdot}  & \M{setminus}        & \M{ominus} \\
\M{mp}     & \M{cap}   & \M{wr}              & \M{otimes} \\
\M{times}  & \M{cup}   & \M{diamond}         & \M{oslash} \\
\M{div}    & \M{uplus} & \M{bigtriangleup}   & \M{odot}   \\
\M{ast}    & \M{sqcap} & \M{bigtriangledown} & \M{bigcirc}\\
\M{star}   & \M{sqcup} & \M{triangleleft}    & \M{dagger} \\
\M{circ}   & \M{vee}   & \M{triangleright}   & \M{ddagger}\\
\M{bullet} & \M{wedge} & \M{oplus}           & \M{amalg}  \\
 \hline
\end{tabular}
\end{center}
\end{table}

%\begin{inout}
%\begin{displaymath}
%  (p\to r)\vee  (q\to s)
%\end{displaymath}
%\end{inout}

\begin{table}[htbp]
 \begin{center}
  \caption{点}\tablab{tenn}
  \index{点}%
  \index{3点リーダ}%
  \index{3点リーダ!中点\zdash}%
  \index{3点リーダ!下付き\zdash}%
  \index{...@\ldots(下付3点リーダ)}%
  \index{...@$\cdots$(中点3点リーダ)}%
  \begin{tabular}{*{5}{c@{\thickspace\thinspace}l}}
   \hline
   \M{dots}  & 
   \M{ldots} &  \M{cdots} & \M{vdots} &  \M{ddots}\\
   \hline
  \end{tabular}
 \end{center}
\end{table}

\begin{inout}
\begin{eqnarray*}
 (a_0+a_1+\cdots+a_n) &\neq& \{a_0,a_1,\ldots,a_n\}\\
 \{f_n\} &=& f_1, f_2, \dots, f_n
\end{eqnarray*}
\end{inout}
\cmd{ldots} や \cmd{cdots} 以外に \C{dots} という命令
もあります.これは前後の数式要素に応じて自動的に \cmd{ldots} と \cmd{cdots} を
切り替える命令です.
%\begin{inout}
%\( \{f_n\} = f_1, f_2, \dots, f_n \)
%\end{inout} 
%しばしば適切に選定されない場合がありますので,その場合は手動で
%対処します.

\begin{table}[htbp]
\begin{center}
 \caption{矢印}\index{矢印}
 \begin{tabular}{*{3}{c@{\thickspace\thinspace}l}}
  \hline
 \M{leftarrow}       & \M{rightarrow}      \\
 \M{uparrow}         & \M{downarrow}        \\
 \M{Leftarrow}       & \M{Rightarrow}      \\
%  \hline
 \M{Uparrow}         & \M{Downarrow}        \\
 \M{updownarrow}     &  \M{Updownarrow}       \\
%  \hline
 \M{mapsto}          & \M{longmapsto}       \\
%  \hline
 \M{hookleftarrow}   & \M{hookrightarrow}   \\
 \M{leftrightarrow}  & \M{Leftrightarrow}     \\
 \M{longleftarrow}   & \M{longrightarrow}   \\
 \M{Longleftarrow}   & \M{Longrightarrow}   \\
 \M{Longleftrightarrow}\\
%  \hline
 \M{leftharpoonup}   & \M{rightharpoonup}   \\
 \M{leftharpoondown} & \M{rightharpoondown} \\
 \M{rightleftharpoons} \\
%  \hline
 \M{nearrow}         & \M{nwarrow}           \\
 \M{searrow}         & \M{swarrow}           \\
  \hline
 \end{tabular}
\end{center}
\end{table}


\begin{table}[htbp]
\begin{center}
\caption{特殊な数学記号}
\begin{tabular}{*{4}{c@{\thickspace\thinspace}l}}
 \hline
 \M{aleph} & \M{partial}  & \M{bot}       & \M{natural}     \\
 \M{hbar}  & \M{infty}    & \M{angle}     & \M{sharp}       \\
 \M{imath} & \M{prime}    & \M{triangle}  & \M{clubsuit}    \\
 \M{jmath} & \M{emptyset} & \M{forall}    & \M{diamondsuit} \\
 \M{ell}   & \M{nabla}    & \M{exists}    & \M{heartsuit}   \\
 \M{wp}    & \M{surd}     & \M{neg}       & \M{spadesuit}   \\
 \M{Re}    & $|$&\verb+|+& \M{backslash} &     &           \\
 \M{Im}    & \M{top}      & \M{flat}      &     &           \\
  \hline
\end{tabular}
\end{center}
\end{table}

%\begin{inout}
%\[ \forall{x}\forall{y}( 
%     P(x,y)\vee(f(x)\wedge g(x))) \] 
%\end{inout}
%
%\begin{inout}
%\( e^{j\theta}=\Re{\{e^{j\theta}\}}
%   +\Im{\{e^{j\theta}\}}
%   =\cos\theta+j\sin\theta\)
%\end{inout}

%\subsection{\Y{latexsym}}

%{\LaTeXe}からはこぼれた記号類を出力するためには,
%\Person{Frank}{Mittelbach}が作成した\Y{latexsym}
%を読み込むと良いでしょう.すでに\Y{amssymb}か
%\Y{amsfonts}を読み込んでいるならば,そちらに定
%義されているので\sty{latexsym}を読み込む必要はありません.

%\begin{table}[htbp]
% \begin{center}
%\caption{\Y{latexsym}パッケージに含まれる数学記号}
%  \begin{tabular}{*{4}{c@{\thickspace\thinspace}l}}
%   \hline
%   \M{mho}     & \M{Join}     & \M{Box}      & \M{Diamond} \\
%   \M{leadsto} & \M{sqsubset} & \M{sqsupset} & \M{lhd} \\
%   \M{unlhd}   & \M{rhd}      & \M{unrhd}    &    & \\
%   \hline
%  \end{tabular}
% \end{center}
%\end{table}



