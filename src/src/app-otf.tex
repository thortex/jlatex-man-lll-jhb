%#!platex -kanji=utf8 hb.tex
%\chapter{記号}

\section{OTFパッケージ}
\begin{usage}
\usepackage[$\<オプション>$]{otf} 
\end{usage}
\Y{OTF}パッケージは\LaTeX で\W{Open Typeフォント}を扱うためのマクロパッ
ケージです.\W{udvips},\W{dvipdfmx},\W{Mxdvi},\W{xdvi}の\W{デバイスド
ライバ} (\W{dviware}) に対応しています.\W{dviout}は制限付きで対応してい
ます.単に\W{ユニコード}文字を使うだけであれば\Y{UTF}パッケージが使えま
す.
%ただし,\sty{UTF}パッケージは開発が終了したものです.
%\sty{OTF}パッケージのオプションとしては以下が用意されています.
\begin{description}
 \item[noreplace] 
	    クラスファイルで元々定義されている\W{TFM}を用います.
	    何も指定しなければTFMが置き換えられます.
 \item[bold]      
	    ゴシック体を太字として割り当てます.
 \item[expert]    
	    組方向に応じた専用仮名を使います.
	    仮名が縦組専用,または横組専用のものに切り替わり,ルビ用の仮
	    名を使えるようになります.
	    \C{rubyfamily} コマンドで使用できます.
 \item[deluxe]    多ウェイト化.
	    明朝体,ゴシック体を2ウェイト化します。(該当フォントが存在
	    する場合)丸ゴシック体も使えるようになります.
 \item[multi]     
	    \W{繁体字}、\W{簡体字}、\W{ハングル}を使えるようにします.
 \item[nomacros]  
	    \Y{ajmacros}を読み込まないようにします.
\end{description}

%\begin{itemize}
% \item \pTeX は\W{JIS~X~0208}までの\W{文字集合}しか扱えません.
% \item OTFは\W{JIS~X~0213}に完全対応しています.
% \item Unicode~3.0の\W{字体}をUTF16コードで指定可能です.
% \item 日本語に限りAdobe-Japan~1-5の\W{CID番号}を指定して出力可能です.
% \item \sty{UTF}パッケージの拡張バージョンであり,\W{Open Typeフォント}
%       が扱えるようになって名称が変更されました.
% \item \Y{OTF}と表記されていますが,実際には \verb|\usepackage{otf}| と
%       原稿に書きます.
% \item \W{CID番号}はAdobe社の開発者向けサイトで公開されている技術資料か
%       ら知る事が出来ます\footnote{\webAdobeJapanList}.
%\end{itemize}

\begin{usage}
\UTF{$\<4桁の16進数>$} % UTF16コードで指定
\CID{$\<10進数>$}% CID 番号で指定
\end{usage}
\W{CID番号}はAdobe社の開発者向けサイトで公開されている技術資料か
ら知る事が出来ます\footnote{\webAdobeJapanList}.

%\begin{itemize}
% \item \C{UTF}\val{4桁の16進数}
% \item \C{CID}\val{10進数}
%\end{itemize}

%\CID{16315}
%\CID{16321}
%\CID{16319}

%\makeatletter
%\@tempcnta=\z@
%\@whilenum\@tempcnta<20317\do{%
%  \CID{\the\@tempcnta}
%  \advance\@tempcnta\@ne
%}
%\makeatother

%\CID{2234}\CID{2233}
%\CID{2244}


\begin{inout}
\UTF{2318}+\keytop{Tab}キーでアプリケーションを選択してから,
\UTF{23ce}キーを押して下さい.
\end{inout}


%HOGE パッケージオプション

\subsection{囲みつき文字 (\protect\ajKuroKaku{5}, \protect\ajMaru{99}) を出力する}

\begin{longtable}{*4l}
  % 最初のヘッダ
  \caption{\sty{OTF}の囲みつき文字}\tablab{OTFの囲みつき文字}\\
  \toprule 
  命令 & 最小値 & 最大値 & 出力例\\
  \midrule
  \endfirsthead
  % 続きのヘッダ
  \toprule 
  命令 & 最小値 & 最大値 & 出力例\\
  \midrule 
  \endhead
  % 続きのフッタ
  \multicolumn{4}{c}{次ページに続きます}\\
  \bottomrule  
  \endfoot
  % 最後のフッタ
  \bottomrule  
  \endlastfoot

 \otfKakomi{ajMaru}{0}{100}
 \otfKakomi[*]{ajMaru}{0}{100}
 \otfKakomi{ajKuroMaru}{0}{100}
 \otfKakomi[*]{ajKuroMaru}{0}{100}
 \otfKakomi{ajKaku}{0}{100}
 \otfKakomi[*]{ajKaku}{0}{100}
 \otfKakomi{ajKuroKaku}{0}{100}
 \otfKakomi[*]{ajKuroKaku}{0}{100}
 \otfKakomi{ajMaruKaku}{0}{100}
 \otfKakomi[*]{ajMaruKaku}{0}{100}
 \otfKakomi{ajKuroMaruKaku}{0}{100}
 \otfKakomi[*]{ajKuroMaruKaku}{0}{100}
 \otfKakomi{ajKakko}{0}{100}
 \otfKakomi[*]{ajKakko}{0}{100}
 \otfKakomi{ajRoman}{1}{15}
 \otfKakomi[*]{ajRoman}{1}{15}
 \otfKakomi{ajroman}{1}{15}
 \otfKakomi{ajPeriod}{1}{9}% AJ1-6?
 \otfKakomi{ajKakkoYobi}{1}{9}
 \otfKakomi{ajKakkoroman}{1}{15}
 \otfKakomi{ajKakkoRoman}{1}{15}
 \otfKakomi{ajKakkoalph}{1}{26}
 \otfKakomi{ajKakkoAlph}{1}{26}
 \otfKakomi{ajKakkoHira}{1}{48}
 \otfKakomi{ajKakkoKata}{1}{48}
 \otfKakomi{ajKakkoKansuji}{1}{20}
 \otfKakomi{ajMaruKansuji}{1}{10}
 \otfKakomi{ajMarualph}{1}{26}
 \otfKakomi{ajMaruAlph}{1}{26}
 \otfKakomi{ajMaruHira}{1}{48}
 \otfKakomi{ajMaruKata}{1}{48}
 \otfKakomi{ajMaruYobi}{1}{7}
 \otfKakomi{ajKuroMarualph}{1}{26}
 \otfKakomi{ajKuroMaruAlph}{1}{26}
 \otfKakomi{ajKuroMaruHira}{1}{48}
 \otfKakomi{ajKuroMaruKata}{1}{48}
 \otfKakomi{ajKuroMaruYobi}{1}{7}
 \otfKakomi{ajKakualph}{1}{26}
 \otfKakomi{ajKakuAlph}{1}{26}
 \otfKakomi{ajKakuHira}{1}{26}
 \otfKakomi{ajKakuKata}{1}{48}
 \otfKakomi{ajKakuYobi}{1}{7}
 \otfKakomi{ajKuroKakualph}{1}{26}
 \otfKakomi{ajKuroKakuAlph}{1}{26}
 \otfKakomi{ajKuroKakuHira}{1}{48}
 \otfKakomi{ajKuroKakuKata}{1}{48}
 \otfKakomi{ajKuroKakuYobi}{1}{7}
 \otfKakomi{ajMaruKakualph}{1}{26}
 \otfKakomi{ajMaruKakuAlph}{1}{26}
 \otfKakomi{ajMaruKakuHira}{1}{48}
 \otfKakomi{ajMaruKakuKata}{1}{48}
 \otfKakomi{ajMaruKakuYobi}{1}{7}
 \otfKakomi{ajKuroMaruKakualph}{1}{26}
 \otfKakomi{ajKuroMaruKakuAlph}{1}{26}
 \otfKakomi{ajKuroMaruKakuHira}{1}{48}
 \otfKakomi{ajKuroMaruKakuKata}{1}{48}
 \otfKakomi{ajKuroMaruKakuYobi}{1}{7}
 \otfKakomi{ajNijuMaru}{1}{10}
  \otfKakomi{ajRecycle}{0}{11}
\end{longtable}

\begin{inout}
\usepackage{otf}
リチウムイオン電池には識別マーク\ajRecycle{0}が表示されていま
すので,使用済み電池はお近くのリサイクル協力店にお持ちください.
\end{inout}

\subsection{合字 (\protect\ajLig{株式会社}, \protect\ajLig{アパート}) を出力する}
\begin{usage}
\ajLig{$\<引数>$}
\end{usage}

\begin{small}
\begin{longtable}{*6l}
  % 最初のヘッダ
  \caption{\sty{OTF}の合字}\tablab{OTFの合字}\\
  \toprule 
  引数 & 横組 & 縦組  & 引数 & 横組 & 縦組\\
  \midrule
  \endfirsthead
  % 続きのヘッダ
  \toprule 
  引数 & 横組 & 縦組  & 引数 & 横組 & 縦組\\
  \midrule 
  \endhead
  % 続きのフッタ
  \multicolumn{6}{c}{次ページに続きます}\\
  \bottomrule  
  \endfoot
  % 最後のフッタ
  \bottomrule  
  \endlastfoot

 \otfLig{明治} \otfLig{大正} \otfLig{昭和} 
 \otfLig{ミリ} \otfLig{キロ} \otfLig{センチ} \otfLig{センチ*} \otfLig{メートル}
 \otfLig{グラム} \otfLig{グラム*} \otfLig{トン} \otfLig{アール}
 \otfLig{アール*} \otfLig{ヘクタール} \otfLig{リットル} \otfLig{ワット}
 \otfLig{ワット*} \otfLig{カロリー} \otfLig{ドル} \otfLig{セント}
 \otfLig{セント*} \otfLig{パーセント} \otfLig{ミリバール} \otfLig{ページ}
 \otfLig{ページ*} \otfLig{キロメートル} \otfLig{キログラム} \otfLig{アパート}
 \otfLig{ビル} \otfLig{マンション} \otfLig{株式会社} \otfLig{有限会社}
 \otfLig{財団法人} \otfLig{平成} \otfLig{フィート} \otfLig{インチ}
 \otfLig{インチ*} \otfLig{ヤード} \otfLig{ヤード*} \otfLig{ヘルツ}
 \otfLig{ヘルツ*} \otfLig{ホーン} \otfLig{ホーン*} \otfLig{コーポ}
 \otfLig{コーポ*} \otfLig{ハイツ} \otfLig{ハイツ*} \otfLig{さじ}
 \otfLig{アト} \otfLig{アルファ} \otfLig{アンペア} \otfLig{イニング}
 \otfLig{ウォン} \otfLig{ウルシ} \otfLig{エーカー} \otfLig{エクサ}
 \otfLig{エスクード} \otfLig{オーム} \otfLig{オングストローム} \otfLig{オンス}
 \otfLig{オントロ} \otfLig{カイリ} \otfLig{カップ} \otfLig{カラット}
 \otfLig{ガロン} \otfLig{ガンマ} \otfLig{ギガ} \otfLig{ギニー}
 \otfLig{キュリー} \otfLig{ギルダー} \otfLig{キロリットル} \otfLig{キロワット}
 \otfLig{グスーム} \otfLig{グラムトン} \otfLig{クルサード} \otfLig{クルゼイロ}
 \otfLig{クローネ} \otfLig{ケース} \otfLig{コルナ} \otfLig{サイクル}
 \otfLig{サンチーム} \otfLig{シリング} \otfLig{ダース} \otfLig{デカ}
 \otfLig{デシ} \otfLig{テラ} \otfLig{ドラクマ} \otfLig{ナノ}
 \otfLig{ノット} \otfLig{バーツ} \otfLig{バーレル} \otfLig{パスカル}
 \otfLig{バレル} \otfLig{ピアストル} \otfLig{ピクル} \otfLig{ピコ}
 \otfLig{ファラッド} \otfLig{ファラド} \otfLig{フェムト} \otfLig{ブッシェル}
 \otfLig{フラン} \otfLig{ベータ} \otfLig{ヘクト} \otfLig{ヘクトパスカル}
 \otfLig{ペセタ} \otfLig{ペソ} \otfLig{ペタ} \otfLig{ペニヒ}
 \otfLig{ペンス} \otfLig{ポイント} \otfLig{ホール} \otfLig{ボルト}
 \otfLig{ホン} \otfLig{ポンド} \otfLig{マイクロ} \otfLig{マイル}
 \otfLig{マッハ} \otfLig{マルク} \otfLig{ミクロン} \otfLig{メガ}
 \otfLig{メガトン} \otfLig{ヤール} \otfLig{ユアン} \otfLig{ユーロ} \otfLig{ラド} \otfLig{リラ} 
 \otfLig{ルーブル} \otfLig{ルクス} \otfLig{ルピア} \otfLig{ルピー} \otfLig{レム} \otfLig{レントゲン} 
 \otfLig{医療法人} \otfLig{学校法人} \otfLig{共同組合} \otfLig{協同組合} \otfLig{合資会社}
 \otfLig{合名会社} \otfLig{社団法人} \otfLig{宗教法人} \otfLig{郵便番号}\\
 \end{longtable}
\end{small}

\begin{table}[htbp]
 \centering
  \caption{\sty{OTF}のくの字などの記号}\tablab{OTFのくの字などの記号類}
 \setcounter{otfligline}{0}
 \setcounter{otfligrow}{2}
 \begin{tabular}{*4l}
  \hline
  引数&合字&引数&合字\\%&引数&合字\\%&引数&合字\\
  \hline
  \otfAjt{Kunoji}
  \otfAjt{KunojiwithBou}
  \otfAjt{DKunoji}
  \otfAjt{DKunojiwithBou}
  \otfAjt{Ninoji}
  \otfAjt{varNinoji}
  \otfAjt{Yusuriten}
  \\
  \hline
 \end{tabular}
\end{table}

\begin{inout}
\usepackage{otf}
宛名書きにおいて「株式会社」を \ajLig{(株)} と表記したり,
(\ajLig{株式会社}) と表記するのは先方に対して失礼になる.
\end{inout}

 \setcounter{otfligline}{0}
 \setcounter{otfligrow}{3}
\begin{longtable}{*6l}
  \caption{\sty{OTF}の仮名文字の合字}\tablab{OTFの仮名文字の合字}\\
  \hline
  引数 & 合字 & 引数 & 合字 & 引数 & 合字\\
  \hline
  \otfLigS{!!} \otfLigS{??} \otfLigS{!?*} 
  \otfLigS{!?} \otfLigS{!*} \otfLigS{?!} \otfLigS{!!*} \\
  \hline\setcounter{otfligline}{0}%
  \otfLigS{う゛} \otfLigS{ワ゛} \otfLigS{ヰ゛} \otfLigS{ヱ゛} \otfLigS{ヲ゛}
  \otfLigS{か゜} \otfLigS{き゜} \otfLigS{く゜} \otfLigS{け゜} \otfLigS{こ゜}
  \otfLigS{カ゜} \otfLigS{キ゜} \otfLigS{ク゜} \otfLigS{ケ゜} \otfLigS{コ゜}
  \otfLigS{セ゜} \otfLigS{ツ゜} \otfLigS{ト゜}
  \hline\setcounter{otfligline}{0}%
  %
  \otfLigS{小か} \otfLigS{小け} \otfLigS{小こ} \otfLigS{小コ}
  \otfLigS{小ク} \otfLigS{小シ} \otfLigS{小ス} \otfLigS{小ト}
  \otfLigS{小ヌ} \otfLigS{小ハ} \otfLigS{小ヒ} \otfLigS{小フ}
  \otfLigS{小ヘ} \otfLigS{小ホ} \otfLigS{小プ} \otfLigS{小ム}
  \otfLigS{小ラ} \otfLigS{小リ} \otfLigS{小ル} \otfLigS{小レ} 
  \otfLigS{小ロ}
  %
  \hline
\end{longtable}

\begin{table}[htbp]
\centering
 \setcounter{otfligline}{0}
 \setcounter{otfligrow}{4}
 \caption{\sty{OTF}の丸文字・括弧文字の合字}\tablab{OTFの丸文字・括弧文字の合字}
 \begin{tabular}{*8l}
  \hline
  引数&合字&引数&合字&引数&合字&引数&合字\\
  \hline
  \otfLigS{○上} \otfLigS{○中} \otfLigS{○下} \otfLigS{○左} 
  \otfLigS{○右} \otfLigS{○〒} \otfLigS{○夜} \otfLigS{○企} 
  \otfLigS{○医} \otfLigS{○協} \otfLigS{○名} \otfLigS{○宗}
  \otfLigS{○労} \otfLigS{○学} \otfLigS{○有} \otfLigS{○株}
  \otfLigS{○社} \otfLigS{○監} \otfLigS{○資} \otfLigS{○財} 
  \otfLigS{○印} \otfLigS{○秘} \otfLigS{○大} \otfLigS{○小}
  \otfLigS{○優} \otfLigS{○控} \otfLigS{○調} \otfLigS{○注}
  \otfLigS{○副} \otfLigS{○減} \otfLigS{○標} \otfLigS{○欠} 
  \otfLigS{○基} \otfLigS{○禁} \otfLigS{○項} \otfLigS{○休} 
  \otfLigS{○女} \otfLigS{○男} \otfLigS{○正} \otfLigS{○写} 
  \otfLigS{○祝} \otfLigS{○出} \otfLigS{○適} \otfLigS{○特} 
  \otfLigS{○済} \otfLigS{○増} \otfLigS{○問} \otfLigS{○答} 
  \otfLigS{○例} \otfLigS{○電}
  \\\setcounter{otfligline}{0}%
  \otfLigS{(株)} \otfLigS{(有)} \otfLigS{(代)} \otfLigS{(至)} 
  \otfLigS{(企)} \otfLigS{(協)} \otfLigS{(名)} \otfLigS{(労)} 
  \otfLigS{(社)} \otfLigS{(監)} \otfLigS{(自)} \otfLigS{(資)} 
  \otfLigS{(財)} \otfLigS{(特)} \otfLigS{(学)} \otfLigS{(祭)} 
  \otfLigS{(呼)} \otfLigS{(祝)} \otfLigS{(休)} \otfLigS{(営)} 
  \otfLigS{(合)} \otfLigS{(注)} \otfLigS{(問)} \otfLigS{(答)} 
  \otfLigS{(例)}
  \\\setcounter{otfligline}{0}%
   \otfLigS{●問} \otfLigS{●答}  \otfLigS{●例} \otfLigS{□印}
   \otfLigS{□問} \otfLigS{□答}  \otfLigS{□例} \otfLigS{□負}
   \otfLigS{■問} \otfLigS{■答}  \otfLigS{■例} \otfLigS{□勝} 
   \otfLigS{◇問} \otfLigS{◇答}  \otfLigS{◇例} 
  \\\setcounter{otfligline}{0}%
   \otfLigS{◆問} \otfLigS{◆答}  \otfLigS{◆例}
  \\
  \hline  
 \end{tabular}
\end{table}

\begin{table}[htbp]
\centering
  \caption{\sty{OTF}の単位の合字}\tablab{OTFの単位の合字}
 \setcounter{otfligline}{0}
 \setcounter{otfligrow}{4}
 \begin{tabular}{*8l}
  \hline
  引数&合字&引数&合字&引数&合字&引数&合字\\
  \hline
  \otfLigS{mm} \otfLigS{cm} \otfLigS{km} \otfLigS{mg}
  \otfLigS{kg} \otfLigS{cc} \otfLigS{m2} \otfLigS{No.}
  \otfLigS{K.K.} \otfLigS{TEL} \otfLigS{cm2} \otfLigS{km2}
  \otfLigS{cm3} \otfLigS{m3} \otfLigS{dl} \otfLigS{l}
  \otfLigS{kl} \otfLigS{ms} \otfLigS{micros} \otfLigS{ns}
  \otfLigS{ps} \otfLigS{KB} \otfLigS{MB} \otfLigS{GB}
  \otfLigS{HP} \otfLigS{Hz} \otfLigS{mb} \otfLigS{ml}
  \otfLigS{KK.} \otfLigS{Tel} \otfLigS{in} \otfLigS{mm2}
  \otfLigS{mm3} \otfLigS{km3} \otfLigS{sec} \otfLigS{min}
  \otfLigS{cal} \otfLigS{kcal} \otfLigS{dB} \otfLigS{m}
  \otfLigS{g} \otfLigS{F} \otfLigS{TB} \otfLigS{FAX}
  \otfLigS{ohm} \otfLigS{AM} \otfLigS{KK} \otfLigS{No}
  \otfLigS{PH} \otfLigS{PM} \otfLigS{PR} \otfLigS{tel}
  \otfLigS{tm} \otfLigS{VS} \otfLigS{a/c} \otfLigS{a.m.}
  \otfLigS{c/c} \otfLigS{c.c.} \otfLigS{c/o} \otfLigS{dl*}
  \otfLigS{hPa} \otfLigS{kl*} \otfLigS{l*} \otfLigS{microg}
  \otfLigS{microm} \otfLigS{ml*} \otfLigS{m/m} \otfLigS{n/m}
  \otfLigS{pH} \otfLigS{p.m.} \otfLigS{mho} %\otfLigS{}
  \\
  \hline
 \end{tabular}
\end{table}

\subsection{濁音・拗音・丸付き/括弧付き文字を簡単に出力する}
\CI{゛,゜,!,○,(}%
\begin{usage}
\゛$\<濁音化したい文字>$    % $\text{例:\゛う}$
\゜$\<半濁音化したい文字>$  % $\text{例:\゜か}$
\!$\<拗音化したい文字>$    % $\text{例:\!ヌ}$
\○$\<丸付きにする文字>$    % $\text{例:\○秘}$
\($\<括弧付きにする文字>$)% $\text{例:\(労)}$
\end{usage}

\begin{usage}
\○$\<引数>$ \●$\<引数>$ \□$\<引数>$ 
\■$\<引数>$ \◇$\<引数>$ \◆$\<引数>$ 
\end{usage}
あ〜ん,ア〜ン,日〜休,半角のa〜z,半角のA〜Zでも合字が出せます.
上の表と同じ出力が得られますが,違う入力方法になります.

%OK: \ajLig{ohm*} \ajLig{mho*} 
%
%OK: \ajLig{euro} \ajLig{euro*} 

%\ajLig{JIS} \ajLig{JAS} 

%NG: \ajLig{No*}

%NG: \ajLig{ppb}  \ajLig{ppm}  \ajLig{'S}  \ajLig{H2} 
% \ajLig{O2}  \ajLig{Ox}  \ajLig{Nx}  \ajLig{Q2} 
% \ajLig{Jr.}  \ajLig{Dr.}  \ajLig{ガル}  \ajLig{グレイ} 
% \ajLig{クローナ}  \ajLig{シーベルト}  \ajLig{シュケル}  \ajLig{ジュール} 
% \ajLig{デシベル}  \ajLig{ドット}  \ajLig{バイト}  \ajLig{ビット} 
% \ajLig{ベクトル}  \ajLig{ボー}  \ajLig{ランド}  \ajLig{リンギット} 
% \ajLig{より} \ajLig{升} \ajLig{コト}  \ajLig{年}  

%NG:  \ajLig{□:A} \ajLig{:B} \ajLig{:C} \ajLig{:D} 
% \ajLig{:E} \ajLig{:F} \ajLig{:終} \ajLig{:CL} 
% \ajLig{:KCL} \ajLig{:BEL} \ajLig{:AS} \ajLig{:AM} 

%\def\@ajmojifam{□}
%\@ajligaturedef{:A}{:B}{:C}{:D}{:E}{:F}{:終}\@nil%AJ1-6
%\@ajligaturedef{:CL}{:KCL}{:BEL}{:AS}{:AM}{:段}{:ゴ}{:ミ}\@nil%AJ1-6

%NG: \ajLig{:段} \ajLig{:ゴ} \ajLig{:ミ} \ajLig{News} \ajLig{PV} 
% \ajLig{MV} \ajLig{SS} \ajLig{S1} \ajLig{S2} \ajLig{S3} 
% \ajLig{HV} 

%OK: \ajPICT{Club} \ajPICT{Heart} \ajPICT{Spade} \ajPICT{Diamond}
% \ajPICT{Club*} \ajPICT{Heart*} \ajPICT{Spade*} \ajPICT{Diamond*}

%OK: \ajPICT{〒} \ajPICT{晴} \ajPICT{曇} \ajPICT{雨}
% \ajPICT{雪} \ajPICT{→} \ajPICT{←} \ajPICT{↑}
% \ajPICT{↓} \ajPICT{野球} \ajPICT{湯} \ajPICT{花}
% \ajPICT{花*} 

%?: \ajPICT{電話} 

%NG: \ajPICT{サッカー} 
%
%?: \ajLig{ほか} \ajLig{▽▽} \ajLig{▽〒}
% \ajLig{△!} \ajLig{■◇} %\ajLig{} \ajLig{}
% \ajLig{} \ajLig{} \ajLig{} \ajLig{}

%\△#1
%\▽#1
% \※花 

%\def\@ajmojifam{◇}
%\@ajligaturedef {News}天再新映声前後終立交{ほか}劇司解株気二多文手{PV}{MV}双{SS}{S1}{S2}{S3}デ{HV}\@nil%AJ1-6

%hoge: \ajHishi{1}
%\ajHishi{2}
%\ajHishi{3}
%\ajHishi{4}

%hoge: %\ajFrac{1}{4}
% \ajFrac{1}{2} \ajFrac{3}{4} \ajFrac{1}{8} \ajFrac{3}{8} 
% \ajFrac{5}{8} \ajFrac{7}{8} \ajFrac{1}{3} \ajFrac{2}{3}
% \ajFrac{}{} \ajFrac{}{} \ajFrac{}{} \ajFrac{}{}
%	\or\ifcase#1\or9826\fi 1
% \ajFrac{1}{1}
% \ajFrac{1}{2} %\ajFrac{2}{2}
% \ajFrac{1}{3} \ajFrac{2}{3} %\ajFrac{3}{3}
% \ajFrac{1}{4} \ajFrac{2}{4} \ajFrac{3}{4} %\ajFrac{4}{4}
% \ajFrac{1}{5} \ajFrac{2}{5} \ajFrac{3}{5} \ajFrac{4}{5} %\ajFrac{5}{5}
% \ajFrac16 \ajFrac26 \ajFrac36 \ajFrac46 \ajFrac56 %\ajFrac66 
% \ajFrac17 \ajFrac27 \ajFrac37 \ajFrac47 \ajFrac57 \ajFrac67 
% \ajFrac18 \ajFrac28 \ajFrac38 \ajFrac48 \ajFrac58 \ajFrac68 \ajFrac78
% \ajFrac19 \ajFrac29 \ajFrac39 \ajFrac49 \ajFrac59 
% \ajFrac69 \ajFrac79 \ajFrac89 
% \ajFrac1{10} \ajFrac2{10} \ajFrac3{10} \ajFrac4{10} \ajFrac5{10} 
% \ajFrac6{10} \ajFrac7{10} \ajFrac8{10} \ajFrac9{10}
% \ajFrac1{11} \ajFrac2{11} \ajFrac3{11} \ajFrac4{11} \ajFrac5{11}
% \ajFrac6{11} \ajFrac7{11} \ajFrac8{11} \ajFrac9{11} \ajFrac{10}{11}
% \ajFrac1{12} \ajFrac2{12} \ajFrac3{12} \ajFrac4{12} \ajFrac5{12} 
% \ajFrac6{12} \ajFrac7{12} \ajFrac8{12} \ajFrac9{12} \ajFrac{10}{12} 
% \ajFrac{11}{12}

%hoge: \ajFrac{1}{100} 

%hoge: \ajArrow{RIGHT*} \ajArrow {LEFT*} \ajArrow{UP*} \ajArrow{DOWN*}

%hoge: 
%\ajArrow{DOWN} \ajArrow{UP} \ajArrow{LEFT} \ajArrow{RIGHT}

%hoge: \ajArrow{RightHand} \ajArrow{LeftHand} \ajArrow{UpHand} 
%\ajArrow{DownHand}

%hoge: 
%\ajArrow{Left/Right} \ajArrow{Right/Left} \ajArrow{Up/Down} 
%\ajArrow{Down/Up}

%hoge: 
%\ajArrow{LeftScissors} \ajArrow{RightScissors} \ajArrow{UpScissors}
%\ajArrow{DownScissors}

%hoge: 
%\ajArrow{LeftTriangle} \ajArrow{RightTriangle} 
%\ajArrow{LeftTriangle*} \ajArrow{RightTriangle*} 
%\ajArrow{LeftAngle} \ajArrow{RightAngle}

%\ajArrow{Left} \ajArrow{Right} \ajArrow{Up} \ajArrow{Down} 
%\ajArrow{LeftDouble}
%\ajArrow{RightDouble}

%\ajArrow{LeftRight*} 
%\ajArrow{LeftRightDouble}
%%\ajArrow{RightDown} \ajArrow{LeftDown}
%%\ajArrow{LeftUp} \ajArrow{RightUp} 
%\ajArrow{Right/Left*}
%\ajArrow{Left/Right*} \ajArrow{Right/Left+} \ajArrow{Down/Up+}
%\ajArrow{Left+} \ajArrow{Right+} \ajArrow{Up+} \ajArrow{Down+}
%\ajArrow{LeftRight+} \ajArrow{UpDown+}

%hoge:  \ajArrow{UpAngle}
%\ajArrow{DownAngle} \ajArrow{LeftAngle*} \ajArrow{RightAngle*} 
%\ajArrow{UpAngle*} \ajArrow{DownAngle*}
%
%\ajArrow{RightUp*}
%\ajArrow{RightDown*}
%
%良く使われる異字体: 
%\ajMayuHama, \ajTatsuSaki, \ajHashigoTaka, \ajTsuchiYoshi

\begin{table}[htbp]
 \centering
 \setcounter{otfligline}{0}
 \setcounter{otfligrow}{2}
  \caption{\sty{OTF}のその他の記号類}\tablab{OTFのその他の記号類}
 \begin{tabular}{*4l}
  \hline
  引数&合字&引数&合字\\%&引数&合字\\%&引数&合字\\
  \hline
 \otfAj{SenteMark}
 \otfAj{GoteMark}
 \otfAj{Club}
 \otfAj{Heart}
 \otfAj{Spade}
 \otfAj{Diamond}
 \otfAj{varClub}
 \otfAj{varHeart}
 \otfAj{varSpade}
 \otfAj{varDiamond}
 \otfAj{Phone}
 \otfAj{Postal}
 \otfAj{varPostal}
 \otfAj{Sun}
 \otfAj{Cloud}
 \otfAj{Umbrella}
 \otfAj{Snowman}
 \otfAj{JIS}
 \otfAj{JAS}
 \otfAj{Ball}
 \otfAj{HotSpring}
 \otfAj{WhiteSesame}
 \otfAj{BlackSesame}
 \otfAj{WhiteFlorette}
 \otfAj{BlackFlorette}
 \otfAj{RightBArrow}
 \otfAj{LeftBArrow}
 \otfAj{UpBArrow}
 \otfAj{DownBArrow}
 \otfAj{RightHand}
 \otfAj{LeftHand}
 \otfAj{UpHand}
 \otfAj{DownHand}
 \otfAj{RightScissors}
 \otfAj{LeftScissors}
 \otfAj{UpScissors}
 \otfAj{DownScissors}
 \otfAj{RightWArrow}
 \otfAj{LeftWArrow}
 \otfAj{UpWArrow}
 \otfAj{DownWArrow}
 \otfAj{RightDownArrow}
 \otfAj{LeftDownArrow}
 \otfAj{LeftUpArrow}
 \otfAj{RightUpArrow}
  \\ \setcounter{otfligline}{0}
 \otfAj{Masu}
 \otfAj{Yori}
 \otfAj{Koto}
 \otfAj{Uta}
 \otfAj{CommandKey}
 \otfAj{ReturnKey}
 \otfAj{Checkmark}
 \otfAj{VisibleSpace}
 \hline
 \end{tabular}
\end{table}


