
\section{番号無しの数式を記述する}

\subsection{文中に数式を記述する}

\glossary{"(@\hspace*{-1.2ex}\verb+\(+}%"}
\glossary{")@\hspace*{-1.2ex}\verb+\)+}%"}
\index{"$@\verb+$+}%"}
\glossary{"$@\verb+$+}%"}
\begin{usage}
$\text{\textdollar}$ $\<数式>$ $\text{\textdollar}$
\( $\<数式>$ \)
\begin{math} $\<数式>$ \end{math}
\end{usage}

\index{=@\str{=}!等号としての\zdash}%
\begin{inout}
$a$ の2乗と $b$ の2乗を足したものが $c$ の2乗に等しいという
事は \( a^2 + b^2 = c^2 \) と表せるが,
\begin{math}  
 a^2 + b^2 = c^2
\end{math} 
と記述する事も可能である.
\end{inout}


\subsection{別行の数式を記述する}

\glossary{[@\hspace*{-1.2ex}\verb+\[+}%
\glossary{]@\hspace*{-1.2ex}\verb+\]+}%
\index{$$@\verb+$$+}%
\index{別行立て数式}
\begin{usage}
$\text{\textdollar\textdollar}$ $\<数式>$ $\text{\textdollar\textdollar}$
\[ $\<数式>$ \]
\begin{displaymath} $\<数式>$ \end{displaymath}
\end{usage}

全て別行立ての数式で,標準的には中央揃えになります.

\begin{inout}
別行立て数式は 
\[ 
  c^2 = a^2 + b^2 
\] 
のように自動的に中央揃えになります.別行立て数式は
\begin{displaymath}
     a^2 + b^2 = c^2
\end{displaymath}
と書く事もできます.
\end{inout}


\subsection{複数行で整列させる数式を記述する}
\index{=@\str{=}!eqnarrayかんきょうちゅの@\texttt{eqnarray}環境中の\zdash}%
\begin{usage}
\begin{eqnarray*}
$\<左辺>$ & $\<関係子>$ & $\<右辺>$ \\
$\<左辺>$ & $\<関係子>$ & $\<右辺>$
\end{eqnarray*}
\end{usage}

流れのある複数行の数式や証明などで関係子(例えば等号`$=$')の位置
を揃えるときは\Env{eqnarray*}環境を使用します.

1行には\Z{アンパサンド}%
\index{"&@\verb+&+!eqnarray*@\texttt{eqnarray*}環境の\zdash}%}
\glossary{"&@\verb+&+!eqnarray*@\texttt{eqnarray*}環境の\zdash}%}
\index{改行!eqnarrayかんきょうでの@\texttt{eqnarray}環境での\zdash}%}
\qu{\str{&}}を二つまで,行の終わりには改行\qu{\texttt{\BS\BS}}%}}}%}$"}
を書きます.ただし最終行には改行を入れません.
また各列における成分は省略する事が可能です.

\begin{inout}
\begin{eqnarray*}
f(x)      & = & x^2  \\
f'(x)     & = & 2x   
\end{eqnarray*}
\end{inout}

%\begin{inout}
%\begin{eqnarray*}
% 5c = 4g \\
% 6a + 3c + 7d = 2g \\
% 3a + 5c = 3g
%\end{eqnarray*}
%\end{inout}
%上記の例はアンパサンド\qu{\string&}による区切りがありませんが,
%おおむね以下の出力例と同じような結果に落ち着いています.

%\begin{inout}
%\begin{eqnarray*}
% 5c           &=& 4g \\
% 6a + 3c + 7d &=& 2g \\
% 3a + 5c      &=& 3g
%\end{eqnarray*}
%\end{inout}

%\begin{inout}
%\begin{eqnarray*}
%&& 5c = 4g \\
%&& 6a + 3c + 7d = 2g \\
%&& 3a + 5c = 3g
%\end{eqnarray*}
%\end{inout}

%\begin{inout}
%\begin{eqnarray*}
%& 5c &= 4g \\
%& 6a + 3c + 7d &= 2g \\
%& 3a + 5c &= 3g
%\end{eqnarray*}
%\end{inout}

%\begin{inout}
%\begin{eqnarray*}
%& 5c = 4g &\\
%& 6a + 3c + 7d = 2g &\\
%& 3a + 5c = 3g&
%\end{eqnarray*}
%\end{inout}

%\begin{inout}
%\begin{eqnarray*}
% 5c&& = 4g \\
% 6a + 3c + 7d &&= 2g \\
% 3a + 5c &&= 3g
%\end{eqnarray*}
%\end{inout}

%\begin{inout}
%\begin{eqnarray*}
%& 5c = 4g \\
%& 6a + 3c + 7d = 2g \\
%& 3a + 5c = 3g
%\end{eqnarray*}
%\end{inout}

%\begin{inout}
%\begin{eqnarray*}
% 5c           &= 4g &\\
% 6a + 3c + 7d &= 2g &\\
% 3a + 5c      &= 3g &
%\end{eqnarray*}
%\end{inout}

\section{番号付きの数式}
\indindz{数式}{番号付きの}%
\indindz{番号}{数式の}%


\subsection{1行だけの番号付き数式を記述する}
数式が1行の場合は
\Env{equation}環境で出力する事ができます.
\begin{usage}
\begin{equation}
$\<数式>$ \label{$\<ラベル>$}
\end{equation} 
\end{usage}

\begin{inout}
\begin{equation}
 a^2 + b^2 = c^2  \label{eq:equ}
\end{equation}
式~(\ref{eq:equ}) より $c^2$ は 
$a^2+b^2$ に等しい.
\end{inout}

\subsection{複数業の番号付き数式を記述する}
\indindz{数式}{複数行の番号付き}%
\indindz{番号}{複数行の数式の}%
\CI{nonumber}%
\begin{usage}
\begin{eqnarray}
$\<左辺>$ & $\<関係子>$ & $\<右辺>$ \label{$\<ラベル>$}\\
$\<左辺>$ & $\<関係子>$ & $\<右辺>$ \nonumber\\% 番号を振らない時
$\<左辺>$ & $\<関係子>$ & $\<右辺>$
\end{eqnarray} 
\end{usage}

\begin{inout}
\begin{eqnarray}
f(x)       &=& x^2 \label{eq1}\\
f'(x)      &=& 2x  \label{eq2}\\
\int f(x)dx&=& x^3/3+C\nonumber
\end{eqnarray}
式~(\ref{eq1}) を微分したものが
式~(\ref{eq2}) である.
\end{inout}


\section{グルーピング}
\indindz{波括弧}{数式モード中の}%
\indindz{括弧}{数式モード中の}%
\begin{usage}
{$\<要素>$} 
\end{usage}

\TeX/\LaTeX において何らかの \val{要素} をグループにしたければ波括弧で
括ります.

変数$a$の$x+y$乗を出力するために{\LaTeX}では
一塊の要素を\W{波括弧}で{\W{グルーピング}}
します.ここではべき乗を例にとって見てみましょう.
\begin{inout}
\( a^x+y \neq a^{x+y} \)
\end{inout}
グルーピングによって数式の要素を一つのグループにし
ます.数式環境に限りませんが{\LaTeX}では一つにした
い要素をグループとして扱い,波括弧でグループ化を
行います.


\section{数式中の書体を変更する}
\zindind{数式}{の書体の変更}%

数式では書体の変更が必要になると思います.
例えば行列を表すものはボールド体に変更し数式中で
文字を表示するときがあるでしょう.そのようなときは
書体変更用のコマンドを使います.数式中では
通常のテキストモードで使う書体変更コマンドは
使えませんので,数式の書体変更用のコマンドを
使います.数式中でしか使用できない書体用コマ
ンドは\tabref{mathfont}の通りです.

\begin{table}[htbp]
\begin{center}
\caption{数式モードにおける書体の変更}\tablab{mathfont}
\begin{tabular}{lll}
\TR
\Th{書体}          & \Th{命令} & \Th{出力} \\
\MR
標準の書体    & \C{mathnormal} & $\mathnormal{ABCabc}$ \\
ローマン体    & \C{mathrm}     & $\mathrm{ABCabc}$ \\
サンセリフ体  & \C{mathsf}     & $\mathsf{ABCabc}$ \\
タイプライタ体& \C{mathtt}     & $\mathtt{ABCabc}$ \\
ボールド体    & \C{mathbf}     & $\mathbf{ABCabc}$ \\
イタリック体  & \C{mathit}     & $\mathit{ABCabc}$ \\
カリグラフィック体& \C{mathcal}& $\mathcal{ABC}$\\
\BR
\end{tabular}
\end{center}
\end{table}

\begin{inout}
\begin{displaymath}
\int f(x) dx \neq 
  \int f(x) \mathrm{d}x
\end{displaymath}
\end{inout} 

\subsection{単位を記述する}\seclab{unit}

\begin{table}[htbp]
 \begin{center}
  \caption{ほげ}
 \begin{tabular}{llll}
  \hline
  ローマン体  & &イタリック体 & \\
  \hline
 $\mathrm{A}$ & アンペア & $A$ & 原子番号 (変数)\\
  e 電子(粒子名) & e 電子電荷(定数)\\
  g グルーオン, g 重力定数\\
   l リットル & l 長さ\\
  m メーター& m 質量\\
  t トン & t 時間\\
  V ボルト & V 体積\\
  \hline
 \end{tabular}
 \end{center}
\end{table}

文中でも数式中でも単位は基本的にはローマン体で
出力するのが普通ですので
\begin{inout}
それは$y=30cm$だから合計300mmだ.
\end{inout}
のような入力はおかしい訳です.この場合は
\begin{inout}
それは$y=30\,\mathrm{cm}$だから
合計300\,mmだ.
\end{inout}
としたほうが良いでしょう\footnote{場合によっては`$y=30\,[\mathrm{cm}]$'
とする事もあると思います.}.
このように単位は数式中でも使う
事があるかもしれませんので \C{ensuremath}で
単位用の命令を作成します.例えば長さの単位である`mm'は
次のように定義します.

\subsection{単位の使い方}
\indindz{記号}{単位}%
\Z{単位}は基本的に\Z{国際単位}\Z{SI}に従いローマン体,
記号はイタリック体で表記します.\zindind{単位}{の接頭語}%
単位の\Z{接頭語}として\tabref{SI:prefix}の\Z{修飾子}が
使用できます\footnote{この他にも$10^{24}$から$10^{-24}$まで
 (Y Z P T G M k m \textmu\ n p f a z y) ありますが,頻繁に用いられるだろ
 う修飾子だけを掲載しました.}.

\begin{table}[htbp]
 \begin{center}
  \caption{SIの基本単位}\label{tab:SI:base}
 \begin{tabular}{llllll}
 \TR
 \Th{名称} & \Th{英語名称} & \Th{記号} & \Th{単位} & \Th{読み} & \Th{英語読み}\\
 \MR
 \Z{長さ}  & \Z{length}& $l$  & m    & \Z{メートル}  & \Z{meter}\\
 \Z{質量}  & \Z{mass}  & $m$  & kg   & \Z{キログラム}& \Z{kilogram}\\
 \Z{時間}  & \Z{time}  & $t$  & s    & \Z{秒}        & \Z{second}\\
 \Z{物理量}& \Z{amount of substance} & $n$ & mol  & \Z{モル}      &\Z{mole}\\
 \Z{電流}  & \Z{electric current}    & $I$ & A    & \Z{アンペア}  &\Z{ampere}\\
 \Z{熱力学温度}& {thermodynamic} 
      &$T$&K & \Z{ケルビン}  &\Z{kelvin}\\
          &  temperature & & & \\
 \Z{光度} & \Z{luminous intensity}  & $I$ & cd   & 
      \Z{カンデラ}  &\Z{candela}\\
 \BR
 \end{tabular}
 \end{center}
\end{table}
%
\begin{table}[htbp]
\newcommand*\DP[1]{$10^{#1}$}
 \begin{center}
  \caption{$10^n$の修飾子}\label{tab:SI:prefix}
  \begin{tabular}{*{15}{l}}
\TR
$10^n$&\DP{12}&\DP{9}&\DP{6}&\DP{3}&\DP{-3}&\DP{-6}&\DP{-9}&\DP{-12}\\
\MR
記号  & T & G & M & k & m & \textmu & n & p \\
名称  & \Z{テラ} & \Z{ギガ} & \Z{メガ} & \Z{キロ} & \Z{ミリ} & \Z{マイクロ}{*} & \Z{ナノ} & \Z{ピコ} \\
英語名称& \Z{tera} & \Z{giga} & \Z{mega} & \Z{kilo} & \Z{milli} & \Z{micro} & \Z{nano} & \Z{pico} \\
\BR
  \end{tabular}
\par\vskip1ex
\begin{footnotesize}
{*} ローマン体のマイクロ (\textmu) を出力するには
\Y{textcomp}パッケージの \cmd{textmu} コマンドを使います.
\end{footnotesize}
 \end{center}
\end{table}
%
\begin{inout}
数値と単位の間には半角程度の空白を挿入
します.3\,mkg(3ミリキログラム)など,
修飾子を複数表記してはいけません.
3\,mkg (×) は正しくは3\,gとなります.
\end{inout}
\indindz{空白}{数値と単位の}%
数値と単位の間には{半角程度の空白を挿入します}.
単位とその修飾子は{いかなる場合でもローマン体とします}.
\zindind{強調}{の中の単位}%
強調部分に単位が含まれる場合でも同様です.


\subsection{フラクトゥール・黒板風書体}
行列を表現するのに\W{ブラックボードボールド体}(\W{黒板風書体})
を使う事があるそうです.これは文字が白抜きになり
ボールド体よりも行列である事が分かりやすくなってい
ます.これを使うには\Y{amssymb}を読み込みます.
数式中で通常のテキストを使いたいときは
\Y{amsmath}パッケージを読み込み \C{text}命令を
使います(\tabref{blackboldfamily}).

\begin{table}[htbp]
\begin{center}\zindind{数式}{の中の文章}%
\caption{\textsf{amssymb}による数式書体の拡張}\tablab{blackboldfamily}
\begin{tabular}{lll}
\TR
\Th{書体} & \Th{命令} & \Th{出力} \\
\MR
フラクトゥール体         &
 \C{mathfrak} & $\mathfrak{ABCabc}$\\
ブラックボードボールド体 & 
 \C{mathbb}   & $\mathbb{ABC}$  \\
数式内テキスト  & 
 \C{text}   & $\text{ABC数式です}$    \\
\BR
\end{tabular}
\end{center}
\end{table}

\begin{inout}
\usepackage{amssymb}
$$ x \in \mathbf{R} \neq 
  x \in \mathbb{R}$$
$$ f(x) = 1/(1 + g(x)), (x = 3 
  \text{とする})$$
\end{inout}


\begin{inout}
\begin{displaymath}
 \bigotimes^n x \stackrel{\mathrm
 {def}}{=} \overbrace{x \otimes (x
 \otimes (\cdots \otimes x) \cdots)}
   ^{\text{$n$ copies of $x$}}
\end{displaymath}
\end{inout}

\section{数式中の空きを調整する}

\begin{itemize}
 \item 空白や改行は常に一つのスペースとして解釈され,そのスペースが反映
       される事はなく,全て\TeX 側で自動的に空きが調整される.
 \item テキストモードでは空行が改段落として扱われていたが,
       数式中ではなんの意味も持たない(見た目は変わる).
 \item 半角英字は全て指示がない限り数式イタリック体に変更され,自動的に
       空白が調整される.
 \item
\end{itemize}

\zindind{数式}{中の空白の調節}%
\zindind{空白}{の調整}%
数式モードでは入力した半角空白が反映されません.{\LaTeX}は
数式モードでは自動的に隣り合う\W{数式要素}(\W{アトム})から挿入すべき空白を決
めています.ですがユーザが空白を調節したほうが正しい表記になると
きがあります.ユーザ側で空白を調節するため
\tabref{mathspaces}のコマンドを使います.
積分`$\int$'や全微分`$dx$'のあいだには
ユーザが空白を入れると意味的に正しくなります.

\begin{table}[htbp]
\begin{center}
\caption{数式における空白の制御}\tablab{mathspaces}
\glossary{" @\hspace*{-1.2ex}\verb*+"\" +}%"
\glossary{" @\hspace*{-1.2ex}\verb+"\"!+}%"
\begin{tabular}{*4l}
\TR
\Th{空白の大きさ} & \Th{命令} & \Th{入力例} & \Th{出力例}\\
\MR
空白なし         & \verb*| |  & \verb*|dx dy|  
   & $dx dy$ \\
かなり小さい空白 & \C{,}    & \verb|dx\, dy| 
   & $dx\, dy$ \\
小さい空白       & \C{:}    & \verb|dx\: dy| 
   & $dx\: dy$ \\
少し小さい空白   & \C{;}    & \verb|dx\; dy| 
   & $dx\; dy$ \\
半角の空白       & \verb*+\ + & \verb*|dx\ dy|  % 適切な単語間空白
   & $dx\ dy$ \\
全角の空白 (1\,em)      & \C{quad} & \verb|dx\quad dy|
   & $dx\quad dy$ \\
全角の2倍の空白 (2\,em)  & \C{qquad}& \verb|dx\qquad dy| 
   & $dx\qquad dy$ \\
負の小さい空白   & \cmd{!}     & \verb|dx\!dy|
   & $dx\! dy$ \\
\BR
\end{tabular}
\end{center}
\end{table}

積分`$\int$'や全微分`$dx$'のあいだには
ユーザが空白を入れると見映えがします.
\begin{inout}
\[ \int\int f(x)dxdy \neq 
   \int\!\!\!\int f(x)\ dx\ dy  \]
\end{inout}

\begin{inout}
\[ g = \frac{1}{5}a + \frac{1}{6}b 
     \quad (a>3,\ b>5) \]
\end{inout}

\begin{inout}
\[ \Gamma(x) = \int^\infty_0 e^{-t} t^{x-1} dt \qquad (x > 0) \]
\end{inout}

\subsection{列挙した文中の数式の空き}

複数の式を1行に列挙する場合,次のようにすると
適切な空白が挿入されません.
\begin{inout}
特性数 $c, v, e$ を考えるとき,
\[c = f_1 + f_2, v = 3f_2 + f_3, e = 2f_1 + 3f_3 \]
\end{inout}


`\verb|$c, v, e$|'の部分には適切な単語間空白
を挿入するのが望ましいでしょうし, \C{quad} 程度の空きをそれぞれの
式のあいだに挿入するのが適切だと思われます.
\begin{inout}
特性数 $c$, $v$, $e$ を考えるとき,
\[c = f_1 + f_2,\quad v = 3f_2
 + f_3, \quad e = 2f_1 + 3f_3 \]
\end{inout}

\section{表示形式の調整}

数式を記述する各環境において自動的に各要素
の大きさが決められます.文中数式での分数は
\( \frac{a}{b} \)という出力になりますが,
これでは少し小さいので$\displaystyle \frac{a}{b}$
としたいときがあると思います.そのようなときは
ユーザが表示形式を変更するには\tabref{math:display}の
命令が使えます.
\begin{table}[htpb]
\begin{center}
\caption{数式の表示形式の変更}\tablab{math:display}
%あぁ、D, T, S, SS, とか圧縮スタイルについての
%説明も入れるべきだろう.
\begin{tabular}{lll}
\TR
\Th{命令} & \Th{出力形式} & \Th{例}{$\left(\frac{a}{b}\right)$}\\
\MR
\rule{0pt}{1.5zw}\C{displaystyle} & 別行立て形式& 
 $\displaystyle \frac{a}{b}$\\[5pt]
\C{textstyle}         & 文中数式形式           & 
$\textstyle \frac{a}{b}$\\
\C{scriptstyle}       & 添え字形式             & 
$\scriptstyle \frac{a}{b}$\\
\C{scriptscriptstyle} & 添え字の中の添え字形式 & 
$\scriptscriptstyle \frac{a}{b}$\\
\BR
\end{tabular}
\end{center}
\end{table}

あまり多用すると段落のあいだが空きすぎて逆に見栄
えが悪くなるのである程度長い数式を文中に入れて
いるときは別行立てにするのが良い方法です.
\index{"/@"\verb+"/+!分数の\zdash}%"
また\zindind{分数}{の書き方}%
文中の数式に限りませんが,分数は$\frac{a}{b}$
と書くよりも$a/b$とするほうがスマートで見やすいので
スラッシュによる表記にしたほうが良いでしょう.%

\index{連分数}\indindz{分数}{連}%%

\begin{inout}
\(f(x)\) の不定積分 \(\int f(x)dx\)
と\(\displaystyle \int f(x)dx\) は
\LaTeX では少し違うし,分数は 
$\frac{a}{b}$ と書くよりも $a/b$ 
と書くほうがスマートである.
\end{inout}

\begin{inout}
\[ 
 \frac{1}{1+\frac{1}{1+\frac{1}{1+x}}}
 \neq \frac{1}{\displaystyle 1+
 \frac{1}{\displaystyle 1+
 \frac{1}{1+x}}} \]
\end{inout}
%
\begin{inout}
\( \int^b_a f(x)dx \neq 
   {\displaystyle\int^b_a g(x)dx}
\) 
\end{inout}

\begin{inout}
\(\int^\beta_\alpha f(x)\,dx \neq 
{\displaystyle\int^\beta_\alpha f(x)\,dx}\)
\end{inout}

\subsection{高さや幅を揃える}
\indindz{記号}{ルート}%
\indindz{記号}{根号}%
\indindz{高さ}{ルートの}%

ルート記号などを使っているとルートの高さが揃わずに
見栄えが悪くなるときがあります.これには数式中で
ルートなどの高さを揃える \C{mathstrut}命令が使えます.
\begin{inout}
\[ \overline{\sqrt a + \sqrt b 
   \neq \sqrt{\mathstrut a}+
   \sqrt{\mathstrut b}} \]
\end{inout}
分かりづらいのですが実は高さのみならず,深さも \cmd{mathstrut}
によって自動的に調整されています.

もう少し高度な命令として \C{phantom},
\C{vphantom},\C{hphantom}の三つが用意されています.
\C{phantom}命令は引数に与えられた要素だけの
高さと幅と深さを持った空白を作成します.\cmd{vhpantom}
は引数に与えた要素の高さと同じ目には見えない箱を
作成します.\cmd{hphantom}はその横方向バージョンです.
\begin{inout}
\[ \sqrt{\int f(x)dx} + \sqrt{g} \neq 
   \sqrt{\int f(x)dx} + \sqrt 
   {\vphantom{\int f(x)dx} g} \]
\end{inout}
もう一つ \C{smash}という命令もあり,これは
引数に与えられた要素の高さと深さを0にする
魔法のようなものです.\cmd{smash} と \cmd{vphantom}
を組み合わせると要素の幅はそのままで高さと深さを0に
したうえで \cmd{vphantom}で指定した高さと深さの
見えない箱を作成できるので,{高さや深さを揃えるのに}
使えます.\indindz{幅}{要素の}%
\begin{inout}
\[ \underbrace{a + b} + \underbrace{i + j}
 \neq \underbrace{\smash{a + b}
 \vphantom{i + j}} + \underbrace{i + j}\]
\end{inout}

高さ,幅,深さを擬似的に模倣する \C{phantom} 命令に
よって,次のような整列が可能となります.
\begin{inout}
\newcommand\PN[1]{\phantom{\mbox{}#1}}
\begin{eqnarray*}
a_{11}x_1  +a_{12}x_1 \PN{+a_{23}x_3}
  &=& b_1\\
a_{21}x_1  +a_{22}x_2 + a_{23}x_3
  &=& b_2\\
a_{31}x_1 \PN{+a_{22}x_2}+ a_{33}x_3
  &=& b_3
\end{eqnarray*}
\end{inout} 
ただし,プラス `\str{+}'の前に何らかの要素をおかないと,
適切な空白が挿入されないため,\C{mbox} 命令を補っています.




















\section{記号以外の数式コマンド}

ギリシャ文字や関係子,二項演算子類の記号は巻末に全てまとめています.

\subsection{添字を記述する}
\index{"^@\verb+^+}%
\index{"_@\verb+_+}%
\glossary{"^@\verb+^+}%
\glossary{"_@\verb+_+}%
\indindz{添え字}{上付きの}\indindz{添え字}{下付きの}%

\begin{usage}
$\<数式要素>$^$\<上付き文字>$ % 上付きのとき
$\<数式要素>$_$\<下付き文字>$ % 下付きのとき
$\<数式要素>$^$\<上付き文字>$_$\<下付き文字>$ % 両方のとき
\end{usage}

上付き・下付きにする文字が一つではないとき,波括弧でグルーピングします.

\begin{table}[htbp]
\begin{center}
\caption{添え字の使い方の例}\tablab{soeji}
\begin{tabular}{lll|lll}
\TR
\Th{意味} & \Th{命令} & \Th{出力} & \Th{意味} & \Th{命令} & \Th{出力}\\
\MR
右上       & \verb|x^{a+b}|      & $x^{a+b}$       & 
左上       & \verb|{}^{a+b}x|    & ${}^{a+b}x$     \\
右下       & \verb|x_{a+b}|      & $x_{a+b}$       & 
左下       & \verb|{}_{a+b}x|    & ${}_{a+b}x$     \\
右上と右下 & \verb|x^{a+b}_{c+d}|& $x^{a+b}_{c+d}$ & 
左上と左下 & \verb|{}^{a}_{b}x|  & ${}^{a}_{b}x$   \\
右上の右上 & \verb|x^{a^{b}}|    & $x^{a^{b}}$     &
左下と右下& \verb|{}_{a}x_{b}|  & ${}_{a}x_{b}$   \\%$
\BR
\end{tabular}
\end{center}
\end{table}

\begin{inout}
\( A^4_3 \neq A\sp4\sb3 \)
\end{inout}

\subsection{数式要素の左側に添字を付ける}
\index{置換行列}%
\index{左側の添字}%

\begin{inout}
\( {}^{a+b}_{x+y}A^{a+b}_{x+y} \)
\end{inout}

以上のような方法では左側に添え字を付けるときに
うまくいかない場合がありますので,\Person{Harald}{Harders}
による\Y{leftidx}パッケージを使います.
%\begin{usage}
%\usepackage{leftidx}
%\leftidx{$\<左側添え字>$}{$\<数式要素>$}{$\<右側添え>$}
%\ltrans{$\<数式要素>$} % 置換行列用
%\end{usage}

%\begin{inout}
%\begin{eqnarray*}
%{}_a^b\left(\frac{x}{y}\right)_c^d 
%  &\neq& \leftidx{_a^b}{\left(
%    \frac{x}{y} \right)}{_c^d}\\
%{}^\mathrm{t} A &\neq& \ltrans{A}
%\end{eqnarray*}
%\end{inout}

\subsection{法(剰余)を表す関数を記述する}
\index{剰余}
また \C{bmod} のように{\W{法}}を表すための命令もあります.
\begin{usage}
\bmod{$\<数式要素>$}% 二項演算子として
\pmod{$\<数式要素>$}
\end{usage}

\begin{inout}
\( \mathrm M\bmod{\mathrm N} \neq 
   \mathrm M\pmod{\mathrm N} \)
\end{inout}

\subsection{分数を記述する}
\index{分数}
\begin{usage}
\frac{$\<分子>$}{$\<分母>$}
\end{usage}

\begin{inout}
\begin{displaymath}
\frac{1}{g(x)} + \frac{1}{2x^3 + 5x^2 + 8x + 5}
\end{displaymath} 
\end{inout}

\subsection{線のない分数}

\begin{inout}
\begin{displaymath}
\frac{a+b}{x+y} \neq \binom{a+b}{x+y}
\end{displaymath} 
\end{inout}


\subsection{根号}
\index{根号}
\begin{usage}
\sqrt[$\<根>$]{$\<数式要素>$}
\end{usage}

\begin{inout}
\begin{displaymath}
g(x) = \sqrt{\frac{1}{f(x)}}
\end{displaymath}
\end{inout}

\subsection{大きさ可変の数学記号(積分/直和/直積)を記述する}
\index{積分記号}%
\index{直和記号}%
\index{直積記号}%

\begin{inout}
\begin{displaymath}
\sqrt{\frac{1}{g(x)} + \sqrt{\int f(x) dx}}
\end{displaymath} 
\end{inout}


\subsection{大きさ可変の数学記号の添字の位置を調整する}
\index{添字の位置}%
\CI{limits}%
\CI{nolimits}%
\begin{usage}
$\<演算子>$\limits   % 上下に付ける
$\<演算子>$\nolimits % 肩に付ける
\end{usage}

\begin{inout}
\begin{eqnarray*}
\sum\nolimits^n_{k=0}k & \neq & \sum^n_{k=0}k\\
\int^b_a dx            & \neq & \int\limits^b_a dx
\end{eqnarray*}
\end{inout}

\begin{inout}
\begin{eqnarray*}
\lim\nolimits_{n\to 0} n & \neq & \lim_{n\to 0} n\\
\prod^n_{i=1}n & \neq &    \prod\nolimits^n_{i=1} n
\end{eqnarray*}
\end{inout}
 
\subsection{括弧などの区切り記号で数式を括る}

\begin{itemize}
 \item \C{left}と \C{right}命令を使って大きさを変える.
 \item 区切り記号の大きさを指定する.
\end{itemize}

\begin{inout}
\[ \left[ \Big(x+y\Big) \right] \]
\end{inout}

\begin{inout}
\[ \left( \frac{1}{1+\frac{1}{1+x}} \right)  \]
\end{inout}

\subsection{自分で大きさを指定して区切り記号を記述する}
%\index{"/@"\verb+"/+}
\index{"/@"\verb+"/+}%
\index{"/@"\verb+"/+!区切り記号の\zdash}%

\begin{inout}
\begin{displaymath}
\Biggl\| \Biggl(\int f(x) dx\Biggr) 
  \Bigg/ \Biggl(\int g(x) dx\Biggr) 
\Biggr\| 
\end{displaymath}
\end{inout}


\begin{inout}
\[ \left\| 
 \left(\int f(x) dx\right) 
 \Bigg/ \left(\int g(x) dx\right) 
 \right\| \]
\end{inout}
片方だけに区切り記号があれば良いときはピリオド`{\str.}'で
いずれかの記号を省略できます.
\begin{inout}
\[ \left( \left\uparrow 
   \int f(x)dx + \int g(x)dx
   \right. \right) \] 
\end{inout}

\subsection{行列}

{\LaTeX}における行列は\Env{array}環境中に
記述します.\env{array}環境はそのままでは
数式にはならず\env{math}環境や\verb+\[\]+の
中に入れたり\verb+$$+の中に入れてあげます.
\env{array}環境の基本的な使い方は
\begin{usage}
\begin{array}[$\<揃える位置>$]{$\<列指定>$}
$\begin{array}{cccccc} \text{成分}_{11} &\verb+&+ & \dots &\verb+&+ & \text{成分}_{1n}  & \verb+\\+\\\vdots &\verb+&+ & \ddots&\verb+&+ & \vdots  & \verb+\\+\\\text{成分}_{m1} &\verb+&+ & \dots &\verb+&+ & \text{成分}_{mn}  & \end{array}$
\end{array}
\end{usage}

というように $m$ 行 $n$ 列の行列を書きます.
\index{"&@\verb+&+!array@\texttt{array}環境の\zdash}%
\glossary{"&@\verb+&+!array@\texttt{array}環境の\zdash}%
ここで\W{アンパサンド}`{\texttt\&'は成分(要素)
の区切りを意味し,`{\texttt{\BS\BS}'は行
の終わりを意味しています.括弧は必要ならば
前述の区切り記号で括ることもできます.表と
行列は基本的に同じ構造で,縦の罫線も横の罫
線も入れることができます.\indindz{罫線}{行列の}

\begin{intext}
\begin{array}{列数と縦罫線の指定}
\end{intext}

この部分では4列あるならば次のようにします.

\begin{intext}
\begin{array}{lc|cr}
\end{intext}

このときの`{\str l}',`{\str c}',
`{\str r}'は行列の中の要素の配置場所を指定するもの
です.真ん中にはテキストバー\qu{\texttt |}があります,
これは縦方向の罫線を表しています.このような記号を
{\W{列指定}}と呼びます.
%\env{array}環境中で指定できる列指定は
%\tabref{math:array}となります.
\indindz{列指定}{行列における}%
\env{array}環境中で指定できる列指定は
\tabref{math:array}となります.
\texttt{array}環境は入れ子にする事も,
行列の中に行列を書いたりする事もできます.
%\texttt{array}環境は入れ子にすることもできます.
%行列の中に行列を書いたりすることもできますので便利です.

\begin{table}[htbp]
\begin{center}
\indindz{幅}{行列の}%
\indindz{長さ}{1列の}%
\zindind{行列}{の幅}%
\caption{\texttt{array}環境の主な列指定}
\tablab{math:array}
\begin{tabular}{cl}
\TR
 \Th{列指定} & \Th{意味}\\
\MR
\str l & 行列の縦1列を左揃えにする\\
\str c & 行列の縦1列を中央揃えにする\\
\str r & 行列の縦1列を右揃えにする\\
\verb+|+ & 縦の罫線を引く\\
\verb+||+ & 縦の2重罫線を引く\\
\str @\param{表現} & 表現を1列追加する\\
\str p\param{長さ} & ある列の幅を直接指定する\\
\str *\param{回数}\param{列指定} &回数分だけ\val{列指定}を繰り返す\\
%\verb+l+ & 行列の縦1列を左揃えにする\\
%\verb+c+ & 行列の縦1列を中央揃えにする\\
%\verb+r+ & 行列の縦1列を右揃えにする\\
%\verb+|+ & 縦の罫線を引く\\
%\verb+||+ & 縦の2重罫線を引く\\
%\verb+@{表現}+& 表現を縦1列追加します\\
%\verb+p{長さ}+& ある列の幅の長さを直接指定します\\
%\verb|*{回数}{項目}|&回数分だけ項目を繰り返す.\\
\BR
\end{tabular}
\end{center}
\end{table}

\begin{inout}
\[ \left( \begin{array}{cc} 
      a & b \\ c & d 
   \end{array} \right) \]
\end{inout}

\index{改行!arrayかんきょうでの@\texttt{array}環境での\zdash}%
横方向に行列が続く場合があるため\env{array}環境の
{最後の行に改行は入れません}.

\begin{inout}
 \[ \left( \begin{array}{cc} 
     a & b \\ c  & d 
    \end{array} \right) 
  \left( \begin{array}{c} 
       m \\ n 
    \end{array} \right)  =
  \left( \begin{array}{c} 
      am+bn \\ cm+dn 
    \end{array} \right) \]
\end{inout}



\subsection{行列に罫線を引く}

水平に罫線などを入れたりするときには \C{hline},
要素の中で縦の罫線を引くときには \C{vline}など
を使います(\tabref{array:lines}).
罫線などの使い方は以下の例を見て理解してください.

\begin{table}[htbp]
\caption{\texttt{array}環境中での罫線の命令}\tablab{array:lines}
\begin{tabular}{ll}
\TR
 \Th{命令} & \Th{意味}\\
\MR
\C{hline}& 
   横に引けるだけの罫線を引きます\\
\cmd{hline}\cmd{hline}&
  引けるだけの2重の\W{横罫線}を引きます\\
\C{vline}& 
   要素の中で引けるだけの縦罫線を引きます\\
\C{cline}\param{範囲}& 
   要素の罫線を行の\val{範囲}を指定して引きます\\
\C{multicolumn}\param{数値} &
   \multirow{2}*{行をつなげて\val{列指定}通りに要素を出力します}\\
\multicolumn{1}{r}{\param{列指定}\param{要素}} & \\
\BR
\end{tabular}
\end{table}

\begin{inout}
\begin{displaymath}
\begin{array}{llc} \hline
\multicolumn{3}{c}{f(x)} \\ \hline
g(x) & h(x) & i(x) \\ \cline{2-2}
j(x)+k(x)+l(x) & o(x) & p(x)\\
\end{array}
\end{displaymath} 
\end{inout}

\begin{inout}
\[  A = \left(  \begin{array}{c|ccc}
    a_{11} &  0 & \cdots & 0\\  \hline
    a_{21} &    & \\
    \vdots &    & B\\
    a_{n1} &    & \\
   \end{array} \right) \] 
\end{inout}

%\subsection{行列の括弧を簡単に記述する}
%
%別の方法として \Person{David}{Carlisle}の \Y{delarray} (\Z{delimiter array})
% パッケージを用いる事もあります.次のようにすると \C{left(} \C{right)} 
%を補った場合と同様の括弧付けになります.
%
%\begin{inout}
%\usepackage{delarray}
%$\begin{array}({cc})
% a_{11} & a_{12} \\
% a_{21} & a_{22} \\
%\end{array}$
%\end{inout}




\subsection{場合分け}
\index{場合分け}%

\AmSLaTeX の \E{cases}環境を説明するので,省略.

\env{array}環境には次に示すような場合分けを行う
使い方もあります.
\begin{inout}
\[ f(x)= \left\{
  \begin{array}{cl}
    x & (x > 0)\\
    0 & (x = 0)\\
   -x & (x < 0) 
  \end{array} 
\right. \] 
\end{inout}

%一つの式から解が複数に{\KY{場合分け}}される場合 \C{cases}
%命令が使えますが\sty{amsmath}の
%\Env{cases}環境のほうがうまく行くでしょう.

%\begin{Syntax}
%\verb|\begin{cases}|\\
%%要素1 \verb|\\| 要素2 \verb|\\| $\ldots$\\
%\val{要素\mbox{$_1$}} \verb|\\| \val{要素\mbox{$_2$}} 
% \verb|\\| $\ldots$ \verb|\\| \val{要素\mbox{$_n$}}\\
%\verb|\end{cases}|
%\end{Syntax}

%\begin{inout}
%\( f(x) = \begin{cases}
% \,x & \quad(x>0)\\ 
% \,0 & \quad(x=0)\\
% \,-x & \quad(x<0)
%  \end{cases} \)
%\end{inout}


%次のように場合分けのときにも使えます.\indindz{括弧}{場合分けの}

%\begin{inout}
%\usepackage{delarray}
%$f(x) = 
%\begin{array}\{{ll}.
% 1  & \mathrm{if}\  x > 0. \\
% 0  & \mathrm{if}\  x = 0. \\
% -1 & \mathrm{if}\  x < 0. \\
%\end{array}$
%\end{inout}

%新たに列指定を宣言して,次のようにもできます.
%\begin{inout}
%\usepackage{delarray}
%\newcolumntype{V}{>{$}l<{$}}
%\begin{displaymath}
% f(x) =
%\begin{array}\{{lV}.
% 1   & if $x > 0$. \\
% 0  & if $x = 0$. \\
% -1 & if $x < 0$. \\
%\end{array}
%\end{displaymath} 
%\end{inout}

\section{定理型環境}

\C{theorem}命令を使うと新規に定義型や定理型の環境を
作成できます.
\begin{usage}
\newtheorem{$\<環境名>$}{$\<見出し>$}[$\<親カウンタ>$]
\newtheorem{$\<環境名>$}[$\<同系の環境名>$]{$\<見出し>$}
\end{usage}

%\begin{Syntax}
%\C{newtheorem}\pa{名前}\pa{ラベル}\opa{親カウンタ}\\
%\C{newtheorem}\pa{名前}\opa{定義済みの環境}\pa{ラベル}
%\end{Syntax}
章や節などを通し番号の前に付けるにはその\val{親カウンタ}を
\tabref{latexcounters}から選びます.別々の環境で同じ
通し番号を使いたい場合は\val{定義済みの環境}を指定します.
具体的な例として \env{Prob}環境と\env{Exe}環境を次のようにプリアンブルに
記述します.

\begin{intext}
\newtheorem{Prob}{問題}[chapter]
\newtheorem{Exe}[Prob]{例題} 
\end{intext}

このようにすれば以下のような『問題』と『例題』の環境を新設できます.

\begin{inout}
\begin{Exe}\label{Hoge:ware}
この文書は難しいか.答えは簡単だ.
\end{Exe}
\begin{Prob}\label{Geho:yueni}
この文書は適切かどうか考えよ.
\end{Prob}
例題~\ref{Hoge:ware}より
問題~\ref{Geho:yueni}が導かれる. 
\end{inout}

実際の出力は異なると思います.\cmd{theorem}命令は
定理型や定義型の環境を作成するために作られたので
日本語用には思うようにカスタマイズできないようです.

\subsection{定理型環境のカスタマイズ}\seclab{theorem}
\indindz{環境}{定理型の}%
\indindz{環境}{問題型の}%
\indindz{環境}{例題型の}%
\Person{Frank}{Mittelbach}が作成した\Sty{theorem}は
{\LaTeX}における \C{theorem}命令を拡張したパッ
ケージです.このパッケージは例えば\yo{定理型}や
\yo{定義型}だけでなく,\yo{問題型}や\yo{例題型}
などの環境を作成するときに満足の行く出力になると
思われます.{\AmSLaTeX}に含まれる\Sty{amsthm}と
%いうパッケージもありますが\Person{Frank}{Mittelbach}が
%作成した\sty{theorem}を使ったほうが便利だと思います.
いうパッケージもありますが\Person{Frank}{Mittelbach}
による\sty{theorem}を使ったほうが便利だと思います.
定理型の環境を新設するときは{\LaTeX}の \cmd{theorem}
命令と同じように環境を新設します.

%\begin{Syntax}
%\C{newtheorem}\pa{環境名}\pa{名前}
%\end{Syntax}

\indindz{カウンタ}{親}%
章などの\Z{親カウンタ}に連動させたい場合は
次のように\val{親カウンタ}を指定します.
%\begin{Syntax}
%\C{newtheorem}\pa{環境名}\pa{名前}\opa{親カウンタ名}
%\end{Syntax}

%同系の環境を作成するときは既存の環境名も指定して定義します.
%\begin{usage}
% 
%\end{usage}
%\begin{Syntax}
%\C{newtheorem}\pa{環境名}\opa{同系の環境名}\pa{名前}
%\end{Syntax}

\sty{theorem}パッケージではさらに
それぞれの定理型環境の書式を以下の命令で変更できます.
\begin{usage}
\theoremstyle{$\<スタイル>$}
\theorembodyfont{$\<書式>$}
\theoremheaderfont{$\<書式>$}
\end{usage}
%\begin{Syntax}
%\C{theoremstyle}\pa{スタイル}\\
%\C{theorembodyfont}\pa{書式}\\
%\C{theoremheaderfont}\pa{書式}
%\end{Syntax}

\val{書式}に対しては書体変更用の宣言型の命令を
使います.\val{スタイル}には以下の六つが使えます.
\begin{description}
\item[\str{plain}] 
   標準の \cmd{theorem}命令と同じ書式にします.
\item[\str{break}] 
   \val{名前}を出力した後に改行をします.
\item[\str{margin}] 
   通し番号を余白に出力します.
\item[\str{change}]
   通し番号と\val{名前}を入れ替えます.
\item[\str{marginbreak}] 
  \qu{\str{margin}}に付け加え,それを出力した後に改行します.
\item[\str{changebreak}]
  \qu{\str{change}}に付け加え,それを出力した後に改行します.
\end{description}
\sty{theorem}パッケージで\yo{例題2.1,参考2.2,問題2.3}の
ような環境を作成したければ次のようにすると良いでしょう.

\begin{intext}
{\theorembodyfont{\normalfont}
\theoremheaderfont{\normalfont\bfseries}
\newtheorem{Exam}{例題}
\newtheorem{Refer}[Exam]{参考}
\newtheorem{Prob}[Exam]{問題}}
\end{intext}

\section{数式記号を自作する}

\subsection{記号の積み重ね}\seclab{stack:math}
\zindind{記号}{の積み重ね}%
イコール\qu{$=$}のうえに\qu{$\mathrm{def}$}をのせて
\qu{$\stackrel{\mathrm{def}}{=}$}のような記号を
出したいときがあります.これには \C{stackrel} という
命令が使えます.一つ目の引数を二つ目の引数のうえに載せて
関係子を作ります.
\begin{usage}
\stackrel{$\<上部の記号>$}{$\<下部の記号>$}
\end{usage}

\begin{inout}
\newcommand{\defeq}{%
   \stackrel{\mathrm{def}}{=}}
\( x \defeq p(t)+q(t)+r(t) \)
\end{inout}


\subsection{複数行の添字を付与する}

記号の積み重ねとは少し違うのですが,次のような
数式を出力するときもあるでしょう.この例では \C{substack}
という\sty{amsmath}パッケージに含まれる
命令を使っています.

\begin{inout}
\begin{displaymath}
 \sum^l_{i=1} \sum^m_{j=1} \sum^n_{k=1} 
 p_i q_j r_k \neq \sum_{
 \substack{i\le 1\le l \\ j\le 1 \le m
 \\ k\le 1 \le n}} p_i q_j r_k
\end{displaymath} 
\end{inout}


\subsection{記号の重ね合わせ}\zindind{記号}{の重ね合わせ}%
二つの記号を重ね合わせて新しい記号を作りたいとき
があります.\cmd{ooalign} と \cmd{crcr}命令を組み合
わせるとうまくできます.
\begin{usage}
{\ooalign{ $\<一つ目の記号>$\crcr$\<二つ目の記号>$}} 
\end{usage}

%\begin{Syntax}
%\verb|{|\C{ooalign}\verb|{|%
% \val{一つ目}\C{crcr}\val{二つ目}\verb|}}|
%\end{Syntax}
二つの記号の内で横幅の広いほうの幅が優先されます.
二つの記号を中心に重ね合わせたいときは \C{hss}という
空白を挿入する命令を使います.さらに文字列に \C{not}%
\zindind{演算子}{の否定}%%
を使っても演算子の否定のようにはなりませんので
次のような定義をしておくと良いでしょう.

\begin{intext}
\newcommand{\cnot}[1]{\ooalign{/\crcr{\hss{#1}\hss}}}
\end{intext}

スラッシュは全角を使っています.

\begin{inout}
\newcommand{\pile}[2]{%
  {\ooalign{#1\crcr#2}}}
\newcommand{\cpile}[2]{{\ooalign{{%
  \hss#1\hss}\crcr{\hss#2\hss}}}}
\newcommand{\cnot}[1]{%
  \ooalign{/\crcr{\hss{#1}\hss}}}
特性数$\pile Y=$は定数$\cpile Y=$の云々で
あり,\cnot{A}は\pile/Aとは別物なのである.
\end{inout}

\subsection{数式の太字}\seclab{bm}

\Y{amsbsy}の \C{boldsymbol}を使うように変更.

\zindind{数式}{の太字}%
何らかの理由である数式の一部や,ある数式全体を
\Z{太字}にする事があるそうです.方法として
\begin{itemize}
 \item \C{mathbf}命令を使う.
 \item \C{boldmath} と \C{unboldmath}を使って太字かどうかを切り替える.
 \item \Sty{amsmath}に含まれる\Sty{amsbsy}パッケージの \C{boldsymbol}命令を使う.
 \item \Sty{bm}パッケージの \C{bm} 命令を使う.
\end{itemize}
などがあります.これは使用している数式書体に
よっては使えない事があります.\Sty{txfonts}や
\Sty{pxfonts}を使うとなんら問題なく出力できます.
一つ目の \cmd{boldmath} と \cmd{unboldmath}は
{数式モード中で使う事ができません}.

\begin{inout}
\(\mathbf{\int^a_b f(x)dx}  \neq\)
\boldmath  \(\int^a_b f(x)dx  \neq\)
\unboldmath\(\int^a_b f(x)dx \)
\end{inout}
\C{mathbf}の場合は\Z{ギリシャ文字}などの特定の記号しか
太字にならないうえにイタリック体ではなくローマン体に
なってしまいます.もう少し局所的に使いたい場合は
\sty{amsbsy}の \C{boldsymbol}を使います.
\begin{inout}
\(\mathbf{\int^a_b f(x)dx}  \neq
\boldsymbol{\int^a_b f(x)dx}\neq
\int^a_b f(x)dx \)
\end{inout}
現在は\sty{amsbsy}を使うよりも\Sty{bm}パッケージの \C{bm}
命令を使うのが良いでしょう.
\begin{inout}
\(\mathbf{\int^a_b f(x)dx} \neq
  \bm{\int^a_b f(x)dx} \neq
  \int^a_b f(x)dx \)
\end{inout}
%結論として \cmd{bm}命令を使うようにすると
%思い通りの結果になるのではないかと思います.

