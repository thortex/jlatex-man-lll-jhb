\section{\Y{fancyhdr}によるヘッダ・フッタの設定}

%\section{ヘッダ・フッタの変更その2\zdash\Y{fancyhdr}}
\indindz{スタイル}{ページ}%
\zindind{ページ}{スタイル}%
{\LaTeX}が標準で用意してくれているページスタイル
では寂しい,そう思う人も多いでしょう.自分で全て
定義する事もできますが,\Person{Piet}{Oostrum}が
作成した \Y{fancyhdr} を使うと比較的簡単にペー
ジスタイルをカスタマイズできます.\sty{fancyhdr}
は\Y{fancyheadings}の後継で,ページのヘッダーと
フッターをカスタマイズできるマクロです.
まず\sty{fancyhdr}を使うために以下をプリアンブルに記述します.

\begin{intext}
\usepackage{fancyhdr} 
\pagestyle{fancy}
\end{intext}

\sty{fancyhdr}では
ヘッダ・フッタを六つに分割しています.
\begin{center}
\begin{tabularx}{.8\linewidth}{|XcX|}
\hline
\cmd{lhead} & \cmd{chead} & \hfill\cmd{rhead}\\
\multicolumn{3}{|c|}{%
   \hrulefill (\C{headrulewidth})\hrulefill} \\   
\multicolumn{3}{|c|}{\val{本文領域}}\\
\multicolumn{3}{|c|}{%
   \hrulefill (\C{footrulewidth})\hrulefill} \\
\cmd{lfoot} & \cmd{cfoot} &\hfill\cmd{rfoot} \\
\hline
\end{tabularx}
\end{center}
\C{lhead},\C{chead},\C{rhead}の
三つはヘッダに使い\C{lfoot},\C{cfoot},
\C{rfoot}の三つはフッタに使います.%
\indindz{罫線}{ヘッダ上部の}\indindz{罫線}{フッタ上部の}%%
\C{headrulewidth}はヘッダ下部の罫線の太さで,
\C{footrulewidth}はフッタ上部の罫線の太さです.
図だけのページや表だけのページはシンプルな
ヘッダ・フッタにしたり,ヘッダ下部の罫線を
引かない場合があります.その場合は \C{headrulewidth} を
\C{iffloatpage}命令で判定するように定義すると良いでしょう.

\begin{intext}
\def\headrulewidth{\iffloatpage{0pt}{.4pt}} 
\end{intext}

\cmd{lhead}などの
他の命令も同様に変更できます.

例えば\cls{jarticle}クラスで次のようなヘッダ・フッタ
\begin{center}
\begin{tabularx}{\linewidth}{|cXcXc|}
\hline
& & \hskip5em & \hfill 資本主義社会の崩壊 & \\
\cline{2-4}
\multicolumn{5}{|c|}{\val{本文領域}}\\
\cline{2-4}
& 名無しの権兵衛 & \hskip3em & \hfill \textbf{ページ番号} & \\
\hline
\end{tabularx}
\end{center}
を設定したければ以下のように記述します.

\begin{intext}
\lhead{} \chead{}
\rhead{資本主義社会の崩壊}
\lfoot{名無しの権兵衛} 
\cfoot{}
\rfoot{\textbf{\thepage}}
\def\headrulewidth{.4pt}
\def\footrulewidth{.4pt}
\end{intext}

書籍用クラス\cls{jbook}などでクラスオプションに
\Option{twoside}が指定されている場合は偶数ページと
奇数ページを個々に設定します.例えば次のような
ヘッダ・フッタにしたいとします.
\begin{center}
\begin{tabular}{|p{7zw}cp{7zw}|}
\hline
第1章\hskip1zw 序論& & \hfill 1.1\hskip1zw 背景\\\hline
\multicolumn{3}{|c|}{本文}\\\hline
& \textbf{2}&\hfill \\\hline
\end{tabular} \hfill
\begin{tabular}{|p{7zw}cp{7zw}|}
\hline
1.2\hskip1zw  目標& & \hfill 第1章\hskip1zw 序論\\\hline
\multicolumn{3}{|c|}{本文}\\\hline
& \textbf{3}& \\\hline
\end{tabular}
\end{center}
ヘッダ・フッタの奇数ページと偶数ページの場所
を区別するために以下のような設定になっています.
\begin{center}
\begin{tabular}{|p{6zw}cp{6zw}|}
\hline
\str{EL(H)}& \str{EC(H)}& \hfill  \str{ER(H)}\\\hline
\multicolumn{3}{|c|}{本文}\\\hline
\str{EL(F)}& \str{EC(F)} &\hfill \str{ER(F)} \\\hline
\multicolumn{3}{c}{偶数ページ側}\\
\end{tabular} \hfill
\begin{tabular}{|p{6zw}cp{6zw}|}
\hline
\str{OL(H)} &\str{OC(H)} & \hfill \str{OR(H)} \\\hline
\multicolumn{3}{|c|}{本文}\\\hline
\str{OL(F)}&\str{OC(F)} & \hfill  \str{OR(F)}\\\hline
\multicolumn{3}{c}{奇数ページ側}\\
\end{tabular}
\end{center}
先程の出力を得るためには \C{fancyhead} と \C{fancyfoot}命令を使います.

\begin{intext}
\documentclass{jbook}
\usepackage{fancyhdr} \pagestyle{fancy}
\fancyhead[ER,OL]{\rightmark}
\fancyhead[EL,OR]{\leftmark}
\fancyfoot[EC,OC]{\textbf{\thepage}}
\end{intext}

欧文のクラスではこれで良いのですが和文では \C{chaptermark} 
と \C{sectionmark}の定義を\sty{fancyhdr}を読み込んだ後に次のように定義
するのが良いでしょう.

\begin{intext}
 \def\chaptermark#1{\markboth{%
   \ifnum \c@secnumdepth >\m@ne
     \if@mainmatter
       \@chapapp\thechapter\@chappos\hskip1zw
     \fi
   \fi
   #1}{}}%
  \def\sectionmark#1{\markright{%
    \ifnum \c@secnumdepth >\z@ \thesection \hskip1zw\fi
    #1}}%
\end{intext}

\subsection{標準設定}
\begin{intext}
\usepackage{fancyhdr} 
\usepackage{fancy}
\fancyhead[LE,RO]{\slshape \rightmark}
\fancyfoot[LE,RE]{\slshape \leftmark}
\renewcommand*{\headrulewidth}{0.4pt}
\renewcommand*{\footrulewidth}{0.4pt}
\end{intext}

標準では見出しが uppercase に変更される.

\begin{usage}
\lhead{\nouppercase{\rightmark}} 
\rhead{\nouppercase{\leftmark}}
\end{usage}

\subsection{chaptermark, sectionmark 再定義}

\begin{intext}
\renewcommand{\chaptermark}[1]{%
   \markboth{\chaptername\ \thechapter.\ #1}{}}
\end{intext}

\subsection{片面印刷時のヘッダ・フッタ設定}

\CIS{lhead,chead,rhead,lfoot,cfoot,rfoot,headrulewidth,footrulewidth}%
\begin{usage}
\lhead{$\<ヘッダ左側>$}
\chead{$\<ヘッダ中央>$}
\rhead{$\<ヘッダ右側>$}
\lfoot{$\<フッタ左側>$}
\cfoot{$\<フッタ中央>$}
\rfoot{$\<フッタ右側>$}
\end{usage}

\subsection{両面印刷時のヘッダ・フッタ設定}
\begin{usage}
\fancyhead{} % to clear all fields
\fancyhead[$\<位置指定>$]{$\<ヘッダ要素>$}
\fancyfoot[$\<位置指定>$]{$\<フッタ要素>$}
\end{usage}

\subsection{罫線}
\begin{usage}
\renewcommand*{\headrulewidth}{$\<ヘッダ下部の罫線の太さ>$}
\renewcommand*{\footrulewidth}{$\<フッタ上部の罫線の太さ>$}
\end{usage}

\subsection{ページ下部中央に「---\protect\val{ページ番号}---」だけ出力する}

\begin{usage}
\fancypagestyle{plain} 
\fancyhf{}
\fancyfoot[c]{---\bfseries\thepage---}
\renewcommand{\headrulewidth}{0pt}
\renewcommand{\footrulewidth}{0pt}
\end{usage}

\subsection{ヘッダ・フッタを2行にする}

\begin{intext}
\addtolength{\headheight}{\baseilneskip}
\renewcommand{\sectionmark}{\markboth{#1}{}}
\renewcommand{\subsectionmark}{\markright{#1}}
\rhead{\leftmark\\ \rightmark}
\end{intext}


\subsection{総ページ数を追加する}

ヘッダ・フッタの設定をフッタだけに「ページ/総ページ」にしたい場合がある
でしょう.これは例えば次のようにプリアンブルに記述します.

\begin{usage}
\usepackage{lastpage}
\cfoot{\thepage/\pageref{LastPage}}
\end{usage}

%この場合は \cmd{ps@total}によって新規に\qu{\str{total}}という
%ページスタイルを定義しています.ページ番号の書体を
%変えたいときは \cmd{normalfont}を \cmd{bfseries}などに
%変更します.

