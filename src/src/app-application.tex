%#!platex -kanji=utf8 hb.tex
\chapter{この冊子の編集者向け情報}% For Authors
% これはただのダミーテキスト.
% 文字コードを判定するための意味のない文字列.
% これくらい記述すれば大丈夫かな.
% 付録のくせに生意気な.
% 付録の分際で.

この冊子は完成形ではありません.ソースを公開しているため,ユーザ自身で
編集する事も可能ですし,著者に修正の差分(パッチ)を送付し,それを適用す
る事も可能です.書籍の執筆といえば業績と肩書きのある方々が常人には計り知
れない力を発揮し完成させるものと思っておりました.しかし,Wikipediaのよ
うに,現在は世界中の著者が一つの辞書を作成するという新しい執筆スタイルが
誕生しています.書籍もこれまでにない新しい執筆スタイルを模索する時代が
来ているのではないかと考えています.

\section{図の作成}
初期の段階ではMac~OS~XのOmniGraffle~Proでほとんどの図を作成していました.
これは中身がXML形式の透過的な画像形式です.
DTD も Apple の preference list に準拠するように作成されています.
\begin{quote}
   \url{http://www.apple.com/DTDs/PropertyList-1.0.dtd}
\end{quote}
しかし,OmniGraffle を持っていないと編集できませんので,SVG形式でも
画像ファイルを提供します.

図を作成するときの書体は以下を使用するのが望ましいでしょう.
\begin{description}
 \item[ローマン体]    
   Palatino
 \item[サンセリフ体]
   Helvetica
 \item[タイプライタ体]
   Courier (Monospaced)
 \item[数式]
  Palatino
\end{description}
Mac OS X には Helvetica Neue と Courier New が含まれていますが,
Neue/Newのウェイトは\LaTeX のPXfontsと親和性が高くない気がしますので,
使わない方が良いかもしれません.

%\section{マクロ}
%使えるマクロ・パッケージは何でも使います.この冊子構成ファイルを叩き台に
%書籍を執筆する人もいるかもしれませんので,自作マクロなどはなるべく避ける
%ようにしたいです.

