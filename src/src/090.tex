%#!platex -kanji=utf8 hb.tex
\chapter{図表の構成}
% これはただのダミーテキスト.
% 文字コードを判定するための意味のない文字列.
% これくらい記述すれば大丈夫かな.
% Emacs のくせに生意気な.
% Emacs の分際で自動判別とか.
% Mac OS X のテキストエディッタの文字コード自動判別はうまくいかないぞ.

\section{}

\section{フロート(浮動体)}

\subsection{図に通し番号を付けてフロートさせる}

\subsection{表に通し番号を付けてフロートさせる}
レポートや論文においては基本的に図表に対して通し番号を振るために,図表は
\Env{table}環境か\Env{figure}環境に入れ子にします.この場合,{\LaTeX}で
は図表を\KY{浮動体} (\Z{float}) と呼ばれる場所に一度退避させ,最適な位置
に図表を配置しようと試みます\pp{\tabref{floats}}.浮動体として退避さ
せた図表は少し制限の多い条件で組版されます.
\begin{table}%[6]{r}{20zw}
 \centering
 \caption{浮動体の種類}\tablab{floats}
 \begin{tabular}{lcc} 
  \TR
  & \Th{表}          & \Th{図}          \\ 
  \MR
  入れる環境 & \Env{table}環境 & \Env{figure}環境 \\ 
  見出しの位置 & 表の上部 & 図の下部 \\
  \BR
 \end{tabular}
\end{table}

\indindz{番号}{図表の通し}\indindz{番号}{表の通し}\indindz{番号}{図の通し}%
表 (\Z{table}) は\Env{tabular}環境で作成し,番号付けしたければ
\Env{table}環境中に入れ子にします.
%
図 (\Z{figure}) は\Env{picture}環境や\indindz{ファイル}{画像}%
画像ファイルを指定し,番号付
けしたければ\Env{figure}環境中に入れ子にします.このようにするとそれ
らの図表は浮動体として扱われます.レポートや論文では図表に通し番号
を付けるのは必須ですから,全ての表は\env{table}環境の中へ,図は
\env{figure}環境の中に入れるのが良いでしょう.
\indindz{画像}{浮動体としての}%


\zindind{ページ}{の最下部}%
\zindind{ページ}{の最上部}%
図表を挿入するときに指定するのはその配置場所です.基本的に{\LaTeX}は
図表をページの最上部か最下部に配置しようとして,それでも無理なときは
別ページへと出力します.ユーザはこれら図表(浮動体)の配置場所を指定する
事ができます.指定できる場所は\tabref{hudoutai}となります.位
置指定は複数指定する事が可能です.

\begin{table}
 \begin{center}
  \caption{浮動体の位置指定}\tablab{hudoutai}
  \begin{tabular}{cl}
   \TR
   \Th{記号} & \Th{浮動体の配置する場所} \\
   \MR
   \str h & まさにその場所に配置しようと試みます\\
   \str t & ページ上部に配置しようと試みます\\
   \str b & ページ下部に配置しようと試みます\\
   \str p & 浮動体を別ページに配置しようと試みます\\
   \str ! & 無理やりその場所に配置します\\
   \BR
  \end{tabular}
 \end{center}
\end{table}
これらの位置指定は\env{table}環境や\env{figure}環境の
{任意引数として渡します}.\env{figure}環境で例を示すと
次のように使います.

\begin{intext}
\begin{figure}[htbp]
ここに図が入ります.
\end{figure}
\end{intext}

図表用の見出しを出力するには \C{caption}命令を
\E{figure}/\E{table}環境中で使用します.
\begin{usage}
\caption[$\<図表目次用見出し>$]{$\<見出し>$}
\label{$\<ラベル>$}
\end{usage}
%\begin{Syntax}
%\C{caption}\pa{図表見出し}\C{label}\pa{ラベル}
%\end{Syntax}
前述のように表見出しは表の上部に出力するために,\C{caption}
命令を先に,図見出しの場合は,図の後に \C{caption}を先に
記述します.

\env{figure}環境中に表を入れたり,\env{table}環境中に
図を入れたりする事ができます.他にも環境中に文字列
を挿入する事も可能です.

一般的なレポート・論文等においては
図表を文章中で参照するときは「上の図は何々」や「前述の図は何々」
と参照してはいけません.必ず\K{付加した通し番号}で「図~3.8は
何々」として \C{ref}命令で参照します.そのためには \C{label}命令で
ラベルを付け加える事になります.間違っても{手動で図表の番号を
書かないで下さい}.%番号の参照は \LaTeX の\secref{xr}を参照して,






\subsection{図表を末尾に移動する (\Y{endfloat})}


%\section{Gnuplot}

%\subsection{Gnuplotのグラフを\LaTeX に張り込む}




\subsection{『表3.8 見出し』を『表3.8: 見出し』に変更する}
\C{figurename}
\C{tablename}
を再定義して '表: ' としたい.
% \makeatletter
\begin{intext}
\def\fnum@figure{\figurename\nobreak\thefigure:\space}
\end{intext}
% makeatother

% \makeatletter
\begin{intext}
\def\fnum@table{\tablename\nobreak\thetable:\space} 
\end{intext}
% makeatother

\subsection{指定した位置に図を配置する}

\begin{enumerate}
 \item オプションを htbp から指定する.
 \item 少し早めの位置にフロートを記述する.
 \item パラメータをいじる.
 \item `!'を付ける.
 \item here.sty を使う.逆にページの境目の調整は手動になる.
 
\end{enumerate}

\subsection{テキストを図に回り込ませる(枠開けを行う)}

\Y{wrapfig}

\subsection{longtableのキャプションが『図』orz}

\C{fnum@table}を適切に定義しても勝手に`:'が追加されてしまいます.

\begin{intext}
\def\LT@makecaption#1#2#3{%
  \LT@mcol\LT@cols c{\hbox to\z@{\hss\parbox[t]\LTcapwidth{%
    \sbox\@tempboxa{#1{#2: }#3}%
    \ifdim\wd\@tempboxa>\hsize
      #1{#2: }#3%
    \else
      \hbox to\hsize{\hfil\box\@tempboxa\hfil}%
    \fi
    \endgraf\vskip\baselineskip}%
  \hss}}}
\end{intext}

\sty{jsclasses}

\begin{intext}
\def\LT@makecaption#1#2#3{%
  \LT@mcol\LT@cols c{\hbox to\z@%
  {\hss\parbox[t]\LTcapwidth{%
    \sbox\@tempboxa{#1{#2\hskip1zw\relax}#3}%
    \ifdim\wd\@tempboxa>\hsize
      #1{#2\hskip1zw\relax}#3%
    \else%
      \hbox to\hsize{\hfil\box\@tempboxa\hfil}%
    \fi%
    \endgraf\vskip\belowcaptionskip}%
  \hss}}%
}% 
\end{intext}
